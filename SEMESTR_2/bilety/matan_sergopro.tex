\documentclass[a4paper]{article}

%plastikovye pakety

\usepackage[utf8]{inputenc}
\usepackage[unicode, pdftex]{hyperref}
\usepackage{cmap}
\usepackage{mathtext}
\usepackage{multicol}
\setlength{\columnsep}{1cm}
\usepackage[T2A]{fontenc}
\usepackage[english,russian]{babel}
\usepackage{amsmath,amsfonts,amssymb,amsthm,mathtools}
\usepackage{icomma}
\usepackage{euscript}
\usepackage{mathrsfs}
\usepackage{geometry}
\usepackage{imakeidx}
\usepackage{graphicx}
\graphicspath{{pictures/}}
\DeclareGraphicsExtensions{.pdf,.png,.jpg}
\usepackage[usenames]{color}
\hypersetup{
     colorlinks=true,
     linkcolor=magenta,
     filecolor=magenta,
     citecolor=black,      
     urlcolor=magenta,
     }
\usepackage{fancyhdr}
\pagestyle{fancy} 
\fancyhead{} 
\fancyhead[LE,RO]{\thepage} 
\fancyhead[CO]{\hyperlink{uk}{к списку объектов}}
\fancyhead[LO]{\hyperlink{sod}{к содержанию}} 
\fancyfoot{}
\newtheoremstyle{indented}{0 pt}{0 pt}{\itshape}{}{\bfseries}{. }{0 em}{ }

\renewcommand\thesection{}
\renewcommand\thesubsection{}

%\geometry{verbose,a4paper,tmargin=2cm,bmargin=2cm,lmargin=2.5cm,rmargin=1.5cm}

\title{Матанализ 2 семестр билеты}
\author{Петров Сергей (@psn2706) \\ 
    при поддержке Кабашного Ивана (@keba4ok),\\ 
    Горбунова Леонида, Савельева Артёма, \\ 
    и под конец знатно напизженно из \\ 
    конспекта М. Опанасенко}
\date{25 июня 2021г.}



%envirnoments
    \theoremstyle{indented}
    \newtheorem{theorem}{Теорема}
    \newtheorem{lemma}{Лемма}
    \newtheorem{alg}{Алгоритм}

    \theoremstyle{definition} 
    \newtheorem{defn}{Определение}
    \newtheorem{exl}{Пример(ы)}
    \newtheorem{prob}{Задача}

    \theoremstyle{remark} 
    \newtheorem{remark}{Примечание}
    \newtheorem{cons}{Следствие}
    \newtheorem{exer}{Упражнение}
    \newtheorem{stat}{Утверждение}
    \newtheorem{remind}{Напоминание}

%declarations
        %arrows_shorten
            \DeclareMathOperator{\la}{\leftarrow}
            \DeclareMathOperator{\ra}{\rightarrow}
            \DeclareMathOperator{\lra}{\leftrightarrow}
            \DeclareMathOperator{\llra}{\longleftrightarrow}
            \DeclareMathOperator{\La}{\Leftarrow}
            \DeclareMathOperator{\Ra}{\Rightarrow}
            \DeclareMathOperator{\Lra}{\Leftrightarrow}
            \DeclareMathOperator{\Llra}{\Longleftrightarrow}

        %letters_different
            \DeclareMathOperator{\CC}{\mathbb{C}}
            \DeclareMathOperator{\ZZ}{\mathbb{Z}}
            \DeclareMathOperator{\RR}{\mathbb{R}}
            \DeclareMathOperator{\NN}{\mathbb{N}}
            \DeclareMathOperator{\HH}{\mathbb{H}}
            \DeclareMathOperator{\LL}{\mathscr{L}}
            \DeclareMathOperator{\KK}{\mathscr{K}}
            \DeclareMathOperator{\GA}{\mathfrak{A}}
            \DeclareMathOperator{\GB}{\mathfrak{B}}
            \DeclareMathOperator{\GC}{\mathfrak{C}}
            \DeclareMathOperator{\GD}{\mathfrak{D}}
            \DeclareMathOperator{\GN}{\mathfrak{N}}
            \DeclareMathOperator{\Rho}{\mathcal{P}}
            \DeclareMathOperator{\FF}{\mathcal{F}}
            \DeclareMathOperator{\s}{\sigma}

        %common_shit
            \DeclareMathOperator{\Ker}{Ker}
            \DeclareMathOperator{\Frac}{Frac}
            \DeclareMathOperator{\Imf}{Im}
            \DeclareMathOperator{\cont}{cont}
            \DeclareMathOperator{\id}{id}
            \DeclareMathOperator{\ev}{ev}
            \DeclareMathOperator{\lcm}{lcm}
            \DeclareMathOperator{\chard}{char}
            \DeclareMathOperator{\codim}{codim}
            \DeclareMathOperator{\rank}{rank}
            \DeclareMathOperator{\ord}{ord}
            \DeclareMathOperator{\End}{End}
            \DeclareMathOperator{\Ann}{Ann}
            \DeclareMathOperator{\Real}{Re}
            \DeclareMathOperator{\Res}{Res}
            \DeclareMathOperator{\Rad}{Rad}
            \DeclareMathOperator{\disc}{disc}
            \DeclareMathOperator{\rk}{rk}
            \DeclareMathOperator{\const}{const}
            \DeclareMathOperator{\grad}{grad}
            \DeclareMathOperator{\Aff}{Aff}
            \DeclareMathOperator{\Lin}{Lin}
            \DeclareMathOperator{\Prf}{Pr}
            \DeclareMathOperator{\Iso}{Iso}
            \DeclareMathOperator{\norm}{||}

        %specific_shit
            \DeclareMathOperator{\Tors}{Tors}
            \DeclareMathOperator{\form}{Form}
            \DeclareMathOperator{\Pred}{Pred}
            \DeclareMathOperator{\Func}{Func}
            \DeclareMathOperator{\Const}{Const}
            \DeclareMathOperator{\Arity}{arity}
            \DeclareMathOperator{\Aut}{Aut}
            \DeclareMathOperator{\Var}{Var}
            \DeclareMathOperator{\Term}{Term}
            \DeclareMathOperator{\sub}{sub}
            \DeclareMathOperator{\Sub}{Sub}
            \DeclareMathOperator{\Atom}{Atom}
            \DeclareMathOperator{\FV}{FV}
            \DeclareMathOperator{\Sent}{Sent}
            \DeclareMathOperator{\Th}{Th}
            \DeclareMathOperator{\supp}{supp}
            \DeclareMathOperator{\Eq}{Eq}
            \DeclareMathOperator{\Prop}{Prop}


%env_shortens_from_hirsh            
    \newcommand{\bex}{\begin{example}\rm}
    \newcommand{\eex}{\end{example}}
    \newcommand{\ba}{\begin{algorithm}\rm}
    \newcommand{\ea}{\end{algorithm}}
    \newcommand{\bea}{\begin{eqnarray*}}
    \newcommand{\eea}{\end{eqnarray*}}
    \newcommand{\be}{\begin{eqnarray}}
    \newcommand{\ee}{\end{eqnarray}}
    \newcommand{\Abs}[1]{\lvert#1\rvert}
    

\begin{document}

%commandy_sergeya

\newcommand{\htarg}[2]{\hypertarget{#1}{\textcolor{magenta}{#2}}}
\newcommand{\hlink}[2]{\hyperlink{#1}{#2}}
\newcommand{\hindex}[2]{\index{#2@\protect\hyperlink{#1}{#2}}}
\newcommand{\Add}[2]
{
    \hypertarget{#1}{\textcolor{magenta}{#2}}
    \index{#2@\protect\hyperlink{#1}{#2}}
}
\newcommand{\df}[2]{\frac{\delta #1}{\delta #2}}

%ya_ebanutyi

\newcommand{\resetexlcounters}{%
  \setcounter{exl}{0}%
} 

\newcommand{\resetremarkcounters}{%
  \setcounter{remark}{0}%
} 

\newcommand{\reseconscounters}{%
  \setcounter{cons}{0}%
} 

\newcommand{\resetall}{%
    \resetexlcounters
    \resetremarkcounters
    \reseconscounters%
}

\newcommand{\cursed}[1]{\textit{\textcolor{magenta}{#1}}}
\newcommand{\de}[3][2]{\hypertarget{#2}{\cursed{#3}}}

\maketitle 

\newpage

\hypertarget{sod}{Содержательное содержание.}

\tableofcontents

\newpage

%main_content

\section{Билеты}

\subsection{Билет 1. Функции ограниченной вариации. Свойства. Замена переменной. Примеры.}

\begin{defn}
    \de{1}{Вариация функции}
    $f: \mathbb{R} \to \mathbb{R}^m$
    \[
        V_f([a,b])=\sup\limits_{
            \begin{smallmatrix}
                x_0 \leq x_1 \leq \dots \leq x_n \\
                x_0=a, x_n=b
            \end{smallmatrix}
        }
        \sum\limits_{k=0}^{n-1} |f(x_{k+1})-f(x_k)|
    \]
\end{defn}

\begin{remark}
    Свойства вариации
    \begin{itemize}
        \item $f: \mathbb{R}\to\mathbb{R}, f$ монотонна $\Rightarrow V_f([a,b])=|f(a)-f(b)|$
        \item $V_f([a,b])=0 \Leftrightarrow f$ константо на $[a,b]$
        \item $V_{f+g} \leq V_f + V_g$
        \item $V_f$ аддитивна по промежутку: \\ $a\leq b\leq c: V_f([a,c])=V_f([a,b])+V_f([b,c])$
    \end{itemize}
\end{remark}

\begin{remark}
    Будем говорить, что $f$ имеет \de{2}{ограниченную вариацию} на $[a,b]$, если $V_f$ конечна на $[a,b]$. 
\end{remark}

\begin{stat}
    Для $f: \mathbb{R}\to\mathbb{R}$ следующие утверждения эквивалентны
    \begin{itemize}
        \item $f$ имеет ограниченную вариацию на $[a,b]$
        \item $f=f_1-f_2$, для каких-то  $f_1, f_2$ неубывающих на $[a,b]$
    \end{itemize}
\end{stat}

\begin{proof} \ 

    $\Ra$ Рассмотрим $\varphi(x) = V_f([a,x]) \Rightarrow \varphi \nearrow$ 
    $f = \varphi - (\varphi-f)$, пусть $h=\varphi-f$. $h\nearrow \Leftrightarrow$ при $x\leq y$: $h(x)\leq h(y) \Leftrightarrow \varphi(x)-f(x) \leq \varphi(y)-f(y) \Leftrightarrow f(y)-f(x) \leq \varphi(y) - \varphi(x) = V_f([x,y])$. \\

    $\La$ $V_{f_1-f_2}[a,b] \leq V_{f_1}[a,b] + V_{-f_2}[a,b] = |f_1(a)-f_1(b)|+|f_2(a)-f_2(b)|$.
\end{proof}

\begin{stat}
    \de{3}{Замена переменной в вариации}. 
    Пусть $g: [a,b] \to [c,d]$ непрерывная биекция, тогда $V_f[c,d]=V_{f\circ g}[a,b]$
\end{stat}

\begin{proof}
    $g$ монотонна, будем считать, что возрастает. Любому набору $x_0, x_1, \dots, x_n$ из определения вариации $V_f$ найдутся соответсвующие  $y_0, y_1, \dots, y_n$,
    удовлетворяющие условию $g(y_k)=x_k$ и подходящие для подстановки в определение $V_{f\circ g}$, т.к. $y_k \nearrow \Leftrightarrow g(y_k) \nearrow$. Тогда $V_f[c,d]\leq V_{f\circ g}[a,b]$,
    но с другой стороны $V_f[c,d]\geq V_{f\circ g}[a,b]$, т.к. можно подставлять $x_k:=g(y_k)$ в определение первой вариации.
\end{proof}



\subsection{Билет 2. Естественная параметризация. Гладкие пути. Длина гладкого пути.}

\begin{defn}
    Множество в $\mathbb{R}^n$ называют \de{4}{кривой}, если оно является образом некоторой непрерывной функции $f: (a,b)\to \mathbb{R}^n$. Дуга кривой (или же путь) - подмножество кривой $f: [c,d] \to \mathbb{R}^n$.
\end{defn}

\begin{defn}
    \de{5}{Длина дуги кривой (пути)} $f: [a,b]\to \mathbb{R}^n$ это - $V_f([a,b])$.
\end{defn}

\begin{remark}
    Если длина пути конечна, то путь называется спрямляемым, иначе - неспрямляемым.
\end{remark}

\begin{defn}
    \de{6}{Естественная параметризация кривой} - параметризация длиной её дуги, отсчитываемой от фиксированной точки.
\end{defn}

\begin{remark}
    Естественная параметризация - параметризация, которая ''равномерна по времени'', т.е. за одинаковый промежуток времени проходим одинаковое расстояние.
\end{remark}

Естественная параметризация спрямляемого пути. $\varphi:[a,b]\to[0,\beta], \varphi(x)=V_f([a,x])$, если $\varphi$ строго возрастает (путь $"$без остановок$"$, $f\not\equiv const$ ни на каком интервале),
то $\varphi$ - биекция и $\exists \psi: [0,\beta]\to[a,b], \psi=\varphi^{-1}, V_f([a,b])=\varphi(b)-\varphi(a)$ , 
$V_{f\circ \psi}([c,d])=V_f([\psi(c),\psi(d)]) = \varphi(\psi(d))-\varphi(\psi(c))=d-c$

\begin{defn}
    \de{7}{Гладкий путь} - образ гладкой $f: [a,b] \to \mathbb{R}^n$ (т.е. $f=(f_1,...,f_n)$, причём все $f_k$ непрерывно дифференцируемы).
\end{defn}

\begin{remind}
    \de{8}{Формула Лагранжа}, $f: [a,b] \to \mathbb{R}$ дифференцируема $\Rightarrow \exists \xi \in (a,b):$
    $$f'(\xi) = \frac{f(a)-f(b)}{a-b}$$
\end{remind}

\begin{stat}
    \de{9}{Длина гладкого пути} 
    $f=(f_1, ..., f_n)$ равна $\displaystyle \int_{a}^{b}{\sqrt{\sum\limits_{m=1}^{n}(f'_m(x))^2}} dx$
\end{stat}

\begin{proof}
    Рассмотрим $V_f([a,b]) = \displaystyle \sup_{x_0=a, ..., x_N = b} \sum\limits_{k=0}^{N-1} |f(x_{k+1})-f(x)|$ и 
    воспользуемся формулой Лагранжа $\sum\limits_{k=0}^{N-1} \sqrt{\sum\limits_{m=1}^{n}(f_m(x_{k+1})-f_m(x_k))^2} = $
    $\sum\limits_{k=0}^{N-1} (x_{k+1}-x_k) \sqrt{\sum\limits_{m=1}^{n} (f'_m(\xi_{m,k}))^2}=(V),$
    $\ \xi_{m,k} \in (x_k,x_{k+1}), (f'_m)^2$ равномерно непрерывна на $[x_k, x_{k+1}]$, значит для любого $\varepsilon > 0$ 
    существует достаточно малое разбиение $[a,b]$ такое, что 
    $(f'_m)^2(\xi_{m,k}) \leq \min_{[x_k, x_{k+1}]} (f'_m)^2 + \varepsilon^2$ . Тогда
    \[  (I) \leq (V) \leq \underbrace{\sum\limits_{k=0}^{N-1} (x_{k+1}-x_k) \sqrt{\sum\limits_{m=1}^{n} \min_{[x_k, x_{k+1}]} (f'_m)^2}}_{(I)} 
    +  \varepsilon \sqrt{n} \cdot \underbrace{(b-a)}_{\sum\limits_{k=0}^{N-1} (x_{k+1}-x_k)} \]
    Левая и правая части стремятся к интегралу из условия (при стремлении 
    мелкости к нулю), тогда по \href{https://ru.wikipedia.org/wiki/%D0%A2%D0%B5%D0%BE%D1%80%D0%B5%D0%BC%D0%B0_%D0%BE_%D0%B4%D0%B2%D1%83%D1%85_%D0%BC%D0%B8%D0%BB%D0%B8%D1%86%D0%B8%D0%BE%D0%BD%D0%B5%D1%80%D0%B0%D1%85}
    {теореме о двух миллиционерах} туда же стремится и $(V)$.
\end{proof}

Естественная параметризация гладкого пути. 

$$\displaystyle \varphi(x)=V_f([a,x])=\int_{a}^{x} |f'(t)| dt = \int_{a}^{x}{\sqrt{\sum\limits_{k=1}^{n}(f'_k(t))^2}} dt.$$
Параметризация всё также по длине дуги $\psi = \varphi^{-1}$. 
\begin{stat}
    $|(f(\psi(x)))'|=1$
\end{stat}

\begin{proof}
    $\varphi'(x)=|f'(x)|, \psi'(x)=\frac{1}{\varphi'(\psi(x))}=\frac{1}{|f'(\psi(x))|}$ , $|(f(\psi(x)))'|=|f'(\psi(x))\cdot\psi'(x)|=1$
\end{proof}

%3
\subsection{Билет 3. Движение по окружности. Единственность простого вращения.}


Единичная окружность описывается уравнением $x^2+y^2=1$.
Хотим обойти её с единичной скоростью, начиная с точки (1,0).

Комплексные обозначения: рассмотрим биекцию $\mathbb{R}^2$ с $\mathbb{C}$ по правилу:
$(x,y) \leftrightarrow (x+iy)$. Тогда путь можно рассматривать как отображение из $\mathbb{R}$ в $\mathbb{C}$.

\begin{defn}
    \de{10}{Простое вращение}
    по окружности это отображение \\
    $\Gamma: \mathbb{R} \to \pi=\{ z\in \mathbb{C} \big| |z| = 1 \} = \{ z\in \mathbb{C} \big| x,y\in \mathbb{R}, x^2+y^2=1, z=x+iy \}$,
    \begin{itemize}
        \item $\Gamma \in C^1$ (гладкая)
        \item $\Gamma(0)=1, \Gamma'(0)=i$
        \item $|\Gamma'(t)|=1$ для любого $t$
    \end{itemize}
\end{defn}

\begin{lemma}
    $\Gamma'(t)\equiv i\Gamma(t)$
\end{lemma}

\begin{proof}
    $\Gamma(t)\in\pi \Rightarrow |\Gamma(t)|=\Gamma(t)\overline{\Gamma(t)} = 1$\\
    $\Rightarrow (\Gamma(t)\overline{\Gamma(t)})' = \Gamma'(t)\overline{\Gamma(t)}+\Gamma(t)\overline{\Gamma'(t)} = 0$\\
    $\Rightarrow 2\Re(\Gamma'(t)\overline{\Gamma(t)})= 0,$
    $|\Gamma'(t)|=|\overline{\Gamma(t)}|=1$ и $\Gamma'(0)\overline{\Gamma(0)}=i$
    $\Rightarrow \Gamma'(t)\overline{\Gamma(t)} \equiv i$
\end{proof}

\begin{stat}
    Если $\Gamma$ существует, то оно единственно.
\end{stat}

\begin{proof}
    Пусть $\Gamma_1, \Gamma_2$ - простые вращения, тогда по лемме\\
    $(\Gamma_1 \overline{\Gamma_2})'=\Gamma_1' \overline{\Gamma_2}+\Gamma_1 \overline{\Gamma_2'}$
    $= i \Gamma_1 \overline{\Gamma_2} + \Gamma_1 \overline{i\Gamma_2} = 0$
    $\Rightarrow \Gamma_1 \overline{\Gamma_2} = const, \Gamma_1(0) \overline{\Gamma_2(0)}=1$\\
    $\displaystyle \Rightarrow \Gamma_1 \overline{\Gamma_2} = 1 \Rightarrow \Gamma_1 = \frac{1}{\overline{\Gamma_2}} = \frac{\Gamma_2}{|\Gamma_2|} = \Gamma_2$
\end{proof}

%4
\subsection{Билет 4. Построения простого вращения. Тригонометрические функции. Свойства. Формула Эйлера.}


\begin{stat}
    \hlink{rotation}{Простое вращение} $\Gamma(t)$ существует.
\end{stat}

\begin{proof}
    Докажем теперь существование. Предъявим сначала произвольную параметрицацию окружности, а затем постараемся сделать в ней замену переменной, чтобы получить хорошую функцию (которая должна быть, конечно, гладкой). Давайте параметризуем верхнюю половину $\mathbb{T}$ самым естественным образом: примем $x=t, \: y=\sqrt{1-t^2}, \ -1\leq t\leq 1$ (двигаемся по часовой стрелке). Теперь нам нужно отпараметризовать нижнюю половину, возьмём для этого $x=-t, \: y=-\sqrt{1-t^2}, \: -1\leq t\leq 1$, двигаться мы теперь будем по нижней половине, но в другом направлении, то есть, одну из половин нужно перевернутьт и ''склеить'' в один целостный проход. Тогда в нижней половине ''сдвинем'' рассмотрение на $1\leq t\leq 3$, и преобразуем: $y=-\sqrt{1-(2-t)^2}$. \ 

    Осталось проверить, что полученная функция гладкая. Вообще, это почти везде очевидно, кроме $\pm 1$, это и проверим. $f(t)=(t, \sqrt{1-t^2})$, а вектор $f'(t)=(1, \frac{-t}{\sqrt{1-t^2}})$. Функция $\varphi(x)$ на $(-1, 1)$ выглядит как
    \[
        \int_{-1}^x|f'(s)|ds=\int_{-1}^x\sqrt{1+\frac{t^2}{1-t^2}}dt=\int_{-1}^x\frac{dt}{\sqrt{1-t^2}}.
    \]
    Функция $\varphi(x)$ - возрастающая биекция, значит, мы можем посмотреть на обратную функцию $\psi(x)=\varphi^{-1}(x)$. Рассмотрим теперь для $x\in(-1, 1)$,
    \[
         (f^{-1}(\psi(x)))'=(f_1'(\psi(x))\psi'(x), f_2'(\psi(x))\psi'(x)).
    \]
    Тогда, так как $\psi'(x)=\frac{1}{\varphi'(\varphi(x))}$, это также и равно $\sqrt{1-\psi^2(x)}$, что также равно 
    \[
    (\psi'(x), \frac{-\psi(x)}{\sqrt{1-\psi(x)}}\sqrt{1-\psi^2(x)}).
    \]
    В последнем также можно сократить числитель и знаменатель. Итого, $f(\psi(x))$ - гладкая на $(-1, 1)$, и более того, если $x\rightarrow\pm 1$, производная имеет конечный предел. Получается, дифференцируема на интервале, и производная имеет предел в крайних точках, тогда она в них также дифференцируема. Таким образом, для верхней половины мы всё показали, для нижней - аналогично, всего лишь с линейной заменой. 
\end{proof}

\begin{defn}
    \[
        \cos(x)=\Real(\Gamma(x)), 
    \]
    \[
        \sin(x)=\Imf(\Gamma(x)).
    \]
\end{defn}

Мы научились строить синус и косинус через вращение окружности. Немного не помню, обговаривали ли мы это на прошлой лекции, но Юрий Сергеевич кратко упомянул, что мы можем разложить $\Gamma(x)$ в ряд Тэйлора в $\sum_{n=0}^\infty\frac{(ix)^n}{n!}$ в силу свойства $\Gamma'(x)=i\Gamma(x)$ и того, что остаточный член в форме Лагранжа будет стремиться к нулю при стремлении $n$ к бесконечности. \ 

Тогда 
\[
    \cos x = \Real \Gamma(x) \Rightarrow \sum_{n=0}^\infty\frac{x^{2n}}{(2n)!}(-1)^n
\]
и аналогично синус по нечётным степеням. [из этого понимаем, что верна \de{11}{формула Эйлера}, то есть, $e^{ix} = \cos(x)+ i \sin(x)$, так как $e^{ix}$ именно таки раскладывается в ряд Тейлора в окрестности нуля] \ 

Мнимая экспонента обладает свойствами, аналогичным обыкновенной экспоненте, поэтому покажем, что $\Gamma(x+y)=\Gamma(x)\Gamma(y)$. Рассмотрим $\Gamma(x+y)\overline{\Gamma(y)}$ - функцию от $x$, а $y$ - параметр. Это - некоторый обход окружности, который также удовлетворяет всем нормировочным условиям. $\varphi(0)=1$, $|\varphi'(x|=1$, и, наконец, $\varphi'(0)=\Gamma'(0)= i$. \ 

Теперь все прекрасные формулы косинуса и синуса суммы и разностей легко выводятся из доказанной формулы. Через мнимую экспоненту запишем: $e^{i(x+y)=e^{ix}\cdot e^{iy}}$, а там уже просто надо посмотреть на мнимые и действительные части. \ 

Из полученных свойств получим, что $\Gamma(x)\Gamma(-x)=\Gamma(0=1)$, тогда $\Gamma(-x)=\overline{\Gamma(x)}$, откуда мы получаем чётность косинуса и нечётность синуса. \ 

Можно упомянуть и формулу муавра. Распишем 
\[
    \cos(x)=\frac{e^{ix}+e^{-ix}}{2}, \: \sin(x)=\frac{e^{ix}-e^{-ix}}{2i}, 
\]
это формулы Муавра. Также можно получить и периодичность, это, вообщем-то очевидно и завершает наш разговор об элементарных функциях. \\


%5 примеры?
\subsection{Билет 5. Дифференцируемость отображений между евклидовыми пространствами. Свойства. Примеры.}


\begin{defn}
    \de{12}{Норма} на евклидовых пространствах - отображение из $\mathbb{R}^n$ в $\mathbb{R}_+$, удовлетворяющее условиям:
    \begin{itemize}
        \item $||x||=0 \Leftrightarrow x=0$
        \item $||\alpha x|| = |\alpha| \cdot ||x||, \forall \alpha \in \mathbb{R}$
        \item $||x+y|| \leq ||x|| + ||y||$
    \end{itemize} 
\end{defn}

\begin{remark}
    Все расстояния мы будем рассматривать с евклидовой нормой
    (т.е. $d(x,y)=||x-y||=\sqrt{\sum\limits_{k=1}^{n}(x_k-y_k)^2}$).
    Такая метрика стандартна. Так как в этом семестре рассматриваемые размерности 
    евлидовых пространств конечные, то с точки зрения сходимостей к нулю мы можем считать 
    разные нормы эквивалентными.
\end{remark}

\begin{remind}
    \de{13}{Модуль (или длина) евклидова вектора} $x=(x_1,x_2, ... , x_n)$:
    \[
        |x|=\sqrt{\sum\limits_{k=1}^{n}x_k^2}
    \]
\end{remind}

\hindex{dif}{Дифференциал}
\hindex{der}{Производная}
\begin{defn}
    $f: \mathbb{R}^n \to \mathbb{R}^m$ дифференцируема в точке $a$, если
    существует линейное отображение $L$, такое что $f(x)=f(a)+L(x-a)+o(||x-a||)$,
    $L$ называют \de{14}{дифференциалом} функции $f$ в точке $a$.
    $L$ определяется матрицей $A$ размера $m \times n$, её столбцы -
    это значения на базисных векторах, $A$ называют \de{15}{производной} функции.
\end{defn}

\begin{remark}
    Запись $f(x)=f(a)+L(x-a)+o(||x-a||)$ означает, что
    \[ \forall \varepsilon > 0 \ \exists \delta: \ \ 0 < ||x-a|| < \delta \Rightarrow \frac{|f(x)-f(a)-L(x-a)|}{||x-a||} < \varepsilon \]
\end{remark}

\begin{remark}
    Если $L$ существует, то оно единственно.
    \begin{proof}
        Пусть $L_1$ и $L_2$ дифференциалы $f$ в точке $a$, тогда \\
        $(L_1-L_2)(x-a)=o(||x-a||)$ при $x\to a$, это возможно только
        если отображение $(L_1-L_2)$ тождественный нуль.\\
        \textit{Пояснение: } пусть $L(x)=o(||x||), x=(x_1,...,x_n)$; $L(x)=\sum\limits_{k=1}^{n} L(x_k e_k) =
        \sum\limits_{k=1}^{n} x_k L(e_k)$, где $e_k$ - базисные вектора. 
        Пусть $\exists k: L(e_k)\not=0 \Rightarrow $ для векторов вида $y=a e_k, a\in\mathbb{R}_{>0}, a \to 0$,
        $\frac{L(y)}{||y||}=\frac{aL(e_k)}{a||e_k||}=\frac{L(e_k)}{||e_k||} \not=0 $, противоречие. 
    \end{proof}
\end{remark}

%6
\subsection{Билет 6. Отделимость линейных отображений от нуля. Норма в пространстве линейных отображений.}


\begin{defn}
    \de{16}{Норма линейного отображения} $L$:
    $$
        ||L|| = \sup_{||x|| \leq 1} ||Lx||
    $$
\end{defn}

\begin{remark}
    Следующие нормы эквивалентны: 
    \begin{itemize}
        \item $||L|| = \sup_{||x|| \leq 1} ||Lx||$
        \item $||L|| = \sup_{||x|| = 1} ||Lx||$
        \item $||L|| = \sup_{||x|| < 1} ||Lx||$
        \item $||L|| = \sup_{||x|| \not= 0} \frac{||Lx||}{||x||}$
    \end{itemize}
    $$$$
\end{remark}

\begin{stat}
    (Линейное отображение липшицево)
    $L: \mathbb{R}^n \to \mathbb{R}^m$ линейно, значит $\exists A: \ \forall x,y \in \mathbb{R}^n: \ ||Lx-Ly|| \leq A||x-y||$
\end{stat}

\begin{remark}
    $||L||=min\{A \big| \   \forall x,y \in \mathbb{R}^n: \ ||Lx-Ly|| \leq A||x-y||\}$
\end{remark}

Важный момент, почему важна операторная норма. Пусть $A:\RR^n\rightarrow \RR^m$, $B:\RR^m\rightarrow \RR^k$, тогда $||BA||\leq ||B||\cdot||A||$, так как левая часть по определению равна $\sup_{||x||\leq 1 (в \RR^n)}\leq \sup_{||y||\leq||A||}||By||\leq ||B||\cdot||A||$. Заметим также две следующие вещи для линейного $A:\RR^n\rightarrow \RR^m$ равносильны: \ 

\begin{itemize}
    \item $\ker A=\{0\}$
    \item $||Ax||\geq \varepsilon ||x||, \exists \varepsilon >0$. 
\end{itemize}

\begin{proof}
    $\{x:||x||=1\}$ - единичная сфера в $\RR^n$. Пусть $f(x):x\rightarrow ||Ax||$, $f$ - непрерывная (?), $f\neq 0$ на единичной сфере, тогда $f\geq \varepsilon >0$, $||Ax||\geq \varepsilon||x||$, $||x||=1$. 
\end{proof}

%7
\subsection{Билет 7. Дифференцирование суммы, произведения, частного.}

%8
\subsection{Билет 8. Дифференцирование суперпозиции функций.}

%9
\subsection{Билет 9. Частные производные. Связь частных производных с дифференцируемостью. Производная по направлению.}


\hindex{partder}{Частная производная}
\begin{defn}
    \de{17}{Частной производной}
    функции $f$ по $i$-ой координате в точке $A=(a_1, a_2, ..., a_n)$ называют предел:
    $$f'_{x_i}(A) = \frac{\delta f}{\delta x_i} (A) = \lim\limits_{x_i \to a_i} \frac{f(a_1, ..., a_{i-1}, x_i, a_{i+1}, ..., a_n) - f(a_1, ..., a_{i-1}, a_i, a_{i+1}, ..., a_n)}{x_i-a_i}$$
\end{defn}

\hindex{smooth}{Гладкость}
\begin{remark}
    Будем называть функцию \de{18}{гладкой}, если все её частные производные непрерывны.
\end{remark}

\begin{remind}
    Неравенство Коши-Буняковского-Шварца (КБШ):
    \[
        \left(\sum\limits_{k=1}^{n}x_k y_k\right)^2 \leq
        \left(\sum\limits_{k=1}^{n}x_k^2\right)
        \left(\sum\limits_{k=1}^{n}y_k^2\right)
    \]
\end{remind}

\begin{theorem}
    Если все частные производные $f: \mathbb{R}^n \to \mathbb{R}^m$ непрерывны в некоторой окрестности точки $x^0$, то $f$ дифференцируема в $x^0$.
\end{theorem}

\begin{proof}
    Рассмотрим случай $m=1$.
    $x^0 = (x_1^0,...,x_n^0)$.
    Применим \hlink{lagrange}{формулу Лагранжа} $f(x)-f(x^0)=f(x_1,...,x_n)-f(x_1^0,...,x_n^0)=$\\
    $\sum\limits_{k=1}^{n} \left(f(x_1^0,...,x_{k-1}^0,x_k,x_{k+1}...,x_n)-f(x_1^0,...,x_{k-1}^0,x_k^0,x_{k+1}...,x_n)\right)=$\\
    $\sum\limits_{k=1}^{n} f'_{x_k}\big|_{(x_1^0,...,x_{k-1}^0,\xi_k,x_{k+1},...,x_n)} (x_k-x_k^0)\overbrace{=}^? \sum\limits_{k=1}^{n} f'_{x_k}\big|_{x^0}(x_k-x_k^0) + o(|x-x^0|)$\\
    $\Leftrightarrow\sum\limits_{k=1}^{n} (f'_{x_k}\big|_{t_k} - f'_{x_k}\big|_{x^0}) (x_k-x_k^0)\overbrace{=}^? o(|x-x^0|)$, |LHS| оценивается по неравенству КБШ как
    $\sqrt{\sum\limits_{k=1}^{n} \left(f'_{x_k}\big|_{t_k} - f'_{x_k}\big|_{x^0}\right)^2} \cdot \sqrt{\sum\limits_{k=1}^{n} \left(x_k-x_k^0\right)^2} < \varepsilon \sqrt{n} \hyperlink{modv}{|x-x^0|}$
    при $|x-x_0| < \delta$ (пользуемся непрерывностью $f'_{x_k}$ в окрестности точки $x_0$ и тем, что $|t_k-x_0| < |x-x_0|$). Для $m>1$ достаточно представить $f$ в виде $f=(f_1,...,f_m)$ и рассмотреть каждую $f_k$ отдельно.
\end{proof}

\begin{remark}
    Наличия частных производных в точке недостаточно, чтобы сказать, что функция дифференцируема. 
\end{remark}

\begin{defn}
    \hindex{dirder}{Производная по направлению}
    \de{19}{Производной функции $f$ по направлению} единичного вектора $e$ в точке $x$ называется предел:
    \[
        \lim\limits_{
            \begin{smallmatrix}
                t\in \mathbb{R} \\
                t \to 0
            \end{smallmatrix}
        }
        \frac{f(x+te)-f(x)}{t}
    \]
\end{defn}

\begin{remark}
    Частная производная $f$ по $k$-ой координате это
    производная по направлению $(\underbrace{0,...,0}_{k-1}, 1, 0, ..., 0)$.
\end{remark}

\begin{remark}
    Производная $f(x_1,x_2,...,x_n)$ по направлению $e=(e_1, e_2, ..., e_n)$ выражается через частные производные $f$.
    \[
        f'_e(x)=\sum\limits_{k=1}^{n} e_k f'_{x_k}(x)
    \]
\end{remark}

%10
\subsection{Билет 10. Лемма о билипшицевости.} \

\textbf{Утв. 1.} $f$ - гладкая в окрестности точки $x^0$ с непрерывными частными производными, тогда $f$ липшицева, то есть, $|f(x)-f(y)|\leq C||x-y||$. Если мы зафиксируем точку $x$, то $|f(x)-f(y)|\leq(||A||+\varepsilon)||x-y||$ $\forall \varepsilon >0$, $A=A_x$. $||A_x||\leq\sum_{k=1}^n\bigg|\frac{\partial f}{\partial x_k}\bigg|_x\bigg|$. \ 

\textbf{Утв. 2. [\de{20}{Лемма о билипшицевости}]} Если к тому же $\Ker (A)=\{0\}$, то $f$ - билипшицево (в окрестности $x^0$), $C_2||x-y||\leq |f(x)-f(y)|\leq C_1||x-y||$. Докажем и его. $f(y)= f(x)+A_x(y-x)+o(||x-y||)$, тогда $||A_{x^0}z||\geq \varepsilon ||z||$, $\forall z\in \RR^n$, $||A_xz||$ - $A_xz= A_{x^0}z+(A_x-A_{x^0})z$. Первый элемент не меньше $\varepsilon ||z||$, а $||A_x-A_{x^0}||$ стремится к 0 в окрестности этой точки, тогда 
    
    \[
        |f(y)-f(x)|=|A_x(y-x)|+o(||x-y||),
    \]

но каждый из них можно ограничить снизу $\frac{\varepsilon}{2}||x-y||$. \ 


%11 доказательство?
\subsection{Билет 11. Теорема об обратном отображении.}


\begin{theorem}
    (\de{21}{Теорема об обратном отображении}).
    Пусть $f: G \to \mathbb{R}^n (G -$ открытое в $\mathbb{R}^n)$.
    У $f$ есть частные непрерывные производные. $A$ - производная $f$
    в точке $x^0$, причём $A$ невырождена. Тогда в окрестности т. $x^0$
    существует гладкая $g$, т.ч. $g(f(x))=x$ и производная $g$ в т. $x^0$ это $A^{-1}$. 
\end{theorem}

\begin{proof}

    Из прошлого билета у нас есть два утверждения

\begin{stat}
    $f$ гладкая в окрестности $x^0$, значит она там же липшецева, т.е. $\exists C \ \forall x,y:\ |f(x)-f(y)| < C ||x-y||$
\end{stat}

\begin{stat}
    $Ker(A)=\{0\} \Rightarrow f$ билипшицева, т.е. $\exists C_1,\ C_2 > 0 \ \forall x,y: \ C_1||x-y|| < |f(x)-f(y)| < C_2 ||x-y||$
\end{stat}

    Итого, у нас есть отображение $f:f(x)=f(x^0)+A(x-x^0)+o(||x-x^0||)$. Рассмотрим шарик $B_r(x^0)=\{x:||x-x^0||<r\}$, $f(B_r(x^0))$, $f$ - биективна. Проверим, что он содержится в каком-то $B_{r'}(f(x^0))$. \ 

    \textbf{Утв. 3.} В условиях теоремы для любого $r$ существует $r'$, $f(\overline{B_r(x^0)})\supset \overline{B_{r'}(f(x^0))}$. Для любого $y\in B_{r'}(f(x^0))$ $f(x)=y$, хотим найти $x$. $F(x)= ||f(x)-y||^2$, гладкая в окрестности $x^0$. Минимум этой функции где-то достигается (непрерывная на компакте). $F(x^0)= ||f(x^0)-y||^2\leq r'^2$, тогда минимум не может достигаться на границе, так как иначе $||x-x^0||=r$. Тогда с одной стороны $||f(x)-y||^2=||f(x)-f(x^0)+f(x^0)-y||^2$. $f$ билипшицева, поэтому разность первых двух можно оценить чнизу $\varepsilon||x-x^0||$, а разность последних двух можно ограничить сверху $r'^2$, то есть, вся эта вещь как минимум $r'^2$. \ 

    Пусть $w$ - минимум $F(x)$ на $B_r(x^0)$, тогда $\grad F(w)=0$,  $=||f(x)-y||^2 = \sum_{k=1}^n(f_k(x)-y_k)^2$, 
    \[
        \frac{\partial F}{\partial x_l}\bigg|_w = \sum_{k=1}^n\frac{\partial f_k}{\partial x_l}\bigg|_w 2(f_k(x)-y_k), 
    \]
    И теперь, если подставить в обе стороны $w$, то получатся нули. Посмотрим на правую часть как на СЛУ. Дроби фиксированы - числа из матрицы Якоби (изнач. - $A_w$), $f_k(x)$ - какие-то неизвестные. Матрица невырождена, так как нвырожденность не меняется от приведённого шевеления. Значит, решение этой системы единственно, но одно из решений мы уже знаем: $f_k(w)=y_k$. Поэтому, если мы подставим точку минимума функции $F$, то окажется, что эта точка переводится как раз в точку $y$, и тогда $F=0$, а мы в точности нашли прообраз.

    Мы уже установили, что $f$ - билипшицева, что $f$ в какой-то окрестность $f(V_{x_0})$ содержит $V_{g_0}$, а также, что обратное отображение $g$ по крайней мере существует в какой-то окрестности. \ 

    Осталось лишь доказать один небольшой оставшийся момент. Пусть $f$ - гладкая в $x^0$, тогда мы можем расписать $f(x)=f(x^0)+A(x-x^0)+R(x)$, где $|R(x)|=o(|x-x^0|)$ (модули над векторами с данного момента - естественно, нормы). Воспользуемся тем, что любая точка $y$, достаточно близкая к $y^0$, то, по доказанному ране, у неё есть прообраз. Напишем тогда ввиду прообразов: $y=y^0+A(g(y)-g(y^0))+R(g(y))$. Она выполнена для любого $y$ в некоторой окрестности $y^0$. \ 

    Нам хотелось бы выразить $g(y^0)$. Рассмотрим $A(g(y)-g(y^0))=y-y^0-R(g(y))$. Это равенство двух векторов $\RR^n$, матрица $A$ невырожденная, потому у неё есть обратная, применим это знание: $g(y)-g(y^0)=A^{-1}(y-y^0)-A^{-1}(R(g(y)))$. Надо оценить остаток, так как всё остальное уже хорошее. Из того, что мы уже знаем, $\forall \varepsilon>0, V_{x^0}^\varepsilon |R(x)|\leq \varepsilon |x-x^0|$ (кажется, в некоторой окрестности, поэтому размер окрестности должен быть другой переменной). Тогда $|R(g(y|\leq \varepsilon |g(y)-g(y^0)|$. Мы знаем, что $f$ билипшицева, как и обратная, поэтому продолжаем неравенство $\leq \varepsilon C |y-y^0|$. Но у нас изначально есть $|A^{-1}(R(g(y)))|\leq ||A^{-1}||\varepsilon C |y-y^0|$. И теперь, собирая всё назад, получаем, что $g(y)=g(y^0)+A^{-1}(y-y^0)+o(|y-y^0|)$, это нам и нужно было: дифференцируемость в $y^0$ и явный дифференциал. \ 

\end{proof}

%12 примеры?
\subsection{Билет 12. Матрица Якоби. Градиент.}


\begin{defn}
    \de{22}{Градиент} это вектор, состоящий из частных производных $f: \mathbb{R}^n \to \mathbb{R}$
    \[
        \nabla f = \textrm{grad } f =
        \left(\frac{\delta f}{\delta x_1}, \frac{\delta f}{\delta x_2}, ... , \frac{\delta f}{\delta x_n} \right)
    \]
\end{defn}

\begin{remark} Свойства
    \begin{itemize}
        \item Градиент указывает направление вектора, вдоль которого функция имеет наибольшее возрастание
        \item $df = \sum\limits_{k}^{} \frac{\delta f}{\delta x_k} dx_k = \langle \textrm{grad } f, dx \rangle$
    \end{itemize}
\end{remark}

\begin{defn}
    \de{23}{Матрица Якоби} - матрица состоящая из всех частных производных
    $f: \mathbb{R}^n \to \mathbb{R}^m, f=(f_1, f_2, ... , f_m)$
    \[
        J(x) =
        \begin{pmatrix}
            \frac{\delta f_1}{\delta x_1}(x) & \frac{\delta f_1}{\delta x_2}(x) & \dots  & \frac{\delta f_1}{\delta x_n}(x) \\
            \frac{\delta f_2}{\delta x_1}(x) & \frac{\delta f_2}{\delta x_2}(x) & \dots  & \frac{\delta f_2}{\delta x_n}(x) \\
            \vdots                           & \vdots                           & \ddots & \vdots                           \\
            \frac{\delta f_m}{\delta x_1}(x) & \frac{\delta f_m}{\delta x_2}(x) & \dots  & \frac{\delta f_m}{\delta x_n}(x) \\
        \end{pmatrix}
    \]
\end{defn}

\begin{remark} Свойства
    \begin{itemize}
        \item Строчки матрицы Якоби - градиенты соответствующих функций
        \item Если все $f_k$ непрерывно дифференцируемы в окрестности $a$, то матрица Якоби - производная $f$ в $a$,
              т.е. $f(x)=f(a)+J(a)(x-a)+o(|x-a|)$
        \item (Свойство функториальности) Если
              $\varphi: \mathbb{R}^n \to \mathbb{R}^m, \psi: \mathbb{R}^m \to \mathbb{R}^k$
              дифференцируемы, то $J_{\psi \circ \varphi}(x) = J_\psi(\varphi(x))J_\varphi(x)$
    \end{itemize}
\end{remark}

%13
\subsection{Билет 13. Дифференцирование обратного отображения.}

\begin{theorem}
    Пусть $f$ - отображение из $X \subset \mathbb{R}^m$ в $Y \subset \mathbb{R}^n$, дифференцируемое в $x_0 \in X$ (его дифференциал там равен $A$), а $g$ - отображение из $Y$ в $Z \subset \mathbb{R}^k$, дифференцируемое в $y_0=f(x_0) \in Y$ (его дифференциал там равен $B$), то композиция $g \circ f: X \ra Z$ этих отображений будет дифференцируема в $x_0$, и её дифференциал там равен $B \circ A$.
\end{theorem}
\begin{proof}
    $f(x)=f(x_0)+A(x-x_0)+o(||x-x_0||)$ при $x \ra x_0$ (отсюда следует, что $f$ непрерывна в $x_0$). Также известно, что $g(y)=g(y_0)+B(y-y_0)+o(||y-y_0||)$. Подставляя первое равенство во второе, получаем, что $g(f(x))=g(\underset{=y_0}{f(x_0)}+A(x-x_0)+o(||x-x_0||))=g(f(x_0))+B(A(x-x_0)+o(||x-x_0||))+o(A(x-x_0)+o(||x-x_0||))=g(f(x_0))+BA(x-x_0)+o(||x-x_0||)$.
\end{proof}
\begin{remark}
    Мы воспользовались утверждением, что $A(h)=O_A(h)$ для вектора $h$ и линейного оператора $A$. Его несложно доказать, используя неравенство треугольника для нормы. Действительно, пусть $h=h_1\vec{e_1}+...+h_n\vec{e_n}$. Тогда $||A(h)||=||\sum_{i=1}^n h_i A(\vec{e_i})|| \leq \sum_{i=1}^n ||h_iA(\vec{e_i})|| \leq (\sum_{i=1}^n ||A(\vec{e_i})||) ||h||=\const_A ||h||$.
\end{remark}

%14
\subsection{Билет 14. Теорема о равенстве смешанных производных.}
\begin{defn}
    \de{24}{Смешанная производная} порядка $k$ определяется индуктивно:\\
    \textit{База} $k=1$: обыкновенная \hyperlink{partial der}{частная производная} $\frac{\delta f}{\delta x_{i_1}}$\\
    \textit{Переход} $k\to k+1$: возьмём частную производную по $i_{k+1}$-ой координате в точке $A$ от частной производной порядка $k$, т.е.
    от $\frac{\delta^k f}{\delta x_{i_k}\delta x_{i_{k-1}}...\delta x_{i_1}}$ (должна быть определена в некоторой окрестности
    $A$), если соответствующий предел существует, то его и назовём смешанной частной производной порядка $k+1$, обозначим так:
    $\frac{\delta^{k+1} f}{\delta x_{i_{k+1}}\delta x_{i_k}...\delta x_{i_1}}$
\end{defn}

\begin{theorem}
    Пусть $f: G\subset \mathbb{R}^2 \to \mathbb{R}$ ($G$ открытое), $f=f(x,y)$,
    смешанные производные $\frac{\delta^2 f}{\delta x \delta y}$,
    $\frac{\delta^2 f}{\delta y \delta x}$ непрерывны в точке $(x_0,y_0)\in G$ и определены в её окрестности.
    Тогда $\frac{\delta^2 f}{\delta x \delta y} (x_0,y_0) = \frac{\delta^2 f}{\delta y \delta x} (x_0,y_0)$.
\end{theorem}

\begin{proof}
    Воспользуемся \hyperlink{lagrange}{формулой Лагранжа} и вспомогательными функциями. [я вырезал пикчу нахуй, надо поправить ситуацию]
    

    \begin{remark}
        Далее используются обозначения
        $f''_{xy}:=\frac{\delta^2 f}{\delta y \delta x}$,
        $f''_{yx}:=\frac{\delta^2 f}{\delta x \delta y}$
    \end{remark}

    \[\varphi(\varepsilon_1, \varepsilon_2) =
        f(x_0, y_0)+f(x_0+\varepsilon_1, y_0+\varepsilon_2)
        -f(x_0+\varepsilon_1, y_0)-f(x_0, y_0+\varepsilon_2)\]
    %
    \[F_1(x)=f(x,y_0+\varepsilon_2)-f(x,y_0)
        \Rightarrow \varphi = F_1(x_0+\varepsilon_1)-F_1(x_0) = \varepsilon_1 F'_1(\xi_1)\]
    \[F'_1(\xi_1)=f'_x(\xi _1, y_0+\varepsilon_2)-f'_x(\xi_1,y_0)=
        \varepsilon_2 f''_{xy}(\xi_1, \xi_2)
        \Rightarrow \varphi = \varepsilon_1 \varepsilon_2 f''_{xy}(\xi_1, \xi_2)\]
    %
    \[F_2(y)=f(x_0+\varepsilon_1,y)-f(x_0,y)
        \Rightarrow \varphi = F_2(y_0+\varepsilon_2)-F_2(y_0) = \varepsilon_2 F'_2(\eta_2)\]
    \[F'_2(\eta_2)=f'_y(x_0+\varepsilon_1,\eta_2)-f'_y(x_0,\eta_2)=
        \varepsilon_1 f''_{yx}(\eta_1, \eta_2)
        \Rightarrow \varphi = \varepsilon_1 \varepsilon_2 f''_{yx}(\eta_1, \eta_2)\]
    %
    \[\varphi = \varepsilon_1 \varepsilon_2 f''_{xy}(\xi_1, \xi_2) = \varepsilon_1 \varepsilon_2 f''_{yx}(\eta _1, \eta _2) \]
    \[\xi_1,\eta_1 \in [x_0, x_0+\varepsilon_1], \xi_2,\eta_2 \in [y_0, y_0+\varepsilon_2]\]
    %
    При $(\varepsilon_1, \varepsilon_2) \to 0:$
    $\frac{\varphi}{\varepsilon_1 \varepsilon_2} \to f''_{xy}(x_0, y_0) = f''_{yx}(x_0, y_0)$
\end{proof}

\begin{cons}
    Если частные производные $\frac{\delta^2 f}{\delta x_i \delta x_j}$ и $\frac{\delta^2 f}{\delta x_j \delta x_i}$ непрерывны в точке и определены в её окрестности, то и равны в ней.
\end{cons}

%15
\subsection{Билет 15. Формула Тейлора с остатком в форме Пеано.}


$f: \mathbb{R}^n \to \mathbb{R}$. Сведём всё к одномерному случаю, т.к. одномерную формулу Тейлора мы знаем,
$x\in \mathbb{R}^n$ - центр разложения в ряд Тейлора, $y\in \mathbb{R}^n, h=y-x$.\\
$[x,y]$ - отрезок, его можно параметризовать так: $x+t(y-x)=x+th, t\in [0,1]$.\\
$\varphi (t) = f(x+th)$, эту функцию мы можем дифференцировать, т.к. это композиция дифференцируемых функций.\\
$\varphi' (t) = \langle \textrm{grad } f, h \rangle = \sum\limits_{k=1}^{n} \frac{\delta f}{\delta x_k} (x+th) h_k $\\
...\\
$\varphi^{(s)} (t) = \sum\limits_{1 \leq k_1,...,k_s \leq n}^{} \frac{\delta^s f}{\delta x_{k_1} ... \delta x_{k_s}} (x+th) h_{k_1}...h_{k_s} = \left(\left(\sum\limits_{k=1}^{n} \frac{\delta}{\delta x_k} h_k\right)^s f\right) (x+th)$\\
(последнее равенство следует воспринимать как удобное обозначение)
\[\varphi(\tau)=\sum\limits_{s=0}^{m} \frac{\varphi^s(0)}{s!}\tau^s + \underbrace{\frac{\varphi^{(m+1)}(\xi)}{(m+1)!}\tau^{(m+1)}}_{R_m(\tau,\varphi)}, \ \  \xi \in [0, \tau] \]
\[R_m(\tau,\varphi) = \int_{0}^{\tau} \frac{\varphi^{(m+1)}(l)}{m!}(\tau^{m}-l)^m dl = \int_{0}^{1} \frac{\varphi^{(m+1)}(l\tau)}{m!} \tau^{m+1} (1-l)^m dl \]
Подставим $\tau=1, t=0$, получим
\begin{stat}
    \de{25}{Формула Тейлора} с остатком в форме Пеано, $h=y-x$
    \[
        f(y) = \sum\limits_{s=0}^{m} \left(\left(\sum\limits_{k=1}^{n} \frac{\delta}{\delta x_k} h_k\right)^s \frac{f}{s!} \right) (x) + o(|h|^m)
    \]
\end{stat}

%16
\subsection{Билет 16. Формула Тейлора с остатком в интегральной форме.}


\begin{stat}
    Формула Тейлора с остатком в интегральной форме (см. прошлый билет), $h=y-x$
    \[
        f(y) = \sum\limits_{s=0}^{m} \left(\left(\sum\limits_{k=1}^{n} \frac{\delta}{\delta x_k} h_k\right)^s \frac{f}{s!} \right) (x) + \int_{0}^{1} \frac{(1-l)^m}{m!} \left(\left(\sum\limits_{k=1}^{n} \frac{\delta}{\delta x_k} h_k\right)^{m+1} f \right) (x+lh) dl
    \]
\end{stat}

%17 какой-то маленький билет
\subsection{Билет 17. Необходимое условие экстремума функции многих переменных.}


\begin{theorem}(\de{26}{Необходимое условие экстремума})
    $f: G\subset \mathbb{R}^n \to \mathbb{R}$ ($G$ открытое) гладкая, $x^0$ - точка локального минимума или максимума.
    Тогда $\textrm{grad } f\big|_{x^0} = 0$
\end{theorem}

\begin{proof}
    $\textrm{grad } f = \left(\frac{\delta f}{\delta x_1}, \frac{\delta f}{\delta x_2}, ... , \frac{\delta f}{\delta x_n} \right)$\\
    Пусть $\frac{\delta f}{\delta x_k}\big|_{x^0} \not= 0$, тогда
    $f(x^0_1, ..., x^0_{k-1}, x_k, x^0_{k+1}, ... , x^0_n) - f(x^0_1, ... ,x^0_n) = $\\
    $\underbrace{\frac{\delta f}{\delta x_k}\big|_{x^0}}_{\not=0}(x_k-x^0_k) +o(|x_k-x^0_k|)$
    $\Rightarrow \rightarrow\leftarrow$ (свелось к одномерному случаю)
    \begin{remind}
        Одномерный случай, $g: \mathbb{R} \to \mathbb{R}$ \\
        $g(x)=g(x^0)+g'(x^0)(x-x^0)+o(|x-x^0|)$
        \begin{itemize}
            \item $g'(x^0)>0$, тогда при достаточно малом 
            $|x-x^0|: \\ 0 < g'(x^0)-\varepsilon< \frac{g(x)-g(x^0)}{x-x^0} < g'(x^0)+\varepsilon$,
            в таких окрестностях $g(x)>g(x^0)$ при $x>x^0$, $g(x)<g(x^0)$ при $x<x^0$, значит $x^0$ - не экстремум.
            \item $g'(x^0)<0$, тогда при достаточно малом 
            $|x-x^0|: \\ g'(x^0)-\varepsilon< \frac{g(x)-g(x^0)}{x-x^0} < g'(x^0)+\varepsilon < 0$,
            в таких окрестностях $g(x)<g(x^0)$ при $x>x^0$, $g(x)>g(x^0)$ при $x<x^0$, значит $x^0$ - не экстремум.
        \end{itemize}
    \end{remind}
\end{proof}

%18 доказательство?
\subsection{Билет 18. Знак квадратичной формы. Достаточные условия экстремума функции многих переменных.}

\begin{defn}
    \de{27}{Квадратичная форма} $Q(x), x=(x_1, ..., x_n)$ - 
    это выражение вида $\sum\limits_{1 \leq k,l \leq n} a_{k,l} x_k x_l$, где $a_{k,l}$ - скаляр.  
\end{defn}

\begin{defn}
    (\de{28}{Знак квадратичной формы})
    Квадратичная форма $Q(x), x\in \mathbb{R}^n, a_{k,l} \in \mathbb{R}$ положительно (отрицательно) определена, если для всех ненулевых $x: Q(x)>0 (Q(x)<0)$
    и знакопеременна, если может принимать как положительные, так и отрицательные значения.
\end{defn}

\begin{remark} (Квадр. форма на сфере)
    Легко видеть, что $Q(cx)=c^2 Q(x)$, где $c$ - скаляр, поэтому $Q(x)$ имеет тот же знак, что и $Q(cx)$,
    т.е. вдоль фиксированного направления кв. форма имеет фиксированный знак, а значит достаточно 
    рассматривать $x$ на сфере, чтобы определить знак кв.формы. Сфера - компакт, $Q(x)$ - непрерывная функция, 
    а потому $Q(x)$ достигает минимума и максимума на сфере. Тогда можно составить равносильное определение
    знака, квадратичная форма называется:
    \begin{itemize}
        \item положительно-определенной, если $ \displaystyle \min_{||x||=1} Q(x) > 0$
        \item отрицательно-определенной, если $ \displaystyle \max_{||x||=1} Q(x) < 0$
        \item знакопеременной, если $ \displaystyle \min_{||x||=1} Q(x) < 0 < \max_{||x||=1} Q(x)$
    \end{itemize}
\end{remark}

\begin{theorem}(\de{29}{Достаточное условие экстремума})
    Пусть у $f: \mathbb{R}^n \to \mathbb{R}$ в точке $x^0$ нулевой градиент и определены все смешанные производные второго порядка.
    Тогда если квадратичная форма
    $Q(h)=\sum\limits_{1 \leq k,l \leq n} \frac{\delta^2 f}{\delta x_k \delta x_l}(x^0) h_k h_l$
    \begin{itemize}
        \item положительно определена, значит $x^0$ точка локального минимума.
        \item отрицательно определена, значит $x^0$ точка локального максимума.
    \end{itemize}
\end{theorem}

\begin{proof}
    Докажем для полож. квадр. формы, \hlink{taylor}{формула Тейлора}
    \[
        f(x) = f(x^0) + \frac{1}{2} \sum\limits_{1 \leq k,l \leq n} \frac{\delta^2 f}{\delta x_k \delta x_l} (x^0) (x_k-x_k^0) (x_l-x_l^0) + o(||x-x_0||^2)
    \]
    Обозначим $a_{k,l} = \frac{1}{2} \frac{\delta^2 f}{\delta x_k \delta x_l} (x^0)$\\
    При $x\not=x^0: \sum\limits_{1 < k,l\leq n} a_{k,l} (x_k-x_k^0) (x_l-x_l^0) = \langle A(x-x^0), (x-x^0) \rangle = Q(x-x^0) > 0$
    $\Rightarrow \exists \varepsilon>0: \ Q(x-x^0) \geq \varepsilon ||x-x^0||^2$

    \textit{Пояснение: } $Q(x)$ - положительная квадратичная форма, а потому 
    $\displaystyle \varepsilon = \min_{||x||=1} Q(x) > 0$,
    тогда $Q(x)=Q(||x||e)=Q(e)||x||^2 \geq \varepsilon ||x||^2$, т.к. $||e||=1$.

    Поделим равенство
    \[
        f(x) - f(x^0) = Q(x-x^0) + o(||x-x_0||^2)
    \]
    на $||x-x^0||^2$, тогда, если $||x-x^0||$ достаточно мало, то правая часть строго положительна.
\end{proof}

\begin{remark}
    Квадратичную форму можно привести к симметричному виду $\sum a_{k,l} h_k h_l$,  $a_{k,l}=a_{l,k}$,
    значит её можно привести и к диагольному виду $Q(h) = \sum_{k=1}^n \lambda_k h_k^2$
    \begin{itemize}
        \item $Q(h)$ положительна $\Leftrightarrow$ все $\lambda_k > 0$ ($x^0$ т. мин.)
        \item $Q(h)$ отрицательна $\Leftrightarrow$ все $\lambda_k < 0$ ($x^0$ т. макс.)
        \item $Q(h)$ знакопеременна $\Leftrightarrow$ $\exists k,l: \lambda_k < 0 < \lambda_l$ ($x^0$ не т. экстр.)
        \item иначе требуется дополнительное исследование
    \end{itemize}
\end{remark}

\begin{remark}
    Почему, если кв. форма знакопеременна, то $x^0$ не экстремум? Зафиксируем $y_1, y_2$ т.ч. $Q(y_1) < 0 < Q(y_2)$ и $||y_1||=||y_2||=1$.
    Тогда в равенство
    \[
        f(x) - f(x^0) = Q(x-x^0) + o(||x-x_0||^2)
    \]
    можно подставлять точки вида $x=cy_1+x^0$ ($\Rightarrow Q(x-x^0)=c^2 Q(y_1)$), в этом случае при достаточно малом $|c|=||x-x^0||$
    правая часть строго отрицательна; если же подставлять $x=cy_2+x^0$ - строго положительна.
\end{remark}

\begin{stat}
    \de{30}{Критерий Сильвестра} для симметричной квадратичной формы:
    \begin{itemize}
        \item для положительной определённости квадратичной формы необходимо и достаточно, чтобы \href{https://ru.wikipedia.org/wiki/%D0%9C%D0%B8%D0%BD%D0%BE%D1%80_(%D0%BB%D0%B8%D0%BD%D0%B5%D0%B9%D0%BD%D0%B0%D1%8F_%D0%B0%D0%BB%D0%B3%D0%B5%D0%B1%D1%80%D0%B0)}
        {угловые миноры} её матрицы были положительны.
        \item Для отрицательной определённости квадратичной формы необходимо и достаточно, чтобы \href{https://ru.wikipedia.org/wiki/%D0%9C%D0%B8%D0%BD%D0%BE%D1%80_(%D0%BB%D0%B8%D0%BD%D0%B5%D0%B9%D0%BD%D0%B0%D1%8F_%D0%B0%D0%BB%D0%B3%D0%B5%D0%B1%D1%80%D0%B0)}
        {угловые миноры} чётного порядка её матрицы были положительны, а нечётного порядка — отрицательны.
    \end{itemize}
\end{stat}

%19
\subsection{Билет 19. Касательные вектора. Касательная плоскость.}


\begin{stat}
    Есть уравнение $z=f(x,y)$ задающее плоскость и у $f$ есть частные производные в 
    $(x_0, y_0)$. 
    Тогда $z=f(x_0,y_0)$
    $+\df{f}{x}\big|_{(x_0,y_0)}(x-x_0)$
    $+\df{f}{y}\big|_{(x_0,y_0)}(y-y_0)$ - уравнение касательной плоскости.
    $(\frac{\delta f}{\delta x}\big|_{(x_0,y_0)}, \frac{\delta f}{\delta y}\big|_{(x_0,y_0)}, -1)$ - вектор нормали к кас.плоскости.
\end{stat}
%20
\subsection{Билет 20. Теорема о неявной функции для двух переменных.}%20


Это про случай $\mathbb{R}^2 \to \mathbb{R}$


\begin{theorem} (\de{31}{О неявной функции}).
    $F: G\to \mathbb{R} (G\subset \mathbb{R}^2 - $ открытое $)$
    \begin{itemize}
        \item $F(x_0, y_0) = 0$
        \item $F \in C^1(G)$
        \item $F'y(x_0, y_0) \not= 0$
    \end{itemize} 
    Тогда $\exists I_x, I_y: \ \ x_0\in I_x, y_0 \in I_y, I_x\times I_y \subset \mathbb{R}^2$\\
    и $f: I_x\times I_y \in \mathbb{R}$ такая, что 
    $$F(x,y)=0 \Leftrightarrow y=f(x)$$
    $$f\in C^1, f'(x)=-\frac{F'_x(x,f(x))}{F'_y(x,f(x))}$$
\end{theorem}

\begin{remark}
    (неформальное рассуждение, почему такая формула) Предположим, что $f$ существует и дифференцируема,
    $F(x,f(x))=0$ на всей области определения $f$, продифференцируем левую и правую часть по правилу композиции
    $F'_x(x,f(x))+F'_y(x,f(x)) f'(x)=0$.
\end{remark}

\begin{proof}
    Существование: пусть функция $g_x(y) = F(x,y)$, будем считать, что в малой окрестности $V_{(x_0,y_0)}: \ $ 
    $F'_y > 0$, тогда $g(y)$ - возрастающая. Тогда $\forall x \in V \exists ! y: \ F(x,y)=0$. \\
    Непрерывность: рассмотрите прямоугольник с разрезами (т.е. $g_x(y)$). Так как решения уравнения $F(x,y)=0$ образуют
    замкнутое множество, то из того, что $x \to a$ и $f(x) \not\to f(a)$, следует, что есть второй корень на линии $x=a$, а у 
    нас корни единственные. \\
    Гладкость: $F(x+h,f(x+h))-F(x,f(x))=F(x+h,f(x)+(f(x+h)-f(x)))-F(x,f(x))=(F)$, при $h\to 0$ можно применять формулу Тейлора 
    для $F$, получим $(F)=h(F'_x(x,f(x))+\frac{f(x+h)-f(x)}{h}F'_y(x,f(x)))+o(|h|)$
\end{proof}


\subsection{Билет 21. Теорема о неявной функции для произвольного числа переменных.}%21


Это про случай $\mathbb{R}^m \to \mathbb{R}$


\subsection{Билет 22. Теорема о неявной функции для систем уравнений. Примеры.}%22

Это про случай $\mathbb{R}^{m+n} \to \mathbb{R}^n$
\begin{theorem}
    $x\in \mathbb{R}^m , y \in \mathbb{R}^n$\\
    $F: G\subset\mathbb{R}^{m+n}\to \mathbb{R}^n$
    \begin{itemize}
        \item $F\in C^1(G)$
        \item $F(x^0,y^0)=0$
        \item $F'_y(x^0,y^0)$ обратима (матрица $n\times n$ из частных производных)
    \end{itemize}
    тогда в некоторой окрестности $V_{(x^0,y^0)} \subset G$: 
    $$F(x,y) = 0 \Leftrightarrow y=f(x), \ f: V_{(x^0,y^0)} \to \mathbb{R}^n$$
    $$f'(x) = -(F'_y(x,f(x)))^{-1} F'_x(x,f(x))$$
\end{theorem}


\begin{proof}
    Докажем индукцией по $n$.\\
    \textit{База: } $n=1 \ \forall m$ доказано ранее.\\
    \textit{Переход: } $F=(F_1, F_2, ... , F_n), F_k: \mathbb{R}^{m+n} \to \mathbb{R}$.\\
    Имеем $n$ уравнений вида $F_k = 0$.\\ 
    \textit{Хотим: $y_k=f_k(x_1, x_2, ..., x_m)$, т.е. выразить каждый игрик через иксы.}\\
    Матрица невырождена, будем считать, что $\frac{\delta F_n}{\delta y_n} \not= 0$.\\
    Тогда $y_n = f^*(x_1, x_2, ... , x_m, y_1, ... , y_{n-1})$
    по предположению индукции. \\
    $\varphi_k(x,y_1, ..., y_{n-1}) = F_k(x,y_1, ..., y_{n-1}, f^*), 1 \leq k \leq n-1$ ($\varphi_n = 0$ из-за того, как выбрали $f^*$).
    $\frac{\delta \varphi_k}{\delta y_l} = \frac{\delta F_k}{\delta y_l} + \frac{\delta F_k}{\delta y_n} \frac{\delta f^*}{\delta y_l}$\\
    Вспомним как выглядит невырожденная матрица $F'_y(x^0, y^0)$:
    $$
    \begin{pmatrix}
        \df{F_1}{y_1} & \df{F_1}{y_2} &\ldots & \df{F_1}{y_{n-1}} & \df{F_1}{y_n}\\
        \df{F_2}{y_1} & \df{F_2}{y_2} &\ldots & \df{F_2}{y_{n-1}} & \df{F_2}{y_n}\\
        \vdots                        & \vdots                        &\ddots & \vdots                        &  \vdots                      \\
        \df{F_{n-1}}{y_1} & \df{F_{n-1}}{y_2} &\ldots & \df{F_{n-1}}{y_{n-1}} & \df{F_{n-1}}{y_n}\\
        \df{F_n}{y_1} & \df{F_n}{y_2} &\ldots & \df{F_n}{y_{n-1}} & \df{F_n}{y_n}\\
    \end{pmatrix}    
    $$
    Добавим к первым $n-1$ столбцам последний, домноженный на скаляр, от этого матрица не перестанет быть невырожденной.
    $$
    \begin{pmatrix}
        \df{F_1}{y_1} + \df{F_1}{y_n} \df{f^*}{y_1} & \df{F_1}{y_2} + \df{F_1}{y_n} \df{f^*}{y_2} &\ldots & \df{F_1}{y_{n-1}} + \df{F_1}{y_n} \df{f^*}{y_{n-1}} & \df{F_1}{y_n}\\
        \df{F_2}{y_1} + \df{F_2}{y_n} \df{f^*}{y_1} & \df{F_2}{y_2} + \df{F_2}{y_n} \df{f^*}{y_2} &\ldots & \df{F_2}{y_{n-1}} + \df{F_2}{y_n} \df{f^*}{y_{n-1}} & \df{F_2}{y_n}\\
        \vdots                                      & \vdots                                      &\ddots & \vdots                                              & \vdots      \\
        \df{F_{n-1}}{y_1} + \df{F_2}{y_n} \df{f^*}{y_{n-1}}& \df{F_{n-1}}{y_2} + \df{F_{n-1}}{y_n} \df{f^*}{y_2} &\ldots & \df{F_{n-1}}{y_{n-1}}  + \df{F_{n-1}}{y_n} \df{f^*}{y_{n-1}} & \df{F_{n-1}}{y_n}\\
        \df{F_n}{y_1} + \df{F_n}{y_n} \df{f^*}{y_1} & \df{F_n}{y_2} + \df{F_n}{y_n} \df{f^*}{y_2} &\ldots & \df{F_n}{y_{n-1}} + \df{F_n}{y_n} \df{f^*}{y_{n-1}} & \df{F_n}{y_n}
    \end{pmatrix}  = 
    $$
    $$
    =
    \begin{pmatrix}
        \df{\varphi_1}{y_1} & \df{\varphi_1}{y_2} &\ldots & \df{\varphi_1}{y_{n-1}} & \df{F_1}{y_n}\\
        \df{\varphi_2}{y_1} & \df{\varphi_2}{y_2} &\ldots & \df{\varphi_2}{y_{n-1}} & \df{F_2}{y_n}\\
        \vdots           & \vdots           &\ddots & \vdots               &  \vdots \\
        \df{\varphi_{n-1}}{y_1} & \df{\varphi_{n-1}}{y_2} &\ldots & \df{\varphi_{n-1}}{y_{n-1}} & \df{F_{n-1}}{y_n}\\
        \df{\varphi_n}{y_1} & \df{\varphi_n}{y_2} &\ldots & \df{\varphi_n}{y_{n-1}} & \df{F_n}{y_n}\\
    \end{pmatrix} 
    =
    \begin{pmatrix}
          &   &        &   &   & \df{F_1}{y_n}    \\
          &   &        &   &   & \df{F_2}{y_n}    \\
          &   &\varphi'_y &   &   & \vdots           \\
          &   &        &   &   & \df{F_{n-2}}{y_n}\\
          &   &        &   &   & \df{F_{n-1}}{y_n}\\
        0 & 0 & \ldots & 0 & 0 & \df{F_n}{y_n}
    \end{pmatrix} 
    $$
    откуда $det(\varphi'_y) \df{F_n}{y_n} \not= 0 \Rightarrow det(\varphi'_y) \not= 0$, а значит для системы $\{\varphi_k\}_{k\leq n-1}=0$ применимо предположение индукции, т.е. $y_k = f_k(x)$ при $k \leq n-1$.
    Также получаем $y_n=f^*(x, f_1(x), ... , f_{n-1}(x))=f_n(x)$, что мы и хотели получить. Итоговая функция: $f(x)=(f_1(x), f_2(x), ... , f_n(x))$. Гладкость и искомая производная следуют из того, что мы умеем брать
    производную по композиции.  
\end{proof}



\subsection{Билет 23. Полярные и сферические координаты. Параметризации поверхностей.}%23

\begin{defn}
    \de{32}{Полярные координаты} - отображение $f: \RR_+^2 \ra \RR^2$ полуплоскости $\RR_+^2 = \{(\rho, \phi) \in \RR^2| \rho \geq 0\}$ на плоскость $\RR^2$, задаваемое формулами 
    
    \bea
        x = \rho \cos \phi \\ 
        y = \rho \sin \phi.
    \eea
\end{defn}

\begin{defn}
    \de{33}{Сферические координаты} - как полярные, только в $\RR^3$. Задаются же соответственные отношения: 

    \bea
        z = \rho \cos \psi \\ 
        y = \rho \sin \psi \sin \phi \\ 
        x = \rho \sin \psi \cos \phi.
    \eea
\end{defn}

\subsection{Билет 24. Задача условного экстремума. Необходимое условие условного экстремума.}%24


\begin{stat}[\de{34}{Задача условного экстремума}]
    Дана гладкая функция $f: G\subset\mathbb{R}^n \to \mathbb{R}$ и
    и соотношения между между переменными - уравнения(условия) с гладкими функциями 
    вида ($m \leq n$):
    \[
        \begin{cases}
            f_1(x_1, x_2, ... , x_n) = 0 \\
            f_2(x_1, x_2, ... , x_n) = 0 \\
            f_3(x_1, x_2, ... , x_n) = 0 \\
            \dots                        \\
            f_m(x_1, x_2, ... , x_n) = 0 
        \end{cases}  
    \]
    требуется найти условные экстремумы $f$.
\end{stat}

\begin{remark}
    В некоторых случаях можно $n-m$ переменных выбирать произвольно, остальные однозначно восстановятся.
\end{remark}

\begin{remark}
    $m$ соотношений задают какую-то поверхность, обозначим её $S$.
    %Будем считать, что $\dim S = n-m$.
    \textit{Все точки из S можно подставлять в $f$, можно вообще 
    считать, что $f_S: S \to \mathbb{R}$ и мы ищем экстремумы $f_S$}

    Зафиксируем точку $x^0 = (x^0_1, x^0_2, ... , x^0_n) \in S$. \\
    \textit{Хотим выяснить является ли она экстремальной.}

    Уравнение $f(x)=f(x^0)$ задаёт поверхность уровня, 
    обозначим её $L$. \\
    \textit{А вообще - это неважно.}

    Попробуем свести всё к одномерному случаю, для
    этого рассмотрим всевозможные гладкие кривые на $S$, проходящие через $x^0$, 
    т.е. гладкие функции вида 
    $x(t), t\in (-a, a), a \in \mathbb{R}_+, x(t) \in S$,
    $x(0)=x^0$. 
    
    Тогда вдоль такой кривой можно рассмотреть $h(t): t \mapsto f(x(t))$\\
    ($h: \mathbb{R} \to \mathbb{R}$), если т. $x^0$ условный экстремум $f$, 
    то $t=0$ экстр. для $h$ \\
    \textit{Пояснение:} иначе вдоль этой кривой из любой 
    окрестности т. $x^0$ можно пойти по $S$ так, чтобы $f$ возрастала; и так, чтобы
    убывала; пойти = сдвинуть значение $t$ ($t$ - это просто число); оставаясь
    на $S$ мы соблюдаем все $m$ условий, так что всё корректно.\\
    %
    Тогда необходимо, чтобы $h'(0)=0$. По правилу композиции: 
    $h'(0)=(f(x(t)))'\big|_{0}=\sum\limits_{k=1}^{n} \df{f}{x_k}(x(0)) \df{x_k}{t}(0)=$
    $\sum\limits_{k=1}^{n} \df{f}{x_k}(x^0) \df{x_k}{t}(0)=\langle \textrm{grad } f(x(0)), x'(0) \rangle = 0$.
    
    \begin{remind}[geometry reference]
        Два вектора называются перпендикулярными, если их 
        скалярное произведение равно нулю. 
    \end{remind}

    \textit{Что это значит? Если говорить о каком-то геометрическом смысле, 
    то мы получили следующее утверждение: скалярное произведение нуль, значит
    градиент $f$ в т. $x^0$ перпендикулярен касательному вектору $S$ в т. $x^0$ 
    (почему $x'(0)$ это касательный вектор? Да потому что он сам по себе градиент,
    т.е. состоит из производных. В данном случае $x(t)$ это кривая, так что
    и касательный вектор, подразумевается, соответствующий этой кривой)}
    
    Рассматривая всевозможные кривые $x(t)$ можно получать 
    касательный вектор $x'(0)$ по любому направлению в касательной плоскости.
    \textit{Пояснение-цитата от Белова:} ну и понятно, что если мы посмотрим 
    все гладкие кривые на поверхности $S$, проходящие через точку $x^0$, 
    то в качестве касательных векторов мы получим любой вектор из касательной 
    плоскости.

    Но тогда мы получаем вывод, что градиент $f$ в т. $x^0$ 
    перпендикулярен любому вектору из касательной плоскости, т.е. перпендикулярен
    всей касательной плоскости к $S$ в т. $x^0$ \textit{(это тавтология по сути)}\\
    \textit{Равносильное утверждение: у $S$ и $L$ совпадают кас. плоскости в т. $x^0$;
    равносильность следует из того, что градиент всегда перпендикулярен касательной плоскости}

    Градиенты $f_k$-ых тоже перпендикулярны касательной плоскости к $S$. 

    $\Rightarrow f \in Lin(f_k)$, т.е. $f=\sum\limits_{k=1}^{m} \lambda_k f_k$
\end{remark}



\subsection{Билет 25. Функции Лагранжа. Достаточное условие условного экстремума. Примеры.}%25

\hindex{flagrange}{Функция Лагранжа}
\begin{defn}
    Пусть есть $m$ условий вида $F_k(x_1, ... , x_n) = 0$ и мы хотим найти условный 
    экстремум $F(x_1, ... , x_n)$. Рассмотрим \de{35}{функцию Лагранжа} 
    $L(x_1, ... x_n, \lambda_1, ..., \lambda_m)=F-\sum\limits_{k=1}^{m} \lambda_k F_k$,
    тогда все частные производные $L$ должны быть нулевыми. 
\end{defn}



\subsection{Билет 26. Теорема о перестановке пределов. Общий вид теоремы Стокса-Зайделя.}%26

\begin{theorem}
    Пусть $X, Y$ - хаусдорфовы топологические, а $Z$ - полное метрическое пространства. Также есть множества $A \subset X$, $B \subset Y$, имеющие предельные точки $a \notin A$ и $b \notin B$ соответственно. $F: X \times Y \ra Z$ - функция. Предположим, выполнены следующие условия:
    \begin{enumerate}
        \item Для любого $y$ существует равномерный по $y$ предел $\phi(y)=\lim_{x \ra a} F(x, y)$
        \item Для любого $x$ существует предел $\psi(x)=\lim_{y \ra b} F(x, y)$
    \end{enumerate}
    Тогда существуют и равны пределы $\lim_{y \ra b}\phi(y)$, $\lim_{x \ra a} \psi(x)$ и $\lim_{x \ra a, y \ra b} F(x, y)$
\end{theorem}
\begin{proof}
    \begin{lemma}
    Пусть $X$ - хаусдорфово топологическое , а $Z$ - полное метрическое (с метрикой $\rho$) пространства, есть множество $C \subset X$, имеющее предельную точку $c \notin C$. Пусть для. любого $\epsilon > 0$ существует окрестность $V_{\epsilon}$ точки $c$ такая, что для любых $w_1, w_2 \in V_{\epsilon}$ $\rho(f(w_1), f(w_2)) < \epsilon$. Тогда $f$ имеет предел в $c$.
    \end{lemma}
    \begin{proof}
    Положим $\epsilon_n=\frac{1}{n}$ и рассмотрим соответствующую ему окрестность $V_{\frac{1}{n}}$. Можем считать, что $V_{\frac{1}{n+1}} \subset V_{\frac{1}{n}}$. В каждой окрестности выберем точку $w_n \in V_{\frac{!}{n}}$. Легко видеть, что $\{f(w_n)\}$ фундаментальная последовательность, имеющая предел $z_0$ (так как $Z$ полно). Тогда $\rho(f(x), z_0) \leq \rho(f(x), f(w_n))+\rho(f(w_n), z_0) < \frac{2}{n}$ - мы воспользовались тем, что $\rho(f(x), f(w_n))<\frac{1}{n}$, если $x$ достаточно близко к $w_n$ (например, лежит в $V_{\frac{1}{n}}$). 
    \end{proof}
    Перейдём к доказательству теоремы. Условие 1) теоремы равносильно следующему: $\forall \epsilon > 0 \exists U_a \ni a : \forall x \in U_a \forall y \in B \rho(F(x, y), \phi(y)) < \epsilon$.
    Зафиксируем сначала произвольное $\epsilon > 0$ (ему соответствует $U_a$), а затем $x_0 \in U_a \cap A$ (это множество непусто, так как $a$ - предельная). Тогда $\rho(F(x_0, y), \phi(y)) < \epsilon$ при всех $y \in B$(*). Кроме того, существует окрестность $V_b \ni b$ такая, что $\rho(F(x_0, y), \psi(x_0))< \epsilon$ для любого $y \in V_b \cap B$ (здесь $V_b$) зависит от $x_0$, так как равномерности нет) (**).
    \\
    Выберем теперь произвольные $y_1, y_2 \in V_b \cap B$. Тогда $\rho(\phi(y_1), \phi(y_2)) \leq \rho(\phi(y_1), F(x_0, y_1))+\rho(F(x_0, y_1), \psi(x_0))+\rho(\psi(x_0), F(x_0, y_2))+\rho(F(x_0, y_2), \phi(y_2)) < 4\epsilon$ (мы воспользовались неравенствами (*)и (**)). По лемме, существует $\lim_{y \ra b} \phi(y)=P$
    \\
    $\rho(\psi(x_0), P) \leq \rho(\psi(x_0), F(x_0, y))+\rho(F(x_0, y), \phi(y))+\rho(\phi(y), P) < 3\epsilon$, если $x_0 \in U_a \cap A$, $y \in V_b \cap B$. Это и значит, что существует $\lim_{x \ra a} \psi(x)=P=\lim_{y \ra b} \phi(y)$.
    \\
    Осталось разобраться с пределом по совокупности переменных. $\rho(F(x, y), P) \leq \rho(F(x, y), \phi(y))+\rho(\phi(y), P) <2\epsilon$. Когда именно? Сначала выбираем такую окрестность $V \ni b$, что $\rho(\phi(y), P) <\epsilon$, тогда окрестность $U \ni a$ такая, что $\rho(F(x, y), \phi(y)) \epsilon$ для всех $y \in V$ найдётся по равномерности предела $\phi(y)=\lim_{x \ra a} F(x, y)$.
\end{proof}
\begin{theorem}
    (\de{36}{Теорема Стокса-Зейделя}). Пусть $X, Y$ - хаусдорфовы топологичесике, а $Z$ - полное метрическое пространства. $A \subset X$, $B \subset Y$ - множества, $b \in B$ - предельная точка. $F: X \times Y \ra Z$ - функция, и $x_0 \in X$. Предположим, выполнены следующие условия:
    \begin{enumerate}
        \item Для любого $y \in B$ функция $F(x, y)$ непрерывна в точке $x_0$.
        \item существует равномерный по $x \in A$ предел $\psi(x)=\lim_{y \ra b} F(x, y)$
    \end{enumerate}
    Тогда $\psi(x)$ непрерывна в $x_0$.
\end{theorem}
\begin{proof}
    Просто применить предыдущую теорему
\end{proof}

\subsection{Билет 27. Голоморфные функции. Примеры. Общий вид дифференциала голоморфной функции.}%27

Функция $f:D \ra C$ - \de{37}{голоморфна} в области(открытое связное множество) $D$, если она имеет комплексную производную в каждой точке $z \in D$. Говорт, что функция голоморфна в точке $z_0$, если она голоморфна в некоторой окрестности этой точки.

Примеры: 
\begin{enumerate}
    \item $f = $ const $; f^{'}(z) = 0$ 
    \item $f(z) = az, a \neq 0;$ Если $a = re^{i\phi}$, то $f$ поворачивает плоскость вокруг 0 на угол $\phi$ и растягивает плоскость в $r$ раз.
    \item $f(z) = z^n, f$ увеличивает угол между лучами, выходящими из 0 в $n$ раз.
\end{enumerate}

\subsection{Билет 28. Степенные ряды. Радиус сходимости степенного ряда.}%28


\subsection{Билет 29. Голоморфность суммы степенного ряда.}%29

\begin{stat}
    Пусть ряд $f(z)=\sum_{k=0}^{\infty} с_k(z-z_0)^k$ сходится в точке $z_1$. Тогда этот ряд сходится абсолютно и равномерно в круге любого радиуса, строго меньшего, чем $|z_1-z_0|$.
\end{stat}
\begin{proof}
   Пусть $r<|z_1-z_0|$ - радиус того самого круга. Так как ряд $\sum_{k=0}^{\infty} с_k(z_1-z_0)^k$ сходится, $\lim_{n \ra \infty} |a_n|\cdot |z_1-z_0|^n=0$. В частности, $|a_n|\cdot |z_1-z_0|^n < C$ при всех $n$. Значит, "хвост" этого ряда - $\sum_{k=N}^{\infty} |a_k| \cdot |z-z_0|^k < C \sum_{k=N}^{\infty} \big ( \underset{\frac{|z-z_0|}{|z_1-z_0|}}{=q}\big )^k \leq C \sum_{k=N}^{\infty} r^k$ - оценивается числом, не зависящим от $z$.
\end{proof}
\begin{theorem}
    Пусть ряд $f(z)=\sum_{k=0}^{\infty} c_k(z-z_0)^k$ (*) имеет радиус сходимости $R$. Тогда функция $f$ дифференцируема в любой точке $t$, в которой $|t-z_0|<R$, причём её производная равна $f'(t)=\sum_{k=0}^{\infty} kc_k(t-z_0)^{k-1}$ (**).
\end{theorem}
\begin{proof}
    Хотим доказать дифференцируемость в какой-то точке $t$. Для начала заметим, что $\lim_{n \ra \infty} \sqrt[n]{n}=1$, и радиусы сходимости $R$ и $R'$ соответственно рядов (*) и (**) равны в силу теоремы Коши-Адамара: $\frac{1}{R'}=\limsup_{n \ra \infty} \sqrt[n]{nc_n}=\limsup_{n \ra \infty} \sqrt[n]{c_n}=\frac{1}{R}$. В частности, ряд (**) равномерно сходится в любом круге радиуса меньше $R$, в том числе и в некоторой окрестности точки $t$. Определим $f_n(z)=\sum_{k=0}^n c_k(z-z_0)^k$, $f'_n(z)=\sum_{k=0}^{n} kc_k(z-z_0)^{k-1}$. Тогда $\frac{f(z)-f(t)}{z-t}-f'(t)=(\frac{f_n(z)-f_n(t)}{z-t}-f'_n(t))+(f'_n(t)-f'(t))+\frac{(f(z)-f_n(z))-(f(t)-f_n(t))}{z-t}$. Первое слагаемое "маленькое", когда $z$ достаточно близко к $t$, второе - когда $n$ достаточно велико. Осталось разобраться с третьим.  $\frac{(f(z)-f_n(z))-(f(t)-f_n(t))}{z-t}=\sum_{k=n}^{\infty}c_k \big ( \frac{z^k-t^k}{z-t}\big ) =\sum_{k=n}^{\infty}c_k \big ( z^{k-1}+z^{k-2}t+...+zt^{k-2}+t^{k-1}\big ) \leq \sum_{k=n}^{\infty}c_k|t-z_0|^{k-1}$ - оно тоже мало при достаточно большом $n$. Значит, выбирая подходящие $n$ и окрестность точки $t$, можно добиться того, чтобы $|\frac{f(z)-f(t)}{z-t}-f'(t)|$ было маленьким, т.е. $f$ будет дифференцируема в точке $t$.
\end{proof}

\subsection{Билет 30. Теорема Стоуна-Вейерштрасса. Лемма об аппроксимации |x|.}%30


\begin{remark}
    $d(f,g) = \sup|f-g|$
\end{remark}

\begin{stat}
    (не надо рассказывать в билетах, но могут спросить как доп. вопрос) Пусть $K$ - компактное множество. $C(K)$ - множество непрерывных функций из $K$ в $\mathbb{R}$. Тогда $C(K)$ - полное метрическое пространство с метрикой $d(f,g) = \sup|f-g|=\max|f-g|$.
\end{stat}
\begin{proof}
    То, что это действительно метрическое пространство, понять легко. Докажем его полноту. Выберем любое $\epsilon > 0$. Так как $\{f_i\}$ - фундаментальная последовательность, то $ \exists N \in \mathbb{N} : \forall n, m>N \sup_{x \in K} |f_n(x)-f_m(x)| = d(f_n, f_m)<\epsilon$. В частности, для любого $x_0 \in K$ последовательность $\{f_n(x_0)\}_{n=1}^{\infty}$ фундаментальна, т.е. имеет предел $f(x_0)$. Более того, если $\forall n, m \geq N=N(\epsilon) |f_n(x_0)-f_m(x_0)|<\epsilon$, то и $|f(x_0)-f_N(x_0| \leq \epsilon$. Значит, стремление $f_n(x_0) \ra f(x_0)$ равномерно по $x_0$. 
    \\
    Хотим установить непрервыность $f$, из этого будет следовать полнота.  $|f(x)-f(x_0)| \leq |f(x)-f_n(x)|+|f_n(x)-f_n(x_0)|+|f_n(x_0)-f(x_0)| < 3\epsilon$, если мы выберем какое-нибудь достаточно большое $n$, чтобы для любого $x$ $|f(x)-f_n(x)| < \epsilon$ (по равномерности стремления $f_n$) и возьмём $x$ к $f$ достаточно близко к $x_0$, чтобы было выполнено $|f_n(x)-f_n(x_0)|$ - так можно сделать по непрерывности $f_n$.
\end{proof}

\begin{defn}
    $\mathcal{A}$ - \de{38}{Алгебра в пространстве непрерывных функций},
    если $\mathcal{A} \subset C(K)$ и $\mathcal{A}$ - линейное векторное 
    пространство такое, что $f,g \in \mathcal{A} \Rightarrow fg \in \mathcal{A}$.
\end{defn}


\begin{defn}
    \de{39}{Выделение/разделение точек}. 
    $\mathcal{A}$ \cursed{выделяет} точки, если
    $\forall x \in K \exists f \in \mathcal{A}: \ f(x)\not=0$. $\mathcal{A}$ \cursed{разделяет} точки, если
    $\forall x_1, x_2 \in K \exists f \in \mathcal{A}: \ f(x_1)\not=f(x_2)$
\end{defn}

\begin{theorem} (\de{40}{Теорема Стоуна-Вейерштрасса}).
    $\mathcal{A}$ - алгебра, $\mathcal{A} \subset C(K)$, $K$ - компактно и хаусдорфово,
    $\mathcal{A}$ выделяет и разделяет точки. Тогда $\overline{\mathcal{A}}=C(K)$.
\end{theorem}
\begin{proof} \ 
    \begin{lemma}
    Для любого $\epsilon > 0$ существует полином $p(x) \in \mathbb{R}[x]$ такой, что $\sup_{-1 \leq x \leq 1} ||x|-p(x)| < \epsilon$ 
    \end{lemma}
    \begin{proof}
        $|x|=\sqrt{x^2}=\sqrt{1-(1-x^2)}$, поэтому нам достаточно научиться приближать полиномами функцию $\sqrt{1-x}$, $x \in [0, 1]$.
        \\
        Зафиксируем $t \in [0, 1)$. Тогда функция $\sqrt{1-tx}$ приближается полиномами Тейлора с центром в нуле на отрезке $[0, 1]$: $\sqrt{1-tx}=P_{n, t}(x)+R_{n, t}(x)$, где остаток $R_{n, t}$ стремится к нулю равномерно по $x$. 
        \\
        Для любого $\epsilon > 0$ найдётся $t$, для которого выполнено неравенство $\sup_{x \in [0, 1]} |\sqrt{1-tx}-\sqrt{1-x}| < \frac{\epsilon}{2}$. Зафиксируем это $t$, и подберём полином $p$ такой, что $\sup_{x \in [0, 1]}|\sqrt{1-tx}-p(x)|< \frac{\epsilon}{2}$, откуда сразу получаем, что $\sup_{x \in [0, 1]} |p(x)-\sqrt{1-x}|<\epsilon$. Осталось лишь подставить в данный полином $1-x^2$.
    \end{proof}
    \begin{remark}
        \begin{itemize}
            \item Все полиномы $p$ не содержат мономов нечётной степени
            \item Можно считать, что для всех приближающих полиномов $p$ выполнено равенство $p(0)=0$.
        \end{itemize}
    \end{remark}
    \begin{proof}
       \begin{itemize}
            \item Очевидно следует из определения.
            \item Пусть $p$ - полином, приближающий с точностью $\epsilon$, т.е. $\sup_{x \in [-1, 1]} ||x|-p(x)| < \epsilon$, тогда, подставляя $x=0$, получаем, что $|p(0)| \epsilon$. Тогда из неравенства треугольнка получаем, что $p'(x)=p(x)-p(0)$ - полином, приближающий с точностью $2\epsilon$.
        \end{itemize} 
    \end{proof}
\end{proof}

%31
\subsection{Билет 31. Теорема Стоуна-Вейерштрасса. Завершение докзательства.}

\begin{lemma}
\begin{enumerate}
    \item Если $f \in \overline{\mathcal{A}}$ (т.е. приближается элементами алгебры), то и $|f| \in \overline{\mathcal{A}}$
    \item Если $f_1, f_2$ приближаются, то $\max(f_1(x), f_2(x))$ и $\min(f_1(x), f_2(x))$ тоже приближаются.
\end{enumerate}
\end{lemma}
\begin{proof}
    \begin{enumerate}
        \item Будем считать, что $|f| \leq 1$ (иначе просто разделим на $\max_{x \in K} |f(x)|$, который существует в силу компактности $K$). Пусть $p_{\epsilon}$ - полином (без свободного члена), приближающий $|x|$ с точностью $\epsilon$. Тогда $p_{\epsilon} \circ f \in \mathcal{A}$ (так как мы можем складывать элементы алгебры и умножать их друг на друга и на константу из $\mathbb{R}$), и $|p_{\epsilon}(f(x))-|f(x)||< \epsilon$ для любого $x \in K$.
        \item $\max(f_1, f_2)=\frac{f_1+f_2}{2}+\frac{|f_1-f_2}{2}$, аналогично с минимумом. А то, что сумма приближаемых приближаема и приближаемая, умноженная на константу, приближаемы, и так очевидно.
    \end{enumerate}
\end{proof}
\begin{lemma}
Пусть даны различные точки $x_1, x_2 \in K$, а также какие-то вещественные числа $u_1, u_2$. Тогда существует функция $f \in \mathcal{A}$, для которой $f(x_1)=u_1$, $f(x_2)=u_2$.
\end{lemma}
\begin{proof}
    Так как $\mathcal{A}$ разделяет точки: найдётся функция $f_1$ такая, что $f_1(x_1) \neq f_2(x_1)$. Рассмотрим функции $f_1$ и $f_1^2$ и попробуем найти вещественные числа $a, b$, для которых верно следующее:
    $\left\{
  \begin{array}{ccc}
    (af_1+bf_1^2)(x_1) & = & u_1 \\
    (af_1+bf_1^2)(x_2) & = & u_2 \\
  \end{array}
\right.$
Матрица этой СЛУ имеет определитель $f_1(x_1)f_1^2(x_2)-f_1^2(x_1)f_1^2(x_2)=f_1(x_1)f_1(x_2)(f_1(x_1)-f_1(x_2))$. Последня скобка не равна нулю в силу выбора функции $f$. Если ни $f_1(x_1)$, ни $f_1(x_2)$ не равны нулю, то СЛУ невырождена и имеет решение. Разберём другой случай. Тогда, не умаляя общности, $f_1(x_2) \neq 0$, и существует $f_2 \in \mathcal{A}$ такая, что $f_2(x_1) \neq 0$. Заменим $f_1$ на $f_1+\tau f_2$ и рассмотрим аналогичную систему и её определитель. Если $\tau$ достаточно мало, все три сомножителя будут ненулевыми, а в этом случае решение существует.
\end{proof}
\begin{lemma}
Пусть $f \in C(K)$, $\epsilon > 0$ и  $x_0 \in K$, тогда найдётся функция $g_{x_0} \in \overline{\mathcal{A}}$ такая, что $g(x_0)=f(x_0)$ и $g(x) \leq f(x)+\epsilon$.
\end{lemma}
\begin{proof}
    Для любой точки $\Tilde{x} \in K$ найдётся функция $g_{\Tilde{x}}$, для которой $g_{\Tilde{x}}(x_0)=f(x_0)$ и $g_{\Tilde{x}}(\Tilde{x})=f(\Tilde{x})$ по предыдущей лемме (если $\Tilde{x}=x_0$, то подойдёт $g_{\Tilde{x}}=f$). Так как $f$ непрерывна, то существует окрестность $V_{\Tilde{x}} \ni \Tilde{x}$, в которой $f$ и $g_{\Tilde{x}}$ не сильно отличаются: $|g_{\Tilde{x}}(y)-f(y)| < \epsilon$ для любого $y \in V_{\Tilde{x}}$. $\{V_{\Tilde{x}}\}_{\Tilde{x} \in K}$ - открытое покрытие компакта $K$ $\implies$ можно извлечь конечное подпокрытие, соответствующее точкам $\{g_{\Tilde{x_i}}\}_{i=1}^n$. Тогда функция $G=\min(g_{\Tilde{x_1}}, ... g_{\Tilde{x_1}})$ лежит в $\overline{\mathcal{A}}$ по лемме из начала билета, а также удовлетворяет равенству $G(x_0)=f(x_0)$ и неравенству $G(x) \leq f(x)+\epsilon$.
\end{proof}
Докажем, наконец, теорему Стоуна-Вейерштрасса. 
\begin{proof}
    Для каждой точки $x_k$ определим непрерывную функцию $g_k$ из предыдущей леммы и окрестность $V(x_k)$ такую, что $|g_k(y)-g_k(x_k)|<\epsilon$ и $|f(y)-f(x_k)|<\epsilon$ при всех $y \in V(x_k)$ (просто пересекали две окрестности, соответствующие двум неравенствам). $\{V_k\}_{x_k \in K}$ - открытое покрытие компакта $K$ $\implies$ можно извлечь конечное подпокрытие, соответствующее, не умаляя общности, точкам $x_1, ... x_n$. Для каждого $x_i$, соответствующая этой точке функция $g_i$ удовлетворяет следующим соотношениям:
    \begin{itemize}
        \item $g_k(x_k)=f(x_k)$
        \item $g_k(x) \leq f(x)+\epsilon$ для всех $x \in K$.
        \item $g_k \in \overline{\mathcal{A}}$.
    \end{itemize}

Определим функцию $G(x)=\max_{1 \leq i \leq n}(g_i)$. Тогда $G(x) \leq f(x)+\epsilon$. Получим теперь оценку в другую сторону. Для каждой точки $y \in K$ есть покрывающая её окрестность $V(x_k)$ для некоторого $k$. Это значит, что $|f(y)-f(x_k)|< \epsilon$ и $|g_k(y)-g_k(x_k)|<\epsilon$, но $g_k(x_k)=f(x_k)$, откуда по неравенству треугольника $|f(y)-g_k(y)|<2\epsilon$. Значит, $G(y) \geq g_k(y)>f(y)-2\epsilon$. Получаем, что $f$ приближается элементом $G \in \overline{\mathcal{A}}$, но этот элемент приближается элементами из $\mathcal{A}$ $\implies$ $G$ тоже приближается элементами $\mathcal{A}$, что и требовалось.
\end{proof}

%32
\subsection{Билет 32. Теорема о неподвижной точке. Приложение к дифференциальным уравнениям.}


\begin{defn}
    \de{41}{Сжимающее отображение} $T: X \to X$, $X$ - полное метрическое пространство с метрикой $\rho$
    $$\exists \alpha < 1: \ \rho(Tx, Ty) \leq \alpha \rho (x,y)$$ 
\end{defn}

\begin{theorem}
    У сжимающего отображения $T$ есть единственная неподвижная точка, т.е. такая, что $Tx=x$. 
\end{theorem}

\begin{proof}
    \textit{Единственность:} пусть $x$ и $y$ - две неподвижные точки отображения $T$, тогда 
    $\rho (x,y) = \rho (T x, T y) \leq \alpha \rho (x,y)$, откуда $1 \leq \alpha$, противоречие. \\
    \textit{Существование:} расммотрим орбиту $x$, т.е. $x, Tx, T(Tx)=T^2 x, ... , T^n x, ...$, такая последовательность фундаментальна:
    $$\forall \varepsilon > 0 \ \exists N \ \forall n,m > N: \ \rho (T^n x, T^m x) < \varepsilon $$
    действительно, при $n \leq m:$ 
    $\rho (T^n x, T^m y) \leq \alpha^n \rho (x, T^{m-n} x) \leq $  
    $\alpha^n \sum\limits_{k=0}^{\infty} \rho (T^k x, T^{k+1} x) \\$ 
    $\leq \alpha^n \sum\limits_{k=0}^{\infty} \alpha^k \rho (x, Tx) = $
    $\frac{\alpha^n \rho (x,Tx)}{1-\alpha}  < \varepsilon$. 
    А значит $\exists \lim\limits_{n \to \infty} T^n x = e \in X$ - неподвижная точка, 
    т.к. $T^n x \to e \Rightarrow T(T^n x) = T^{n+1} x \to Te = e$ (суть в том, что $Te$ отличается от $T^{n+1}e$ на сколь угодно мало,
    а значит $Te = \lim\limits_{n+1 \to \infty} T^{n+1} x = e$).  
\end{proof}

\begin{remark}
    $\rho(T^n x, T^m y) < \varepsilon$ при $n,m>N$.
\end{remark}
%33
\subsection{Билет 33. Топология в пространстве бесконечно дифференцируемых над R функций. Метризуемость.}

\begin{remark}
    Сходимость: будем говорить, что последовательность
    $\{f_j\}_{j\in\mathbb{N}} \subset C^\infty(R)$ сходится к $f \in C^\infty(R)$, 
    если для любого компактного множества $K \subset R$ функции $f_j$
    сходятся к $f$ равномерно на $K$.
\end{remark}

\begin{stat}
    $$
    d(f,g) = \sum\limits_{j,n\geq 0} \frac{ 2^{j+n} ||f^{(j)}-g^{(j)}||_{\infty, I_n}}{1+||f^{(j)}-g^{(j)}||_{\infty, I_n}}
    $$
    ($I_n$ - компакты, $||\cdot||_\infty = \sup||\cdot||$) тогда $d$ - метрика.
\end{stat}

\begin{stat}
    Топология порожденная такой метрикой и топология порожденная такой сходимостью совпадают.
\end{stat}

\subsection{Билет 34. Интеграл в смысле главного значения. Преобразование Гильберта. Гладкость.}
\begin{defn}
Пусть функция $f$ интегрируема по Риману на множестве $[a, b] \textbackslash \{x_0\}$, $x_0 \in (a, b)$. Будем говорить, что $f$ \de{42}{интегрируема в смысле главного значения} и писать $(p.v.) \int_a^b f(x) dx$, если существует предел $\lim_{\epsilon \ra 0_+}$ $\Big ( \int_a^{x_0-\epsilon} f(x)dx +\int_{x_0+\epsilon}^b f(x)dx \Big )$.
\end{defn}
\begin{defn}
Пусть $f$ - гладкая на $[a, b]$, $y \in (a, b)$. Отображение, которое функции $f$ сопоставляет функцию $(p.v.) \int_a^b \frac{f(x)}{y-x} dx$, называется \de{43}{преобразованием Гильберта}
\end{defn}
\begin{theorem}
В условиях предыдущего определения интеграл $(p.v.) \int_a^b \frac{f(x)}{y-x} dx$ существует.
\end{theorem}
\begin{proof}
Можем считать, что $y$ - середина отрезка $[a, b]$ (иначе просто откусим от бОльшей половины кусок, он не влияет на существование интеграла). Рассмотрим функцию $g(x)$, которая равна $f'(y)$, если $x=y$ и $\frac{f(x)-f(y)}{x-y}$ иначе. Она непрерывна на $[a, b]$, поэтому интегрируема по Риману (а тем более в смысле главного значения). Но Легко видеть, что функция $\frac{f(y)}{y-x}$ интегрируема в смысле главного значения, и её интеграл $(p.v.) \int_a^b \frac{f(y)}{y-x}$ равен нулю (симметрия относительно оси ординат). Поэтому, функция $\frac{f(x)}{y-x}=-\frac{f(x)-f(y)}{x-y}+\frac{f(y)}{y-x}$ тоже интегрируема в смысле главного значения.
\end{proof}
\begin{remark}
Заметим, что мы пользовались только непрерывностью $f$ на $[a, b]$ и дифференцируемостью $f$ в точке $y$. Последнее условие можно заменить на липшицевость в окрестности точки $y$: $|f(x)-f(y)| \leq C|x-y|^{a}$, $a>0$. Действительно, тогда $\lim_{t \ra y_-} \int_a^t g(x) dx \leq \lim_{t \ra y_-} \int_a^t C|y-x|^{a-1} dx=\lim_{t \ra y_-} -C((y-t)^a-(y-a)^a)$ - существует и конечен. Аналогично поступаем с $t \ra y_+$. получаем, что $\int_a^b g(x) dx$ - конечный несобственный интеграл, и наше рассуждение работает.
\end{remark}


\subsection{Билет 35. Аддитивные функции промежутка. Полукольца. Примеры.}
\begin{defn}
Пусть $X$ - множество произвольной природы, $A$ - система его подмножеств. Функция $\mu: A \ra [0, +\infty]$ называется \de{44}{аддитивной}, если для любого конечного набора попарно дизъюнктных множеств $a_1, ... +a_n \in A$ таких, что $a_1 \cup ... \cup a_n$ тоже лежит в $A$, верно равенство $\mu(a_1 \cup ... \cup a_n)=\mu(a_1)+...+\mu(a_n)$.
\end{defn}
\begin{exl}
\begin{enumerate}
Во всех примерах $X=\mathbb{R}$
    \item $A=\{<a, b> | a<b, a, b \in \mathbb{R}\}$, $\mu(<a, b>)=b-a$
    \item $A=\Rho(\mathbb{R})$, $\mu(a)=\min(|a|, \infty)$
    \item Пусть $g$ - неубывающая функция на $\mathbb{R}$. Она имеет разрывы только первого рода. Определим $g-_(x)=\lim_{t \ra x_-} f(t)$, $g_+(x)=\lim_{t \ra x_+} g(t)$. Пусть $A=\{<a, b> | a<b, a, b \in \mathbb{R}\}$.
    \\
    $\mu([a, b])=g_+(b)-g_-(a)$
    \\
    $\mu([a, b))=g_-(b)-g_-(a)$
    \\
    $\mu((a, b])=g_+(b)-g_+(a)$
    \\
    $\mu((a, b))=g_-(b)-g_+(a)$
\end{enumerate}
\end{exl}
\begin{defn}
Множество $A \subset \Rho(X)$ называется \de{45}{полукольцом}, если:
\begin{itemize}
    \item $\emptyset \in A$
    \item $a, b \in A$ $\implies a \cap b \in A$
    \item $a, b \in A$ $\implies $ $A \textbackslash B$ представляется в виде конечного объединения попарно дизъюнктных элементов из $A$.
\end{itemize}
\end{defn}
\begin{defn}
Множество $A \subset \Rho(X)$ называется \de{46}{кольцом}, если оно полукольцо, и $a, b \in A$ $\implies A \textbackslash B \in A$.
\end{defn}
\begin{defn}
Множество $A \subset \Rho(X)$ называется \de{47}{алгеброй}, если оно кольцо, и $X \in A$.
\end{defn}
\begin{remark}
Если $a, b \in A$, и $A$ - алгебра, то $a \cup b \in A$
\end{remark}
\begin{proof}
$a \cup b = X \textbackslash ((X \textbackslash a)\cap (X \textbackslash b))$
\end{proof}
\begin{remark}
Если на $A$ определить операции умножения $\times := \cap$ и сложения $+ := \Delta$ (симметрическая разность), то $A$ превратится в алгебраическое кольцо
\end{remark}
\begin{proof}
Просто проверить ручками.
\end{proof}
\begin{exl}
\begin{enumerate}
    \item $A=\{[a, b) | a<b, a, b \in \mathbb{R}\}$ - \textit{полукольцо ячеек} в $X=\mathbb{R}$
    \item $A=\{[a_1, b_1) \times ... \times [a_n, b_n) | a_i<b_i, a_i, b_i \in \mathbb{R}\}$ - \textit{полукольцо ячеек} в $X=\mathbb{R}^n$
\end{enumerate}
\end{exl}
\begin{theorem}
Если $A$ - полукольцо в $X$, а $B$ - полукольцо в $Y$, то $A \times B$ - полукольцо в $X \times Y$
\end{theorem}
\begin{proof}
Просто проверить ручками.
\end{proof}

\begin{remark}
    Поскольку $a = (a \cap b) \sqcup (a \textbackslash b)$, то $\mu(a)=\mu(a \cap b) + \mu(a \textbackslash)$. Вспоминаем, что функция $\mu$ неотрицательна и получаем монотонность: если $c \subseteq d$, то $\mu(c) \leq \mu(d)$
    \end{remark}

\subsection{Билет 36. Простые функции. Интеграл от простой функции.}
\begin{defn}
Функция $f: X \ra \mathbb{R}$ называется \de{48}{простой}, если $f(x)=\sum_{k=1}^n c_k \chi_{E_k}(x)$ для некоторого натуралнього $n$, вещественных $c_1, ... c_n$ и попарно дизъюнктных множеств $E_1, ... E_n$.
\end{defn}
\begin{defn}
\de{49}{Интегралом Лебега от простой функции} $f(x)=\sum_{k=1}^n c_k \chi_{E_k}(x)$ называется величина $I(f)=\sum_{k=1}^n c_k \mu(E_k)$, где $\mu$ - аддитивная функция на $X$.
\end{defn}
\begin{stat}
\begin{enumerate}
    \item Если $f \geq 0$, то $I(f) \geq 0$
    \item $I(f+g)=I(f)+I(g)$, где $f, g$ - простые (тогда их сумма тоже простая)
    \item $I(af)=aI(f)$, где $a \in \mathbb{R}$, $f$ - простая.
\end{enumerate}
\end{stat}

\begin{theorem}
Пусть $A \subset \Rho(X)$ - полукольцо с аддитивной функцией $\mu$, а $B \subset \Rho(Y)$ - полукольцо с аддитивной функцией $\nu$. Тогда $\phi (a \times b) := \mu (a) \nu(b)$ - аддитивная функция на полукольце $A \times B$.
\end{theorem}

\subsection{Билет 37. Сигма алгебры. Свойства.}
\begin{defn}
$\mu: A \ra [0, +\infty]$ \de{50}{счётно аддитивна}, если для любого счётного набора попарно дизъюнктных множеств $a_1, a_2,... \in A$ $\mu(\bigsqcup_{k=1}^{\infty}a_k)=\sum_{k=1}^{\infty} \mu(a_k)$
\end{defn}
\begin{defn}
Аддитивная функция $\mu$ называется \de{51}{регулярной}, если:
\begin{enumerate}
    \item $\mu(a)=\sup_{K \in A, K - \text{компакт}, K \subseteq a} \mu(K)$
    \item $\mu(a)=\sup_{G \in A, G - \text{открытое}, a \subseteq G} \mu(G)$
\end{enumerate}
\end{defn}
\begin{theorem}
(Теорема Александрова, её в билете не спрашивают) Регулярная мера счётноаддитивна.
\end{theorem}
\begin{proof}

\end{proof}
\begin{defn}
$A \subseteq \Rho(X)$ называется \de{52}{сигма-алгеброй}, если оно алгебра, и для любого счётного набора множеств из $A$ их объединение тоже лежит в $A$. 
\end{defn}
\begin{remark}
Можно рассматривать не счётное объединение, а счётное пересечение, поскольку $\bigcup_{i=1}^{\infty} a_i = \bigcap_{i=1}^{\infty} (X \textbackslash a_i)$ и наоборот.
\end{remark}
\begin{stat}
Пусть $A$ - сигма-алгебра, и $\mu$ - аддитивная функция на $A$. Следующие утверждения эквивалентны:
\begin{enumerate}
    \item $\mu$ счётноаддитивна
    \item Если $a_1 \subseteq a_2 \subseteq ... $ - счётная последовательность вложенных множеств из $A$, и $a=\bigcup_{k=1}^{\infty} a_k$, то $\mu(a)= \lim_{k \ra \infty} \mu(a_k)$.
\end{enumerate}
\end{stat}
\begin{proof}
\textbf{Из 1 в 2:} Определим последовательность попарно дизъюнктных множеств $\{b_k\}$ по правилу $b_1=a_1$, $b_{k+1}=a_{k+1} \textbackslash a_k$. Тогда $a = \bigcup_{k=1}^{\infty}b_k$. Так как $\mu$ счётноаддитивна, то $\mu(a) = \sum_{k=1}^{\infty} \mu(b_k) = \lim_{n \ra \infty} \sum_{k=1}^{n} \mu(b_k)  = \lim_{n \ra \infty} \mu(a_n)$.
\\
\textbf{Из 2 в 1:} Пусть $\{b_k\}$ - счётный набор попарно дизъюнктных множеств. Определим последовательность вложенных множеств $\{a_k\}$ по правилу $a_1=b_1$, $a_{k+1}=a_k \cup b_k$, тогда $\bigcup_{k=1}^{\infty} a_k = \bigcup_{k=1}^{\infty} b_k$ и  $\mu(\bigcup_{k=1}^{\infty} b_k) = \mu(\bigcup_{k=1}^{\infty} a_k) =  \lim_{n \ra \infty} \mu(a_n) = \lim_{n \ra \infty} \sum_{k=1}^{n} \mu(b_k) = \sum_{k=1}^{\infty} \mu(b_k) $.
\end{proof}

\subsection{Билет 38. Внешняя мера. Свойства.}



\begin{defn}
	\index{Внешняя мера}
	Отображение $ \tau\colon 2^X\to\RR_+\cup\{+\infty\} $ называется \de{53}{внешней мерой}, если:
	\begin{enumerate}
		\item[(1)] $ \tau(\emptyset)=0 $;
		\item[(2)] для любого множества $ A\subset X $ и последовательности такой $ \{A_k\}_{k=1}^{\infty} $, что $ A\subset\bigcup_{k=1}^{\infty}A_k  $, выполнено неравенство
		\begin{equation*}
			\tau(A)\le \sum_{k=1}^{\infty}{\tau(A_k)}.
		\end{equation*}
	\end{enumerate}
\end{defn}

\index{Внешняя мера!конечно-полуаддитивна}
\index{Внешняя мера!монотонна}
При $ A_n=\emptyset $ для всех $ n>N \in\NN $, из условия (2) следует, что внешняя мера \de{54}{конечно-полуаддитивна}. В частности, внешняя мера \de{55}{монотонна}, то есть 
$$ A\subset B \implies \tau(A)\le \tau(B), $$
так как можно взять $ A_1=B $ и $ A_k=\emptyset $ при всех $ k>1 $.

\begin{defn}
	\index{Измеримое множество!относительно внешней меры}
	Множество $ A\subset X $ будем называть \de{56}{измеримым относительно внешней меры} $ \tau $, если для всех $ E\subset X $ имеет место равенство
	\begin{equation}\label{eq:6}
		\tau(E)=\tau(E\cap A)+\tau(E\setminus A).
	\end{equation}
\end{defn}

\subsection{Билет 39. Предмера. Теорема Лебега-Каратеодори.}

\begin{defn}
	Пусть $ \mathcal{P} $ --- полукольцо в $ X $, $ \mu_0 $ --- счётно-аддитивная функция на $ \mathcal{P} $. Для каждого $ A\subset X $ определим
	\begin{equation}\label{eq:9}
		\mu^*(A)\coloneqq\inf\left\{\sum_{k=1}^{\infty}\mu_0(P_k)\,\middle|\,\{P_k\}\subset \mathcal{P} : A\subset \bigcup_{k=1}^{\infty}P_k\right\}.
	\end{equation}
	Отметим, что если $ A $ нельзя покрыть счётным объединением множеств из $ \mathcal{P} $, то $ \mu^*(A)=+\infty $.
\end{defn}

\begin{theorem}[\de{57}{Каратеодори}]
	\index{Теорема Каратеодори}
	Пусть $ \mu_0\colon\mathcal{P}\to\RR_+\cup\{+\infty\} $ --- счётно-аддитивная функция на полукольце $ \mathcal{P}\subset 2^X $, а $ \mu^* $ определено как в \eqref{eq:9}. Тогда:
	\begin{enumerate}
		\item $ \mu^* $ --- внешняя мера на $ X $.
		\item $ \s $-алгебра $ \mathfrak{U}_{\mu^*} $ содержит в себе $ \mathcal{P} $. 
		\item Если  $ \mu $ --- мера, получающаяся ограничением внешней меры $ \mu^* $ на $ \mathfrak{U}_{\mu^*} $, то \\ $ \mu(P)=\mu_0(P) $ для всех элементов $ P\in \mathcal{P} $.
		\item Если $ \mathfrak{A} $ --- $ \s $-алгебра, содержащая $ \mathcal{P} $, а $ \nu $ --- мера на $ \mathfrak{A} $ такая, что $ \nu(P)=\mu_0(P) $ для всех $ P\in\mathcal{P}, $
		то 
		$ \mu(A)=\nu(A) $ для всех $ A\in\mathfrak{A}\cap \mathfrak{U}_{\mu^*} $ таких, что $ \mu(A)<\infty. $
		
		Более того, если $\mu $ --- $ \s$-конечна, то условие конечности $ \mu(A) $ можно отбросить, то есть
		$ \mu(A)=\nu(A) $ для всех $ A\in \mathfrak{A}\cap \mathfrak{U}_{\mu^*}. $
	\end{enumerate}
\end{theorem}
\begin{proof}

	\

	\begin{enumerate}
		\item $ \mu^*(\emptyset)=0 $, так как $ \mu_0(\emptyset)=0. $ Проверим счётную полуаддитивность. Пусть
		$ A\subset\bigcup_{k=1}^{\infty}A_k $. Поскольку $ \mu^* $ определяется как инфимум, для каждого $ k\in\NN $ мы можем найти такой набор множеств $ \{P_{kj}\}_{j=1}^{\infty}\subset\mathcal{P} $, что 
		\begin{equation*}
			\sum_{j=1}^{\infty}\mu_0(P_{kj})\le \mu^*(A_k)+\frac{\epsilon}{2^k},
		\end{equation*}
		где $ \bigcup_{j=1}^{\infty}P_{kj}\supset A_k $, а $ \epsilon>0 $ --- некоторое  число.
		Тогда $ \bigcup_{k=1}^{\infty}\bigcup_{j=1}^{\infty}P_{kj}\supset A $, и по определению $ \mu^* $
		\begin{equation*}
			\mu^*(A)\le\sum_{k=1}^{\infty}\left(\sum_{j=1}^{\infty}\mu_0(P_{kj})\right)\le \sum_{k=1}^{\infty}\left(\mu^*(A_k)+\frac{\epsilon}{2^k}\right)\le \sum_{k=1}^{\infty}\mu^*(A_k)+\epsilon.
		\end{equation*}
		Поскольку $ \epsilon $ был произвольным, отсюда следует, что отображение $ \mu^* $ счётно-полуаддитивно и является внешней мерой.
		
		\item Пусть $  P\in\mathcal{P},\,E\subset X $. Проверим неравенство 
		\begin{equation*}
			\mu^*(E)\ge\mu^*(E\cap P)+\mu^*(E\setminus P).
		\end{equation*}
		Если $ \mu^*(E)=+\infty $, то оно очевидно. Иначе (опять же, пользуясь свойствами инфимума) выберем множества $ \{P_k\}_{k=1}^{\infty}\subset \mathcal{P} $ так, что $  E\subset\bigcup_{k=1}^{\infty}P_k, $
		\begin{equation*}
			\sum_{k=1}^{\infty}\mu_0(P_k)\le \mu^*(E)+\epsilon.
		\end{equation*}
		Тогда 
		$$ 
			E\cap P\subset \bigcup_{k=1}^{\infty}(P_k\cap P),\qquad E\setminus P\subset\bigcup_{k=1}^{\infty}(P_k\setminus P)=\bigcup_{k=1}^{\infty}\bigsqcup_{j=1}^{n_k}Q_{kj}, 
		$$ 
		где $ Q_{kj}\in\mathcal{P} $. По определению $ \mu^* $ имеем 
		\begin{equation*}
			\mu^*(E\cap P)+\mu^*(E\setminus P)\le \sum_{k=1}^{\infty}\left(\mu_0(P_k\cap P)+\sum_{j=1}^{n_k}\mu_0(Q_{kj})\right).
		\end{equation*}
		Заметим, что $ P_k=(P_k\cap P)\cup(P_k\setminus P)=(P_k\cap P)\sqcup \bigsqcup_{j=1}^{n_k}Q_{kj}$, а значит 
		\begin{equation*}
			\mu_0(P_k)=\mu_0(P_k\cap P)+\sum_{j=1}^{n_k}{\mu_0(Q_{kj})}.
		\end{equation*}
		Таким образом,
		\begin{equation*}
			\mu^*(E\cap P)+\mu^*(E\setminus P)\le \sum_{k=1}^{\infty}\mu_0(P_k)\le \mu^*(E)+\epsilon.
		\end{equation*}
		Устремляя $ \epsilon $ к нулю, получаем требуемое неравенство.
		Таким образом, мы доказали, что $ P\in \mathfrak{U}_{\mu^*} $, то есть, что $ \mathcal{P}\subset \mathfrak{U}_{\mu^*} $.
		
		\item Проверим, что $ \mu^*(P)=\mu_0(P) $ для всех $  P\in\mathcal{P} $. Поскольку $ P\subset P $, $ \mu^*(P)\le \mu_0(P) $. С другой стороны, если $  P\subset\bigcup_{k=1}^{\infty}{P_k} $, то $  P=\bigcup_{k=1}^{\infty}{(P_k\cap P)} $, и
		\begin{equation*}
			\mu_0(P)\le\sum_{k=1}^{\infty}\mu_0(P_k\cap P)\le \sum_{k=1}^{\infty}\mu_0(P_k).
		\end{equation*}
		Беря инфимум по $ P_k $, получаем 
		\begin{equation*}
			\mu_0(P)\le \inf{\sum_{k=1}^{\infty}\mu_0(P_k)}=\mu^*(P). 
		\end{equation*}
		Таким образом, $ \mu_0(P)=\mu^*(P) $.
		\item Пусть $ A\in \mathfrak{A}\cap \mathfrak{U}_{\mu^*} $. Тогда для произвольного набора $ \{P_k\}_{k=1}^\infty\subset \mathfrak{A}\cap \mathfrak{U}_{\mu^*} $ такого, что $ \bigcup_{k=1}^{\infty}P_k \supset A $, выполнено неравенство 
		\begin{equation*}
			\nu(A)\le\sum_{k=1}^{\infty}\nu(P_k)=\sum_{k=1}^{\infty}\mu_0(P_k). 
		\end{equation*}
		%\implies \nu(A)\le \mu(A). $$
		Беря в правой части инфимум по всем наборам $ P_k $, получаем, что $ \nu(A)\le\mu(A) $. 
		
		Поймём, что
		$ \mu(A\cap P)=\nu(A\cap P) $ для всех $ P\in\mathcal{P} $ таких, что $ \mu(P)<\infty $. Действительно, если бы это было не так, то получилось бы, что
		\begin{equation*}
			\mu(P)=\nu(P)=\nu(P\cap A)+\nu(P\setminus A)<\mu(P\cap A)+\mu(P\setminus A)=\mu(P),
		\end{equation*}
		а это невозможно.
		
		Если $ \mu(A)<\infty $ или мера $ \mu $ $ \s $-конечна, то множество $ A $ можно покрыть элементами $ P_k $ полукольца $ \mathcal{P} $ конечной меры, причём (как мы уже показывали), их можно считать дизъюнктными. Тогда
		\begin{equation*}
			\mu(A)=\mu\left(\bigsqcup_{k=1}^\infty(A\cap P_k)\right)=\sum_{k=1}^\infty\mu(A\cap P_k)=\sum_{k=1}^\infty\nu(A\cap P_k)=\nu(A).
		\end{equation*}
		% 		Пусть теперь $ \mu $ --- $ \s $-конечна, то есть $ \exists U_k\in U_{\mu^*} : \bigcup_{k=1}^{\infty}U_k=X $, $ \mu(U_k)<\infty $. В этом случае можно записать $ A=\bigcup_{k=1}^\infty{A_k} $, где $ \mu(A_k)<\infty $ и $ A_i\subset A_j $ для $ i<j $, а именно: 
		% 		$$ A_k\coloneqq A\cap \left(\bigcup_{i=1}^k{U_i}\right). $$
		% 		Тогда по непрерывности меры снизу получаем
		% 		\begin{equation}
		% 		    \mu(A)=\lim_{k\to\infty}\mu(A_k)=\lim_{k\to\infty}\nu(A_k)=\nu(A),
		% 		\end{equation}
		% 		что и требовалось. \qedhere
		Таким образом, последний пункт теоремы доказан. \qedhere
	\end{enumerate}
\end{proof}

\subsection{Билет 40. Теорема о структуре измеримых множества. Единственность продолжения.}

\begin{defn}
	\index{Измеримое множество!относительно внешней меры}
	Множество $ A\subset X $ будем называть \de{58}{измеримым относительно внешней меры} $ \tau $, если для всех $ E\subset X $ имеет место равенство
	\begin{equation}\label{eq:6}
		\tau(E)=\tau(E\cap A)+\tau(E\setminus A).
	\end{equation}
\end{defn}

\begin{defn}
	Пусть $ \mathcal{P}\subset 2^X $ --- полукольцо, $ \mu_0 $ --- счетно-аддитивная функция из $ \mathcal{P} $ в $ \RR_+\cup\{+\infty\}. $ Мера $ \mu $, построенная в теореме Каратеодори, называется \de{59}{стандартным продолжением} $ \mu_0 $.
\end{defn}

\subsection{Билет 41. Борелевская сигма-алгебра. Мера Лебега.}



\begin{defn}
	\index{Борелевская $ \sigma $-алгебра}
	Пусть $ (X,T) $ --- топологическое пространство. Тогда \de{60}{борелевской $ \sigma $-алгеброй} в $ X $ будем называть минимальную $ \sigma $-алгебру, содержащую $ T $ (то есть все открытые, а значит и замкнутые, множества).\footnote{В дальнейшем мы сможем определить понятие длины для всех множеств из борелевской $ \sigma $-алгебры над $ \RR $.}
\end{defn}

Заметим, что теорема Каратеодори даёт не только существование стандартного продолжения, но и формулу, по которой можно считать меру через исходную функцию $ \mu_0 $:
\begin{equation*}
	\mu(A)=\inf\left\{\sum_{k=1}^{\infty}\mu_0(P_k)\,\middle|\,\{P_k\}\subset \mathcal{P} : A\subset \bigcup_{k=1}^{\infty}P_k\right\}.
\end{equation*}
Из теоремы Каратеодори и утверждения \ref{prop:5} следует, что стандартное продолжение --- полная мера.

Мы показали, что если мера $\mu $  $\s$-конечна, то ее продолжение единственно. Можно привести примеры, показывающие, что в общей ситуации это условие нельзя отбросить.

\begin{defn}
	\index{Мера Лебега}
	Стандартное продолжение функции $ \l_1\colon\mathcal{P}_1\to\RR_+ $ называется \de{61}{мерой Лебега} на $ \RR $. 
\end{defn}



\subsection{Билет 42. Единственность меры Лебега. Регулярность меры Лебега.}

Так как $ \bigcup_{n\in\NN}[-n,n)=\RR $ и $ \l_1([-n,n))<\infty $ для всех $ n\in\NN $, то $ \l_1 $ $ \s $-конечна, то есть мера Лебега определена на $ \s $-алгебре $ \mathfrak{U}_{\l_1^*} $ единственным образом.

\subsection{Билет 43. Измеримые отображения. Свойства.} 

\begin{defn}
	\index{Измеримое пространство}
	Пара $ (X,\mathfrak{A}) $, где
	$ X $ --- множество, а $ \mathfrak{A} $ --- $ \s $-алгебра в $ X $,  называется \de{62}{измеримым пространством}. 
\end{defn}

Поскольку дальше в этом параграфе много утверждений связано с прообразами, введём следующие удобные обозначения:
\begin{align*}
&E(a<f<b)=f^{-1}((a,b))=\{x\in E : f(x)\in (a,b)\},\\
&E(f\le a)=f^{-1}([-\infty,a]),
\end{align*}
и так далее (здесь подразумевается, что $ f $ определено на $ E $).

\begin{defn}
	\index{Измеримая функция}
	Пусть $ E $ --- измеримое множество относительно $ \s $-алгебры $ \mathfrak{A} $, $ f\colon E\to \overline{\RR}=\RR\cup\{\pm\infty\} $. Говорят, что функция $ f $ \de{63}{измерима} относительно $ \s $-алгебры $ \mathfrak{A} $, если
	\begin{equation*}
		E(f>a)=f^{-1}\big((a,+\infty]\big)\in\mathfrak{A}\quad(\forall a\in\RR).
	\end{equation*}
\end{defn}

\subsection{Билет 44. Интеграл Лебега. Примеры.}

\begin{defn}
	Тройка $ (X,\mathfrak{A},\mu) $ называется \de{64}{пространством с мерой}, если $ X $ --- множество, $ \mathfrak{A} $ --- $ \s $-алгебра подмножеств, $ \mu $ --- мера на $ \mathfrak{A} $. 
\end{defn}

\begin{defn}
	Пусть $ f $ --- простая функция на $ X $, принимающая значения $ c_k $ на множествах разбиения $ E_k $, $ E $ --- измеримое множество. Тогда \de{65}{интегралом $ f $ по множеству $ E $} называется значение  
	\begin{equation}
	\int_E{f d \mu}\coloneqq\sum_{k=1}^N{c_k\mu(E\cap E_k)}.
	\end{equation}
	%$ f\in\sum_{k=1}^{N}{c_k\chi_{E_k}}, $ где $ c_k\in\R_+ $, $ E_k\in\mathcal{A} $. Пусть $ E\in\mathcal{A} $. Тогда
\end{defn}

\newpage

%ukazatel'. chto ne vidno blyat'?

\section{Указатель}

\hypertarget{uk}{Основные понятия.}

\begin{multicols}{2}

    \hyperlink{47}{алгебра} \ 
    
    \hyperlink{44}{аддитивная функция} \ 
    
    \hyperlink{38}{алгебра в пространстве непр.} \ 
    
    \hyperlink{60}{борелевская $\sigma$-алгебра} \ 

    \hyperlink{1}{вариация функции} \ 

    \hyperlink{53}{внешняя мера} \ 

    \hyperlink{38}{выделение точек} \ 

    \hyperlink{18}{гладкая функция} \ 

    \hyperlink{7}{гладкий путь} \ 

    \hyperlink{37}{голоморфная функция} \ 

    \hyperlink{22}{градиент} \ 

    \hyperlink{14}{дифференциал} \ 

    \hyperlink{9}{длина гладкого пути} \ 

    \hyperlink{5}{длина дуги кривой} \ 

    \hyperlink{29}{достаточное условие экстремума} \ 

    \hyperlink{6}{ест. параметр. кривой} \ 

    \hyperlink{34}{задача условного экстремума} \ 

    \hyperlink{3}{замена переменнной в вариации} \ 

    \hyperlink{28}{знак квадратичной формы} \ 

    \hyperlink{63}{измеримая функция} \ 

    \hyperlink{62}{измеримое пространства} \ 

    \hyperlink{49}{интеграл Лебега от пф} \ 

    \hyperlink{65}{интеграл по множеству} \ 

    \hyperlink{42}{интегрируема в смысле гз} \ 

    \hyperlink{27}{квадратичная форма} \ 

    \hyperlink{46}{кольцо} \ 

    \hyperlink{54}{конечно-полуадд. мера} \ 

    \hyperlink{4}{кривая} \ 

    \hyperlink{30}{критерий Сильвестра} \ 

    \hyperlink{20}{лемма о билипшицевости} \ 

    \hyperlink{23}{матрица Якоби} \ 

    \hyperlink{61}{мера Лебега} \ 

    \hyperlink{56}{множество измеримо относ. вм} \ 

    \hyperlink{13}{модуль вектора} \ 

    \hyperlink{26}{необходимое условие экстремума} \ 

    \hyperlink{16}{норма ло} \ 

    \hyperlink{12}{норма на еп} \ 

    \hyperlink{2}{ограниченная вариация} \ 

    \hyperlink{45}{полукольцо} \ 

    \hyperlink{32}{полярные координаты} \ 

    \hyperlink{43}{преобразование Гильберта} \ 

    \hyperlink{15}{производная} \ 

    \hyperlink{19}{производная по направлению} \ 

    \hyperlink{48}{простая функция} \ 

    \hyperlink{10}{простое вращение} \ 

    \hyperlink{64}{пространство с мерой} \ 

    \hyperlink{39}{разделение точек} \ 

    \hyperlink{51}{регулярная (аддит.) функция} \ 

    \hyperlink{41}{сжимающее отображение} \ 

    \hyperlink{52}{сигма-алгебра} \ 

    \hyperlink{24}{смешанная производная} \ 

    \hyperlink{59}{стандартное продолжение} \ 

    \hyperlink{50}{счётно аддитивная функция} \ 

    \hyperlink{33}{сферические координаты} \ 

    \hyperlink{57}{теорема Каратеодори} \ 

    \hyperlink{31}{теорема о неявной функции} \ 

    \hyperlink{21}{теорема об обратном отображении} \ 

    \hyperlink{36}{теорема Стокса-Зейделя} \ 

    \hyperlink{40}{теорема Стоуна-Вейерштрасса} \ 

    \hyperlink{8}{формула Лагранжа} \ 

    \hyperlink{25}{формула Тейлора} \ 

    \hyperlink{11}{формула Эйлера} \ 

    \hyperlink{35}{функция Лагранжа} \ 

    \hyperlink{17}{частная производная} \ 

\end{multicols}

\end{document}
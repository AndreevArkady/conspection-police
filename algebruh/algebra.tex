\documentclass[a4paper,100pt]{article}

\usepackage[utf8]{inputenc}
\usepackage[unicode, pdftex]{hyperref}
\usepackage{cmap}
\usepackage{mathtext}
\usepackage{multicol}
\setlength{\columnsep}{1cm}
\usepackage[T2A]{fontenc}
\usepackage[english,russian]{babel}
\usepackage{amsmath,amsfonts,amssymb,amsthm,mathtools}
\usepackage{icomma}
\usepackage{euscript}
\usepackage{mathrsfs}
\usepackage{geometry}
\usepackage[usenames]{color}
\hypersetup{
     colorlinks=true,
     linkcolor=red,
     filecolor=red,
     citecolor=black,      
     urlcolor=cyan,
     }
\usepackage{fancyhdr}
\pagestyle{fancy} 
\fancyhead{} 
\fancyhead[LE,RO]{\thepage} 
\fancyhead[CO]{\hyperlink{t2}{к списку объектов}}
\fancyhead[LO]{\hyperlink{t1}{к содержанию}} 
\fancyhead[CE]{текст-центр-четные} 
\fancyfoot{}
\newtheoremstyle{indented}{0 pt}{0 pt}{\itshape}{}{\bfseries}{. }{0 em}{ }

%\geometry{verbose,a4paper,tmargin=2cm,bmargin=2cm,lmargin=2.5cm,rmargin=1.5cm}

\title{Алгебра}
\author{Мастера конспектов}
\date{22 января 2020 г.}

\theoremstyle{indented}
\newtheorem{theorem}{Теорема}
\newtheorem{lemma}{Лемма}

\theoremstyle{definition} 
\newtheorem{defn}{Определение}
\newtheorem{exl}{Пример(ы)}

\theoremstyle{remark} 
\newtheorem{remark}{Примечание}
\newtheorem{cons}{Следствие}

\begin{document}

\newcommand{\resetexlcounters}{%
  \setcounter{exl}{0}%
} 

\newcommand{\resetremarkcounters}{%
  \setcounter{remark}{0}%
} 

\newcommand{\reseconscounters}{%
  \setcounter{cons}{0}%
} 

\newcommand{\resetall}{%
    \resetexlcounters
    \resetremarkcounters
    \reseconscounters%
}

\maketitle 

\newpage

\hypertarget{t1}{Честно говоря, ненависть} к этой вашей топологии просто невообразимая.
\tableofcontents

\newpage



\section{Билеты}



\subsection{Определение кольца. Простейшие следствия из аксиом. Примеры. Области целостности}

\medskip

\begin{defn}
    \hypertarget{n1}{\textcolor{red}{\textit{Кольцом}}} называется множество $R$ вместе с бинарными операциями $+$ и $\cdot$ (которые называются сложением и умножением соответственно), удовлетворяющим аксиомам:\

    \begin{itemize}
        \item операция сложения ассоциативна;
        \item по отношению к сложению существует нейтральный элемент;
        \item у каждого элемента есть обратный по сложению
        \item операция сложения коммутативна;
        \item умножение ассоциативно;
        \item умножение дистрибутивно по сложеиню.
    \end{itemize}
\end{defn}

Также можно добавить, что если на множестве выполныны три первые аксиомы, то оно будет называться \textit{группой}, а если выполнены первые четыре, то это уже \textit{абелева группа}. Нейтральный по сложению элемент кольца называют \textit{нулём}.

\begin{exl}
    Кольцо называется:\

    \begin{itemize}
        \item \textit{коммутативным}, если оно коммутативно по умножению;
        \item \textit{кольцом с единицей}, если оно содержит нейтральный элемент по умножению (единица);
        \item \textit{телом}, если в нём есть 1, и для любых $a\neq 0 \rightarrow a \cdot a^{-1}=a^{-1}\cdot a=1$;
        \item \textit{полем}, если это коммутативное тело;
        \item \textit{полукольцом}, если нет требования противоположного элемента по сложению.
    \end{itemize}
\end{exl}

\begin{cons}
    Некоторые следствия из аксиом:\

    \begin{itemize}
        \item $0\cdot a = 0$
        \begin{proof}
            \[
               0\cdot a = (0+0)\cdot a = 0\cdot a + 0\cdot a
            \]
            Прибавим к обеим частям $-0\cdot a$ и получим требуемое.
        \end{proof}
        \item Нейтральный элемент по сложению единственный
        \begin{proof}
            Рассмотрим их сумму справа и слева.
        \end{proof}
        \item $a\cdot 0 = 0$
        \begin{proof}
            \[
                a\cdot 1 = a \Longrightarrow (0+1)a = a \Longrightarrow 0\cdot a+1\cdot a = a \Longrightarrow 0\cdot a = 0
            \]
        \end{proof}
    \end{itemize}
\end{cons}

\begin{defn}
    Коммутативное кольцо $R$ с единицей, обладающее свойством
    \[
        xy=0 \Longrightarrow x=0 \vee y=0 \text{ }(\forall x, y\in R)
    \]
    называется \hypertarget{n2}{\textcolor{red}{\textit{областью целостности}}} или просто \textit{областью}.
\end{defn}

\begin{defn}
    Число $d\neq 0$ называется \hypertarget{n3}{\textcolor{red}{\textit{делителем нуля}}}, если существует такое $d'\neq 0$, что $dd'=0$.
\end{defn}

Нетрудно понять, что область целостности - в точности коммутативное кольцо с единицей без делителей нуля.

\resetall

\subsection{Евклидовы кольца. Евклидовость $\mathbb{Z}$. Неприводимые и простые элементы.}

\medskip

Для начала, некоторые связанные понятия, не упомянутые в билетах.

\begin{defn}
    Говорят, что $d$ \textit{делит} $p$ и пишут $d|p$, если $p=dq$ для некоторго $q\in R$.
\end{defn}

\begin{defn}
    Элемент $\varepsilon$ называется \textit{обратимым}, если он делит единицу, то есть существует такое $\varepsilon^{-1} \in R$, что $\varepsilon^{-1} \cdot \varepsilon = 1$. 
\end{defn}

\begin{defn}
    Будем говорит, что элементы $a$ и $b$ \hypertarget{n4}{\textcolor{red}{\textit{ассоциированы}}} и писать $a\sim b$, если выполнено одно из двух эквивалентных условий:
    \begin{itemize}
        \item существует обратимый элемент $\varepsilon$, для которого $a=\varepsilon b$;
        \item $a|b$ и $b|a$.
    \end{itemize}
\end{defn}

Покажем, что эти условия действительно эквивалентны.

\begin{proof}
    Докажем в обе стороны:\

    \textcolor{red}{$\Rightarrow$} Если $a=\varepsilon b$, то $\varepsilon^{-1}a=b$. Это и есть второе условие.\

    \textcolor{red}{$\Leftarrow$} Пусть $a=bc$ и $b=ac'$ для каких-то $c, c'$. Тогда $a = (ac')c=a(cc')\leftrightarrow a(1-cc')=0$. Тогда либо $a=0$, либо $cc'=1$, потому что делителей нуля в нашем кольце нет. В любом случае, $a$ и $b$ отличаются на обратимый: либо они оба равны нулю, либо $c$ - обратимый.
\end{proof}

А теперь, что касается самого билета.

\begin{defn}
    Область целостности  $R$ называется \hypertarget{n5}{\textcolor{red}{\textit{евклидовым кольцом}}}, если существует евклидова норма $N: R\rightarrow \mathbb{N}_0$ такая, что $N(0)=0$ и для любых элементов $a, b \in R$, где $b \neq 0$, существует меньший чем $b$ по норме элемент $r\in R$ такой, что выполнено равенство $a=bq+r$.
\end{defn}

\begin{exl}
    Кольцо целых чисел $\mathbb{Z}$ евклидово.
\end{exl}

\begin{proof}
    Пусть у нас имеются целое число $a$ и ненулевое целое $b$. Тогда существуют такие целые числа $q$ и $r$, что модуль $r$ меньше модуля $b$, а также $a=bq+r$. Отметим на оси все ератные $b$. Тогда если число $a$ попало на отрезок $[kb, (k+1)b]$, $k$ будет частным, а $a-kb$ - остатком. Дальнейшую формализация можно провести индукцией.
\end{proof}

Опять несколько небольших новых определений перед тем как перейти к последнему пункту билета (их можно упустить).

\begin{defn}
    Пусть $R$ - область целостности; $a,b\in R$. Элемент $d\in R$ называется \hypertarget{n6}{\textcolor{red}{\textit{наибольшим общим делителем}}} $a$ и $b$, если 
    \begin{itemize}
        \item $d|a$ и $d|b$;
        \item для любого $d'\in R$, который также делит $a$ и $b$, выполнено также, что он делит $d$.
    \end{itemize}
\end{defn}

\begin{theorem}
    (О линейном представлении НОД в евклидовых кольцах). Пусть $R$ - евклидово кольцо, $a, b\in R$.Тогда существуют $d:=\gcd (a,b)$ и такие $x,y\in R$, что $d=ax+by$.
\end{theorem}\

Теперь про простые и неприводимые.

\begin{defn}
    Пусть $R$ - область. Необратимый элемент $p\in R$ - \hypertarget{n7}{\textcolor{red}{\textit{неприводимый}}}, если 
    \[
        \forall d\in R:d\vert p \Longrightarrow d\sim 1 \vee d\sim p
    \]
\end{defn}

\begin{defn}
    Пусть $R$ - область. Ненулевой необратимый элемент $p\in R\backslash 0$ называется \hypertarget{n8}{\textcolor{red}{\textit{простым}}}, если $\forall a,b\in R: p\vert ab \Longrightarrow p\vert a \vee p\vert b$.
\end{defn}

\begin{lemma}
    (Простые $\subset$ неприводимые). Если $p$ - простой элемент произвольного коммутативног кольцв с единицей, то $p$ - неприводим.
\end{lemma}

\begin{proof}
    Пусть $d$ - какой-то делитель $p$, что эквивалентно равенству $p=da$ для какого-то $a$. Проверим, что либо $d\sim 1$, либо $d\sim p$. Раз $p$ - простой, то либо он делит $d$, либо он делит $a$. Если первое, что сразу $d\sim p$.Если второе, перепишем в виде $da=p\vert a$. Это то же самое, что $bda=a$ для некоторого $b$. Здесь либо $a=0$, то тогда $p=o$, что невозможно по определению простого, либо мы можем сократить на $a$ и получим $bd=1$, тогда $d$ ассоциирован с 1.
\end{proof}

Теперь немного добавки про простые и неприводимые, на всякий случай.\\

\begin{lemma}
    (Неприводимые $\subset$ простые в ОГИ). Пусть $p$ - неприводимый в области главных идеалов. Тогда $p$ - простой.
\end{lemma}

\begin{proof}
    Пусть $p\vert ab$, хотим показать, что $p\vert a \vee p\vert b$. Воспользуемся тем, что мы в области главных идеалов: $(p,a)=(d)$, где $d:=\gcd (a, p)$, а тогда $px+ay=d$ для каких-то $x, y$. $d\vert p$, воспользуемся неприводимостью $p$: либо $d\sim p$, либо $d\sim 1$.\
    
    В первом случае $p\vert d$, тогда $p\vert d\vert a$.\

    Во втором случае можно после домножения на обратимые считать, что $px+ay=1$. Потом домножим на $b:pbx+aby=b$. $p$ явно делит первое слагаемое, ровно как и второе (по предположению). Значит, $p\vert b$.\

    В любом случае, приходим к желаемому.
\end{proof}

\resetall

\subsection{Идеалы, главные идеалы. Евклидово кольцо как кольцо главных идеалов}

\begin{defn}
    Подмножество 
    \[
        (a_1, \dots, a_n):=\{a_1x_1+\dots+a_nx_n\vert x_i\in R \text{ для всех } i\}
    \]
    коммутативного кольца $R$ называется \hypertarget{n9}{\textcolor{red}{\textit{идеалом}}}, порождённым $a_1, \dots, a_n$.
\end{defn}

\begin{defn}
    Подкольцо $I$ кольца $R$ называется \textit{левым идеалом}, если оно замкнуто относительно домнодения слева на элементы кольца: $RI=I$. Соответственно, также различают \textit{правые} и \textit{двусторонние идеалы}. 
\end{defn}

Также идеал можно задать следующими свойствами:
\begin{itemize}
    \item $\forall x, y\in I \Longrightarrow x+y\in I$;
    \item $\forall x\in I, \forall r\in R \Longrightarrow xr\in I$;
    \item $-x\in I$;
    \item $I$ - непустой.
\end{itemize}

\begin{defn}
    Идеал называется \textit{главным}, если он порождён одним элементом.
\end{defn}

\begin{defn}
    \hypertarget{n10}{\textcolor{red}{\textit{Область главных идеалов}}} - область целостности, в который каждый идеал главный.
\end{defn}

\begin{theorem}
    (Евклидовы кольца $\subset$ ОГИ). Пусть $R$ - евклидово кольцо, $I\unlhd R$ - идеал. Тогда $I$ - главный.
\end{theorem}

\begin{proof}
    Найдём элемент, который порождает идеал $I$.\
        
    Вырожденный случай: если $I=\{0\}$, тогда $I=(0)$.\

    Иначе возьмём $d\in I\backslash 0$ с минимальной нормой (по принципу индукции мы можем это сделать). Хотим показать, что $I=(d)$. Покажем это в обе стороны.\

    \textcolor{red}{$\Rightarrow$} Легко видеть, что $(d) \subset I$.\

    \textcolor{red}{$\Leftarrow$} Пусть $a \in I$, тогда поделим $a$ на $b$ с остатком: $a=bd+r$. Предположим, $r\neq 0$, $N(r)<N(d)$. Выразим $r$ линейной комбинацией $a\in I$ и $d \in I$: $r=a-bd\in I$ - противоречие с минимальностью нормы $d$. Значит, $r=0$, а тогда $a=bd\in (d)$.
\end{proof}

\resetall

\subsection{Основная теорема арифметики}

Сначала опять немного информации, которая к билету не относится, но к нему логично подводит.\

\begin{defn}
    Коммутативное кольцо с единицей $R$ удовлетворяет \hypertarget{n11}{\textcolor{red}{\textit{условию обрыва возрастающих цепей главных идеалов}}} или, что то же самое, является \textit{нетёровым кольцом}, если не существует бесконечной строго возрастающей цепочки главных идеалов $(d_1)\subsetneq (d_2)\subsetneq \dots$ Иначе говоря, бесконечной цепочки $\dots \vert d_2\vert d_1$, где все $d_i$ попарно не ассоциированы.
\end{defn}

\begin{theorem}
    (ОГИ $\subset$ нетёровы кольца). Область главных идеалов удовлетворяет условию обрыва возрастающих цепей главных идеалов (далее - УОВЦГИ).
\end{theorem}

\begin{proof}
    Предположим, что нашлась такая бесконечная цепочка $\{d_i\}$. Объединим $I:=\bigcup_{i=0}^\infty (d_i)$.\

    Покажем, что $I$ - идеал. $0\in I$. Пусть $u\in (d_i)$ и $v\in (d_j)$, где $i\leq j$, проверяем остальные условия. $u+v\in(d_j)$, потому что $u\in (d_j)$, с остальными аналогично, не очень сложно.\ 

    Вспомним, что мы находимся в ОГИ, то есть, каждый идеал главный. Пусть $d$ - \textit{генератор} $I$ ($I=(d)$). Любой $(d_i)$ строго содержится в $(d_{i+1})$, а этот содержится в $(d):(d_i)\subsetneq (d_{i+1})\subset (d)$, значит, любой из $\{(d_i)\}$ строго содержится в $(d)$. Но сам генератор $d$ тоже должен принадлежать какому-то из $\{(d_i)\}$, а значит, на каком-то моменте $(d)\subset (d_i)$. Противоречие.
\end{proof}

\begin{defn}
    Кольцо называется \hypertarget{n12}{\textcolor{red}{\textit{факториальным}}}, если одновременно выполнено:
    \begin{itemize}
        \item $R$ - область;
        \item любой неприводимый элемент $R$ - простой;
        \item $R$ - нетёрово.
    \end{itemize}
\end{defn}

\begin{exl}
    Как мы уже знаем, ОГИ $\subset$ факториальные кольца.
\end{exl}

А теперь, к основному.\\

\begin{theorem}
    (\hypertarget{n13}{\textcolor{red}{\textit{Основная теорема арифметики}}}). Пусть $R$ - факториальное кольцо.\ 

    Тогда любой элемент $x\in R$, если он не нуль и не обратимый, представляется в виде $r=p_1\dots p_n$, где $n\geq 1$, а $\{p_i\}$ - простые.\ 

    При этом, если $r=q_1\dots q_m$ - другое такое разложение, то $m=n$ и существует перестановка индексов $\pi: n\rightarrow n$, такая, что $p_i\sim q_{\pi_i}$ для всех $i$.
\end{theorem}

\begin{proof}
    Докажем существование. Зафиксируем $x$. Если он неприводимый, то он и простой по определению факториального кольца, поэтому сам будет своим подходящим разложением. Пусть $x=yz$, где $y, z \nsim 1$.Если $y$ необратим и приводим, разложим и его: $y=y_1 z_1$, где $y_1, z_1 \nsim 1$. Будем раскладывать так игреки, пока можем, и получим строго возрастающую цепочку идеалов $(y)\subsetneq(y_1)\subsetneq(y_2)\subsetneq\dots$ Вспомним нетёровость нашего кольца: бесконечно возрастать она не может, значит, на каком-то моменте заработаем для $x$ один не приводимый делитель $p: x=pw$ для какого-то $w$. Если $w$ необратим и приводим, разложим и его: $w=p_1 w_1$. Продолжим и получим ещё одну возрастающую цепочку идеалов: $(x)\subsetneq (w)\subsetneq (w_1)\subsetneq \dots$ К тому времени, когда она оборвётся, у нас будет разложение $x$ в конечное произведение неприводимых: $x=p_1\dots p_n$. Существование доказано.\ 

    Теперь перейдём к доказательству единственности. Разложим двумя способами: $r=p_1\dots p_n=q_1\dots q_m$. По индукции пожно вывести из определения простого, что\\

    \begin{lemma}
        Если $p$ - простой и $p\vert a_1\dots a_n$, то $p\vert a_i$ для какого-то $i$.
    \end{lemma}\

    Воспользуемся этим фактом:например, мы теперь знаем: что $q_m\vert p_i$ для какого-то $i$. Но $p_i$ неприводим, поэтому любой его делитель либо обратим, либо ассоциирован с ним. $q_m$ не боратим, так как он простой; значит, $q_m \sim p_i$. Переставим $p_i$ и $p_n$ и считаем, что $q_m$ теперь $\sim p_n$. Осталось вывести следующий факт:\\

    \begin{lemma}
        Пусть $a\sim b$, $ac\sim bd$, а $b\neq 0$. Тогда $c\sim d$.
    \end{lemma}

    \begin{proof}
        $a=\varepsilon b$и $ac=\varepsilon bc =\nu bd$ для каких-то обратимых $\varepsilon$ и $\nu$. Последнее равенство можем сократить на $b\neq 0$, потому что мы в области.
    \end{proof}

    Теперь $p_1\dots p_{n-1}\sim q_1\dots q_{m-1}$. Можем теперь сказать, что равенство $p_1\dots p_{n-1} = q_1\dots q_{m-1}$ верно по предположению индукции по $n$. Так же по индукции $n=m$, потому что получим противоречие, если какая-то из серий сомножителей $\{p_i\}$, $\{q_i\}$ закончится раньше.
\end{proof}

\begin{exl}
    Обыкновенное кольцо $\mathbb{Z}$ $\in$ евклидовы кольца $\subset$ ОГИ $\subset$ факториальные кольца.
\end{exl}

\resetall

\subsection{Кольцо вычетов $\mathbb{Z}/_{n\mathbb{Z}}$. Китайская теорема об остатках}

\begin{exl}
    Множество $\mathbb{Z}/_{n\mathbb{Z}} = \{[0], \dots, [n-1]\}$ остатков при делении на $n\in \mathbb{N}$ - коммутативное кольцо с единицей. \hypertarget{n14}{\textcolor{red}{\textit{Кольцо вычетов}}} (остатков) по модулю.
\end{exl}

\begin{defn}
    $m$, $n$ \textit{взаимно просты}, если $(m,n)=(1)=R$.
\end{defn}

\begin{lemma}
    Пусть $R$ - факториальное кольцо, $m, n\in R$ - взаимно простые элементы. Пусть, к тому же, $m$ и $n$ - делители $r:m,n\vert r$. Тогда их произведение тоже делит $r:mn\vert r$.
\end{lemma}

\begin{proof}
    Можно вывести из ОТА.
\end{proof}

\begin{theorem}
    (\hypertarget{n15}{\textcolor{red}{\textit{Китайская теорема об остатках}}}). Если $(m,n)=(1)$, то $\mathbb{Z}/(m)\times \mathbb{Z}/(n)\cong \mathbb{Z}/(mn)$.
\end{theorem}

\begin{proof}
    Пусть $x$ - классы, соответствующие числу $x$ в $\mathbb{Z}/(m)$ и $\mathbb{Z}/(n)$, соответственно. Рассмотрим гомеоморфизм $f=x\mapsto ([x]_m, [x]_n)$.\ 

    Его ядро - числа, которые делятся и на $m$, и на $n$, а поскольку они взаимно просты, то и на $mn$. Значит, $\text{Ker} f= (mn)$.\ 

    Проверим $f$ на сюръективность. Для этого просто хитро покажем, что $\text{Im} f \cong \mathbb{Z}/(mn)$. Тогда $mn = \vert \mathbb{Z}/(mn)\vert = \vert \text{Im} f\vert$. При этом $\text{Im} f \subset \mathbb{Z}/(m)\times \mathbb{Z}/(n)$ по определению (подкольцо) и $\vert \mathbb{Z}/(m)\times \mathbb{Z}/(n) \vert = mn$ простым подсчётом, откуда следует, что $\text{Im}=\mathbb{Z}/(m)\times \mathbb{Z}/(n)$.
\end{proof}

\begin{cons}
    $\mathbb{Z}/(n)$ - область целостности $\Longleftrightarrow$ $n$ - простое.
\end{cons}

\resetall

\subsection{Определение поля. $\mathbb{Z}/_{p\mathbb{Z}}$ как поле. Поле частных целостного кольца}

Напомним ещё раз определение поля.\

\begin{defn}
    \hypertarget{n16}{\textcolor{red}{\textit{Поле}}} - коммутативное кольцо с единицей, в котором также существует обратный элемент по умножению для ненулевых элементов.
\end{defn}

\begin{exl}
    $\mathbb{Z}/_{p\mathbb{Z}}$ - поле.
\end{exl}

\begin{proof}
    Мы уже много чего знаем про эту структуру (см. конец предыдущего билета). Для доказательства вышеприведённого факта нужно показать, что у каждого элемента есть обратный по умножению (кроме, конечно, нуля). Рассмотрим ненулевой элемент $a$, и умножим его на все остатки по модулю $p$, получим $\{0a, 1a, \dots, (p-1)a\}$. Заметим, что все полученные остатки различны. Предположим противное: $ka\equiv ma \Leftrightarrow (k-m)a\equiv 0$, но так как мы находимся в области, то либо $a=0$ (сразу нет), либо $k-m=0$, но так как они оба меньше $p$, то такого тоже, очевидно, не бывает. Тогда мы получили, что все остатки, полученные таким образом, различны. Но так как их ровно $p$, то найдётся и равный 1, элемент на который мы умножаем в том случае и будет обратным к $a$.
\end{proof}

В общем и целом, мы сейчас будем получать что-то вроде $\mathbb{Q}$, но над любым кольцом $R$. Введём отношение $\sim$ на множестве пар $R\times (R\backslash 0)$. Пусть $(a, b)\sim(a', b') \Leftrightarrow ab'=a'b$. Проверим, что мы получили отношение эквивалентности:

\begin{proof}
    Нужно показать рефлексивность, симметричность и транзитивность. Первые два утверждения очевидны, покажем последнее. Пусть $(a, b)\sim(a', b')$ и $(a', b')\sim(a'', b'')$, мы хотим показать, что $(a, b)\sim(a'', b'')$, то есть, $ab''=a''b$. Воспользуемся тем, что мы находимся в области целостности - домножим левую часть последнего равенства на ненулевой $b'$ и преобразуем, используя гипотезы:
    \[
        (ab')b''=b(a'b'')=bb'a''.
    \]
    Теперь сократим на $b'$.
\end{proof}

\begin{defn}
    Фактор $R/{\sim}$ называется \hypertarget{n18}{\textcolor{red}{\textit{полем частных}}} области целостности $R$ и обозначается за $\text{Frac}R$. Элементы будем обозначать дробями.
\end{defn}

Сложение и умножение определяется как в обычной жизни. Осталось проверить, что это действительно поле.

\begin{proof} Нужно выполнить совсем немного проверок: \
    \begin{itemize}
        \item $0\over 1$ - нуль;
        \item $1\over 1$ - единица;
        \item $\frac{-a}{b}$ - обратный к $\frac{a}{b}$ по сложению;
        \item $\frac{b}{a}$ - обратный к $\frac{a}{b}$ по умножению для ненулевых.
    \end{itemize}
\end{proof}

\resetall

\subsection{Определение гомоморфизма и изоморфизма колец. Фактор-кольцо}

\begin{defn}
    Пусть $R$ и $S$ - кольца. Функция $f:R\rightarrow S$ называется \hypertarget{n19}{\textcolor{red}{\textit{гомоморфизмом колец}}}, если для произвольных элементов выполняется
    \begin{itemize}
        \item $f(r_1+r_2)=f(r_1)+f(r_2)$;
        \item $f(r_1r_2)=f(r_1)f(r_2)$.
    \end{itemize}
\end{defn}

\begin{lemma}
    Если $f$ - гомоморфизм, то $f(0)=0$ и $f(-r)=-r$.
\end{lemma}

\begin{proof}
    В обоих пунктах - подсчёт двумя способами:\ 

    \begin{itemize}
        \item $f(0)+f(0)=f(0+0)=f(0)$;
        \item $f(r)+f(-r)=f(r+(-r))=f(0)=0$
    \end{itemize}
\end{proof}

Кстати говоря, не любой гомоморфизм сохраняет единицу.

\begin{exl}
    Пусть $f:r\rightarrow R\times S$ и $f=r\mapsto (r,0)$. Тогда $f(1)=(1,0)\neq 1$.
\end{exl}

\begin{defn}
    Если для гомоморфизма $f$ выполнено $f(1)=1$, то говорят, что он \textit{сохраняет единицу}.
\end{defn}

С гомоморфизмом связаны два важных понятия, которые мы рассмотрим далее.

\begin{defn}
    \hypertarget{n20}{\textcolor{red}{\textit{Ядро}}} $\text{Ker}f$ гомоморфизма $f:R\rightarrow S$ - полный прообраз нуля, $f^{-1}(0)$.
\end{defn}

\begin{lemma}
    Гомоморфизм $f$ инъективен тогда и только тогда, когда его ядро тривиально: $\text{Ker}f=\{0\}$.
\end{lemma}

\begin{proof}
    Потому что $f(x_1)=f(x_2)\Longleftrightarrow f(x_1-x_2)=0$
\end{proof}

\begin{lemma}
    $\text{Ker}f$ - двусторонний идеал в $R$.
\end{lemma}

\begin{proof}
    Пусть $k\in \text{Ker}f$, тогда для любого $r\in R$ $f(rk)=f(r)f(k)=f(r)\cdot 0 = 0 = 0\cdot f(r)=f(kr)$. Ещё, например, $f(k_1+k_2)=f(k_1)+f(k_2)=0$. Остальные пункты из определения так же очевидны.
\end{proof}

\begin{defn}
    \hypertarget{n21}{\textcolor{red}{\textit{Образ}}} области определения гомоморфизма $f$ обозначается как $\text{Im}f$.
\end{defn}

\begin{lemma}
    Если $f:R\rightarrow S$ - гомоморфизм, то $f(R)$ - кольцо.
\end{lemma}

\begin{proof}
    $f(a)+f(b)=f(a+b)$, $f(a)f(b)=f(ab)$ - как раз.
\end{proof}

\begin{defn}
    \hypertarget{n22}{\textcolor{red}{\textit{Изоморфизм}}} - биективный гомоморфизм. Пишут $R\cong S$, если между ними существует изоморфизм.
\end{defn}

А теперь про фактор-кольца.

\begin{defn}
    Пусть $R$ - кольцо (возможно, некоммутативное и без единицы), а $I$ - двусторонний идеал. Говорят, что $a$ \hypertarget{n23}{\textcolor{red}{\textit{сравнимо}}} с $b$ по модулю $I$ и пишут $a\equiv b \mod I$, если $a-b\in I$.
\end{defn}

\begin{lemma}
    Сравнимость по модулю - отношение эквивалентности.
\end{lemma}\

Так как мы получили отношение эквивалентности, по нему можно факторизовать. Тогда аналогами классов эквивалентности становятся множества вида $[a]:=\{b\in \mathbb{R}\vert b\equiv a \mod I\}$. Обозначим кмножество всех этих классов за $R/I$. Осталось ввести структуру кольца на этом множестве.\ 

Определим действия: $[a]+[b]=[a+b]$ и $[a][b]=[ab]$. Нетрудно понять, что действия над классами не зависят от выбора \textit{представителя}. Сложение вообще очевидно, а при умножении нужно "прибавить и вычесть",  чтобы собрать.\\

\begin{theorem}
    Пусть $R$ - произвольное кольцо, возможно, некоммутативное и без единицы; $I\trianglelefteq R$ - двустронний идеал.\ 

    Обозначим за $R/I$ фактор $R$ по отношению эквивалентности $\{a\equiv b\vert a-b\in I\}$, за $[a]$ - класс эквивалентности элемента $a\in R$.\ 

    Тогда:
    \begin{itemize}
        \item операции $[a]+[b]=[a+b]$ и $[a][b]=[ab]$ определены корректно и задают на $R/I$ структуру кольца;
        \item если $R$ коммутативно, то $R/I$ - тоже;
        \item если $R$ - кольцо с единицей, то $[1]$ - единица $R/I$.
    \end{itemize}
\end{theorem}

\begin{proof}
    В первом пункте мы уже проверили все неочевидные пункты в определении кольца, остальное - тривиально.
\end{proof}

\begin{defn}
    $R/I$ - \hypertarget{n24}{\textcolor{red}{\textit{фактор-кольцо}}} $R$ по $I$.
\end{defn}

\resetall

\subsection{Теорема о гомоморфизме}

\begin{theorem}
    (\hypertarget{n25}{\textcolor{red}{\textit{Теорема о гомоморфизме}}}). Пусть $f:R\rightarrow S$ - гомоморфизм колец. Тогда $f(R)\cong R/\text{Ker}f$.
\end{theorem}

\begin{proof}
    Что мы будем делать по сути: вместо того, чтобы сразу отправлять элемент из $R$ в $S$ посредством $f$, сначала спроецируем его в $R/\text{Ker} f$ и оттуда уже отобразим в $f(R)$. Проверяем следующее для формальности:\ 

    \begin{itemize}
        \item \textit{Корректность определения}. Пусть $[r]=[r']$. Тогда $r'-r\in \text{Ker} f$, что равносильно $f(r'-r)=0$, а тогда $f(r)=f(r')$.
        \item \textit{Сюръективность}. По определению $f(R)$ любой элемент оттуда - это $f(r)$ для какого-то элемента $r\in R$, а $f(r)$ - образ $[r]$ при нашем отображении.
        \item \textit{Сюръективность}. Пусть $f(r_1)=f(r_2)$, тогда $f(r_1-r_2)=0$. Значит, $r_1-r_2\in \text{Ker} f$, что эквивалентно $r_1\equiv r_2 \mod \text{Ker} f$.
        \item \textit{Сохраняет операции}. $\varphi([a])+\varphi([b])=\varphi([a]+[b])=\varphi([a+b])=f(a+b)=f(a)+f(b)=\varphi([a])+\varphi([b])$. С умножением - агалогично.
    \end{itemize}
\end{proof}

Так как билет и так короткий - припишем сюда ещё одну теорему, которой почему-то нет в билетах.\\

\begin{theorem}
    (\hypertarget{n26}{\textcolor{red}{\textit{Универсальное свойство фактор-кольца}}}). Пусть $R$ - кольцо, $I\trianglelefteq R$ - двусторонний идеал, $\pi: R\rightarrow R/I$ - канонический гомоморфизм, $\varphi: R\rightarrow S$ - гомоморфизм колец, ядро которого содержит $I$: $\varphi(I)=\{0\}$. Тогда:\

    \begin{itemize}
        \item существует единственный гомоморфизм $\bar{\varphi}:R/I\rightarrow S$ такой, что $\varphi=\bar{\varphi}\circ \pi$;
        \item $\bar{\varphi}$ задаётся формулой $\bar{\varphi}=[x]\mapsto \varphi(x)$.
    \end{itemize}
\end{theorem}

\begin{proof}
    Раз уж теоремы в списке нет, то доказывать её не будем. Если вкратце, то сначала несложно проверяется единственность, затем - корректность, и, наконец, рутинная проверка на гомоморфизм.
\end{proof}

\resetall

\subsection{Кольцо многочленов. Целостность и евклидовость кольца многочленов над полем}

\begin{defn}
    \hypertarget{n27}{\textcolor{red}{\textit{Многочлен}}} - комбинация вида $\sum_{i=0}^\infty a_ix^i$, где почти все (кроме конечного числа) $\{a_i\}$ равны нулю. В кольце может и не быть единицы, но даже тогда мы определяем $a_0x^0:=a_0$ для удобства нотации.
\end{defn}

\begin{defn}
    $a$ \textit{коммутриует} с $b$, если $ab=ba$.
\end{defn}

\begin{defn}
    \hypertarget{n28}{\textcolor{red}{\textit{Кольцо многочленов}}} $R[x]$ - кольцо $R$ вместе с некоторыми $x\notin R$, для которых выполняются следующие свойства:\ 

    \begin{itemize}
        \item $\forall a\in R:ax=xa$;
        \item $\sum a_ix^i+\sum b_ix^i=\sum(a_i+b_i)x^i$;
        \item $-\sum a_ix^i=\sum-a_ix^i$;
        \item нуль есть $\sum 0x^i$;
        \item умножение по формуле свёртки: если 
        \[
            \biggl(\sum_i a_i x^i\biggr)\biggl(\sum_j b_j x^j\biggr)=\sum_k c_k x^k, 
        \]
        то
        \[
            c_k=a_kb_0+a_{k-1}b_1+\dots+a_0b_k=\sum_{i+j=k}a_ib_j.
        \]
    \end{itemize}
\end{defn}

\begin{defn}
    \hypertarget{n29}{\textcolor{red}{\textit{Степень многочлена}}} $\deg \sum a_i x^i$ - наибольшее $i$ такое, что $a_i\neq 0$. Если таких $i$ нет (многочлен нулевой), то его степень определять не будем.
\end{defn}

\begin{cons}
    Из определения степени сразу следует несколько свойств: \

    \begin{itemize}
        \item $\deg(f+g)\leq\deg f+\deg g$;
        \item $\deg(fg)\leq\deg f+\deg g$ для ненулевых $f, g$;
        \item $\deg(fg)=\deg f+\deg g$ для ненулевых $f, g$, если мы находимся в области целостности.
    \end{itemize}
\end{cons}

Последнее получается постольку поскольку старшие коэффициенты просто перемножеются, поэтому можно сформулировать такую лемму:\\

\begin{lemma}
    Если $R$ - область, то и $R[x]$ - область.
\end{lemma}\

А сейчас будем учиться делить многочлены с остатком, тем самым, покажем, что полученное кольцо евклидово (не всегда, конечно).\\

\begin{lemma}
    Пусть $R$ - кольцо, $f=a_nx^n+\dots \in R[x]$, $g=b_mx^m+\dots\in R[x]$, $\forall i: b_m^{n-m+1}\vert a_i$. Тогда существуют многочлены $q,r\in R[x]$ такие, что $f=gq+r$ и $r=0\vee \deg r<\deg g$.
\end{lemma}

\begin{proof}
    Докажем индукцией по $n$. База: если  $n<m$, то положим $q:=0$ и $r:=f$.\ 

    Пусть теперь $n\geq m$. По условию делимости $b_m^{n-m-1}c:=a_n$ для некоторог $c$. Посмотрим на $f_1:=f-tg$, где $t:=cb_m^{n-m}x^{n-m}$ - страший член неполного частного. $\deg tg = n$, потому старший член сократился при делении. Предположение индукции верно для пары $f_1,g$, так как единственный аспект под вопросом - делимость коэффициентов, но он тоже верен, что видно из определения $f_1$. Тогда применим индукцию: $f_1=q_1g+r$. Подставим $f=(t+q_1)g+r$. $r$ найден.
\end{proof}

\begin{cons}
    Пусть $F$ - поле. Тогда $F[x]$ - евклидово кольцо.
\end{cons}

\begin{proof}
    По доказанному выше, $\deg: F[x]\backslash 0\rightarrow \mathbb{N}_0$ - евклидова норма, потому что старший коэффициент ненулевого многочлена всегда обратим.
\end{proof}

\begin{remark}
    А вот для евклидова $R$, $R[x]$ не обязательно будет евклидовым кольцом.
\end{remark}

\resetall

\subsection{Лемма Гаусса}

\subsection{Факториальность кольца многочленов}

\subsection{Теорема Безу. Производная многочлена и кратные корни}

\subsection{Интерполяция Лагранжа}

\subsection{Интерполяция Эрмита}

\subsection{Поле разложение многочлена}

\subsection{Комплексные числа. Решение квадратных уравнений в $\mathbb{С}$}

\subsection{Основная теорема алгебры}

\subsection{Разложение рациональной функции в простейшие дроби над $\mathbb{C}$ и над $\mathbb{R}$}

\subsection{Определение векторного пространства. Линейная зависимость. Существование базиса}

\subsection{Размерность векторного пространства}

\subsection{Линейные отображения векторных пространств. Подпространство, фактор-пространство. Ранг линейного отображения}

\subsection{Матрица линейного отображения. Композиция линейных отображений и произведение матриц. Кольцо матриц}

\subsection{Элементарные преобразования. Метод Гаусса. Системы линейных уравнений}

\subsection{Теорема Кронекера-Капелли}

\subsection{Определение группы. Циклическая группа. Порядок элемента}

\subsection{Группа перестановок. Циклы, транспозиции. Знак перестановки}

\subsection{Действие группы на множестве. Орбиты. Классы сопряженности}

\subsection{Группа обратимых элементов кольца. Вычисление обратимых элементов $\mathbb{Z}/_{n\mathbb{Z}}$. Функция Эйлера}

\subsection{Гомоморфизмы и изоморфизмы групп. Смежные классы, теорема Лагранжа. Теорема Эйлера}

\subsection{Многочлены деления круга}

\subsection{Конечные поля (существование, единственность, цикличность мультипликативной группы)}

\subsection{Фактор-группа, теорема о гомоморфизме}

\subsection{Определитель матрицы. Инвариантность при элементарных преобразованиях, разложение по строчке и столбцу}

\subsection{Присоединенная матрица. Формула Крамера. Определитель транспонированной матрицы}

\subsection{Вычисление определителя методом Гаусса}

\subsection{Принцип продолжения алгебраических тождеств. Определитель произведения матриц}



\newpage

\hypertarget{t2}{И в заключение...}



\section{Пофамильный указатель всех мразей}

\begin{multicols}{2}
    [
    Быстрый список для особо заебавшегося поиска.
    ]

    \hyperlink{n4}{ассоциированность}\

    \hyperlink{n19}{гомоморфизм}\
    
    \hyperlink{n3}{делитель нуля}\

    \hyperlink{n5}{евклидово кольцо}\

    \hyperlink{n9}{идеал}\

    \hyperlink{22}{изоморфизм}\
    
    \hyperlink{n1}{кольцо, а также его вариации}\

    \hyperlink{n14}{кольцо вычетов}\

    \hyperlink{n28}{кольцо многочленов}\

    \hyperlink{n15}{КТО}\

    \hyperlink{n27}{многочлен}\

    \hyperlink{n7}{неприводимые}\

    \hyperlink{n6}{НОД}\

    \hyperlink{n21}{образ}\

    \hyperlink{n10}{ОГИ}\

    \hyperlink{n13}{ОТА}\

    \hyperlink{n2}{область целостности}\

    \hyperlink{n16}{поле}\

    \hyperlink{n18}{поле частных}\

    \hyperlink{n8}{простые}\ 

    \hyperlink{n23}{сравнимость}\

    \hyperlink{n29}{степень многочлена}\

    \hyperlink{n25}{теорема о гомоморфизме}\

    \hyperlink{n26}{универсальное св-во фактор-кольца}\

    \hyperlink{n11}{УОВЦГИ}\

    \hyperlink{n12}{факториальность}\

    \hyperlink{n24}{фактор-кольцо}\

    \hyperlink{n20}{ядро}\
    

\end{multicols}



\end{document}
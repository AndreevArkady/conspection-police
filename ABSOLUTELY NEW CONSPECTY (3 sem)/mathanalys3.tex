\documentclass[a4paper]{article}

%plastikovye pakety

\usepackage[12pt]{extsizes}
\usepackage[utf8]{inputenc}
\usepackage[unicode, pdftex]{hyperref}
\usepackage{cmap}
\usepackage{mathtext}
\usepackage{multicol}
\setlength{\columnsep}{1cm}
\usepackage[T2A]{fontenc}
\usepackage[english,russian]{babel}
\usepackage{amsmath,amsfonts,amssymb,amsthm,mathtools}
\usepackage{icomma}
\usepackage{euscript}
\usepackage{mathrsfs}
\usepackage[dvipsnames]{xcolor}
\usepackage[left=2cm,right=2cm,
    top=2cm,bottom=2cm,bindingoffset=0cm]{geometry}
\usepackage[normalem]{ulem}
\usepackage{graphicx}
\usepackage{makeidx}
\makeindex
\graphicspath{{pictures/}}
\DeclareGraphicsExtensions{.pdf,.png,.jpg}
%\usepackage[usenames]{color}
\hypersetup{
     colorlinks=true,
     linkcolor=coralpink,
     filecolor=coralpink,
     citecolor=black,      
     urlcolor=coralpink,
     }
\usepackage{fancyhdr}
\pagestyle{fancy} 
\fancyhead{} 
\fancyhead[LE,RO]{\thepage} 
\fancyhead[CO]{\hyperlink{dex}{к списку объектов}}
\fancyhead[LO]{\hyperlink{sod}{к содержанию}} 
\fancyfoot{}
\newtheoremstyle{indented}{0 pt}{0 pt}{\itshape}{}{\bfseries}{. }{0 em}{ }

\renewcommand\thesection{}
\renewcommand\thesubsection{}

%\geometry{verbose,a4paper,tmargin=2cm,bmargin=2cm,lmargin=2.5cm,rmargin=1.5cm}

\title{Конспект лекций по матанализу}
\author{Горбунов Леонид \\ 
 при участии и редакторстве @keba4ok \\ 
    на основе лекций Любарского Ю. И.}
\date{13 сентября 2021г.}


%envirnoments
    \theoremstyle{indented}
    \newtheorem{theorem}{Теорема}
    \newtheorem{lemma}{Лемма}
    \newtheorem{alg}{Алгоритм}

    \theoremstyle{definition} 
    \newtheorem{defn}{Определение}
    \newtheorem{exl}{Пример(ы)}
    \newtheorem{prob}{Задача}

    \theoremstyle{remark} 
    \newtheorem{remark}{Примечание}
    \newtheorem{cons}{Следствие}
    \newtheorem{exer}{Упражнение}
    \newtheorem{stat}{Утверждение}
%esli ne hochetsa numeracii - nuzhno prisunut' zvezdochku-pezsochku

\definecolor{coralpink}{rgb}{0.19, 0.55, 0.91}

%declarations
        %arrows_shorten
            \DeclareMathOperator{\la}{\leftarrow}
            \DeclareMathOperator{\ra}{\rightarrow}
            \DeclareMathOperator{\lra}{\leftrightarrow}
            \DeclareMathOperator{\llra}{\longleftrightarrow}
            \DeclareMathOperator{\La}{\Leftarrow}
            \DeclareMathOperator{\Ra}{\Rightarrow}
            \DeclareMathOperator{\Lra}{\Leftrightarrow}
            \DeclareMathOperator{\Llra}{\Longleftrightarrow}

        %letters_different
            \DeclareMathOperator{\CC}{\mathbb{C}}
            \DeclareMathOperator{\QQ}{\mathbb{Q}}
            \DeclareMathOperator{\ZZ}{\mathbb{Z}}
            \DeclareMathOperator{\RR}{\mathbb{R}}
            \DeclareMathOperator{\NN}{\mathbb{N}}
            \DeclareMathOperator{\HH}{\mathbb{H}}
            \DeclareMathOperator{\LL}{\mathscr{L}}
            \DeclareMathOperator{\KK}{\mathscr{K}}
            \DeclareMathOperator{\GA}{\mathfrak{A}}
            \DeclareMathOperator{\GB}{\mathfrak{B}}
            \DeclareMathOperator{\GC}{\mathfrak{C}}
            \DeclareMathOperator{\GD}{\mathfrak{D}}
            \DeclareMathOperator{\GN}{\mathfrak{N}}
            \DeclareMathOperator{\Rho}{\mathcal{P}}
            \DeclareMathOperator{\FF}{\mathcal{F}}

        %common_shit
            \DeclareMathOperator{\Ker}{Ker}
            \DeclareMathOperator{\Frac}{Frac}
            \DeclareMathOperator{\Imf}{Im}
            \DeclareMathOperator{\cont}{cont}
            \DeclareMathOperator{\id}{id}
            \DeclareMathOperator{\ev}{ev}
            \DeclareMathOperator{\lcm}{lcm}
            \DeclareMathOperator{\chard}{char}
            \DeclareMathOperator{\codim}{codim}
            \DeclareMathOperator{\rank}{rank}
            \DeclareMathOperator{\ord}{ord}
            \DeclareMathOperator{\End}{End}
            \DeclareMathOperator{\Ann}{Ann}
            \DeclareMathOperator{\Real}{Re}
            \DeclareMathOperator{\Res}{Res}
            \DeclareMathOperator{\Rad}{Rad}
            \DeclareMathOperator{\disc}{disc}
            \DeclareMathOperator{\rk}{rk}
            \DeclareMathOperator{\const}{const}
            \DeclareMathOperator{\grad}{grad}
            \DeclareMathOperator{\Aff}{Aff}
            \DeclareMathOperator{\Lin}{Lin}
            \DeclareMathOperator{\Prf}{Pr}
            \DeclareMathOperator{\Iso}{Iso}
            \DeclareMathOperator{\diam}{diam}

        %specific_shit
            \DeclareMathOperator{\Tors}{Tors}
            \DeclareMathOperator{\form}{Form}
            \DeclareMathOperator{\Pred}{Pred}
            \DeclareMathOperator{\Func}{Func}
            \DeclareMathOperator{\Const}{Const}
            \DeclareMathOperator{\arity}{arity}
            \DeclareMathOperator{\Aut}{Aut}
            \DeclareMathOperator{\Var}{Var}
            \DeclareMathOperator{\Term}{Term}
            \DeclareMathOperator{\sub}{sub}
            \DeclareMathOperator{\Sub}{Sub}
            \DeclareMathOperator{\Atom}{Atom}
            \DeclareMathOperator{\FV}{FV}
            \DeclareMathOperator{\Sent}{Sent}
            \DeclareMathOperator{\Th}{Th}
            \DeclareMathOperator{\supp}{supp}
            \DeclareMathOperator{\Eq}{Eq}
            \DeclareMathOperator{\Prop}{Prop}


%env_shortens_from_hirsh            
    \newcommand{\bex}{\begin{example}\rm}
    \newcommand{\eex}{\end{example}}
    \newcommand{\ba}{\begin{algorithm}\rm}
    \newcommand{\ea}{\end{algorithm}}
    \newcommand{\bea}{\begin{eqnarray*}}
    \newcommand{\eea}{\end{eqnarray*}}
    \newcommand{\be}{\begin{eqnarray}}
    \newcommand{\ee}{\end{eqnarray}}
    \newcommand{\abs}[1]{\lvert#1\rvert}
        \newcommand{\bp}{\begin{prob}}
        \newcommand{\ep}{\end{prob}}

    
\begin{document}
%ya_ebanutyi
\newcommand{\resetexlcounters}{%
  \setcounter{exl}{0}%
} 
\newcommand{\resetremarkcounters}{%
  \setcounter{remark}{0}%
} 
\newcommand{\reseconscounters}{%
  \setcounter{cons}{0}%
} 
\newcommand{\resetall}{%
    \resetexlcounters
    \resetremarkcounters
    \reseconscounters%
}

\newcommand{\cursed}[1]{\textit{\textcolor{coralpink}{#1}}}
\newcommand{\de}[3][2]{\index{#2}{\textbf{\textcolor{coralpink}{#3}}}}
\newcommand{\re}[3][2]{\hypertarget{#2}{\textbf{\textcolor{coralpink}{#3}}}}
\newcommand{\se}[3][2]{\index{#2}{\textit{\textcolor{coralpink}{#3}}}}

\maketitle 

\newpage

\hypertarget{sod}
\tableofcontents

\newpage


\section{Теория меры}

\subsection{Алгебраические структуры подмножеств}

Пусть нам дано множество $\mathcal{X}$ произвольной природы и система его подмножеств $\GA$.

\begin{defn}
    $\GA$ - \se{Полукольцо множеств}{полукольцо множеств}, если для любых $A$, $B$ $\in \GA$ их пересечение $A \cap B$ тоже лежит в $\GA$, а их разность $A \backslash B$ представляется в виде конечного объединения попарно дизъюнктных множеств из $\GA$.
\end{defn}

\begin{remark}
    Легко понять, что любое полукольцо содержит пустое множество.
\end{remark}

\begin{defn}
    $\GA$ - \se{Кольцо множеств}{кольцо множеств}, если для любых $A$, $B$ $\in \GA$ их пересечение $A \cap B$, объединение $A \cup B$ и разность $A \backslash B$ лежат в $\GA$
\end{defn}

\begin{remark}
    Легко понять, что тогда и $A \triangle B$ лежит в $\GA$. Тогда если на элементах кольца множеств определить операции сложения $+:= \triangle$ и умножения $\times := \cap$, то оно превратится в алгебраическое кольцо. 
\end{remark}

\begin{defn}
    $\GA$ - \se{Алгебра множеств}{алгебра множеств}, если оно кольцо, и для любого $A \in \GA$ множество $X \backslash A$ тоже лежит в $\GA$
\end{defn}

\begin{stat}
    Пусть $\GA \subseteq \Rho(X)$ и $\GB \subseteq \Rho(Y)$ - полукольца. Тогда $\GA \times \GB \subseteq \Rho(X \times Y)$ - тоже полукольцо.
\end{stat}

\begin{stat}
    Пусть множества $A$, $B_1$, ... $B_n$ принадлежат какому-то полукольцу. Тогда $A \backslash (B_1 \cup ... \cup B_n)$ представляется в виде объединения конечного числа элементов этого полукольца.
\end{stat}

\begin{proof}
    $A \backslash (B_1 \cup ... \cup B_n)=(A \backslash B_1) \cap ... \cap (A \backslash B_n)=(\bigsqcup_{i=1}^{k_1}C_{1, i}) \cap ... \cap (\bigsqcup_{i=1}^{k_n}C_{n, i}) = \bigsqcup_{i_1, ... i_n} (C_{1, i_1} \cap ... \cap C_{n, i_n})$. В последнем выражении все множества попарно дизъюнктны, так как если бы, например, $(C_{1, i_1} \cap ... \cap C_{n, i_n}) \cap C_{1, j_1} \cap ... \cap C_{n, j_n} \ni x$, то для каждого $k$ от $1$ до $n$ $x \in C_{k, i_k} \cap C_{k, j_k}$, что возможно только при $i_k=j_k$, но для всех $k$ это равенство быть верным не может.
\end{proof}

\begin{exl}
    $P(\RR)=\{[a, b) | a, b, \in \RR \cup \{\pm \infty\}\}$ - \se{Полукольцо ячеек}{полукольцо ячеек}
    \\
    $P(\RR^n)=\{[a_1, b_1) \times ... \times [a_n, b_n) | a_i, b_i, \in \RR \cup \{\pm \infty\}\}$ - тоже полукольцо ячеек, только многомерных
\end{exl}

\subsection{Вводим меру}

Пусть $\mathfrak{X}$ - множество произвольной природы, $\GA \subseteq \Rho(\mathfrak{X})$. 

\begin{defn}
    Функция $\mu: \GA \ra \RR_{\geq 0} \cup \{+\infty\}$ называется \se{Мера}{мерой}, если для любых попарно дизъюнктных множеств $A_1$, ... $A_k$ $\in \GA$ и таких, что $\bigsqcup_{i=1}^k A_i \in \GA$, верно равенство $\mu(\bigsqcup_{i=1}^k A_i)=\sum_{i=1}^k \mu(A_i)$
\end{defn}

\begin{remark}
    Данное свойство называется \textit{аддитивностью}
\end{remark}

\begin{exl}
    \
    \begin{itemize}
        \item $\mathfrak{X}$ - дискретное пространство, и для любого $x \in \mathfrak{X}$ $\mu({x})=1$. Тогда $\mu(A)=\sum_{x \in A} 1$
        \item $\mathfrak{X}$ - дискретное пространство, и для любого $x \in \mathfrak{X}$ $\mu({x})=p_x$, причём $\sum_{x \in \mathfrak{X}} p_x=1$. Тогда мы получаем в точности вероятностное пространство.
        \item $\mathfrak{X}=\RR$, $\GA$ - полукольцо конечных ячеек. Тогда $\mu([a, b))=b-a$ - мера.
        \item То же, что и в предыдущем примере, только теперь $\mu([a, b))=f(b)-f(a)$, где $f$ - монотонно возрастающая функция.
    \end{itemize}
\end{exl}

\begin{stat}
    Мера, определённая на полукольце, монотонна: если $A$, $B$ $\in \GA$, и $B \subseteq A$, то $\mu(B) \leq \mu(A)$.
\end{stat}

\begin{proof}
    $\mu(A)=\mu(B)+\mu(A \backslash B) = \mu(B)+\mu(\bigsqcup_{i=1}^n C_i)=\mu(B)+\sum_{i=1}^n \mu(C_i) \geq \mu(B)$
\end{proof}

\subsection{Простые функции}

\begin{defn}
    Пусть $\GA$ - полукольцо, и $A \in \GA$. Определим \se{Функция-индикатор}{функцию-индикатор} (или \se{Характеристическая функция}{характеристическую функцию}): 

\begin{equation*}
    \chi_A(x)=
  \begin{cases}
     1 , \: \text{если} \: x \in A, \\
     0 , \: \text{если} \: x \notin A\\
  \end{cases}
\end{equation*}

\end{defn}

\begin{defn}
    \se{Простая функция}{Простая функция} - это функция вида $f(x)=\sum_{i=1}^n a_i \chi_{A_i}(x)$, где $A_i \in \GA$ и $a_i \in \RR$
\end{defn}

\begin{remark}
    Сумма и произведение простых функций - простые функции.
\end{remark}

\subsection{Элементарный интеграл}

Пусть мы имеем $\GA$ - полукольцо, $\mu$ - меру и $f$ - простую функцию (всё пока что конечно). Можем тогда ввести следующее понятие:

\begin{defn}
    \se{Интеграл!элементарный}{Элементарным интегралом} называется 

    \[
        \int f(x)dx = \sum a_i \mu(A_i)
    \]
\end{defn}

\begin{stat}
    Определение корректно.
\end{stat}

\begin{remark}
    Я не понял, что тут рассказывает Юрий Ильич, поэтому доказательство найдено в других источниках. Суть просто в попарном подразбиении и перегуппировке.
\end{remark}

\begin{proof}
    Пусть $f = \sum \alpha_i \cdot \chi(a_i) = \sum \beta_j \cdot \chi(b_j)$, рассмотрим тогда $c_{ij} = a_i \cap b_j$.  

    \[
        \sum \mu(a_j) \cdot \alpha_j = \sum \mu(c_{ij}) \cdot \alpha_i = \sum \mu(c_{ij}) \cdot \beta_j = \sum \mu(b_j) \beta_j
    \]
\end{proof}

\begin{stat}[Техническое замечение]
    \[
        \int \chi_A = \mu(A).
    \]
\end{stat}

\begin{stat}
    Рассмотрим свойства интеграла: 

    \begin{itemize}
        \item Линейность. Если у нас есть две простые функции: $f$ и $g$, а также два числа: $\alpha, \beta \in \RR$, тогда 
        \[
            \int \alpha f + \beta g = \alpha \int f + \beta \int g.
        \]
        \item Монотонность. Пусть $f$ и $g$ - простые функции, а также $f \leq g$. Тогда 
        \[
            \int f \leq \int g.
        \]
    \end{itemize}
\end{stat}

\begin{remark}
    Для доказательства практически всего нужно просто рассмотреть дизъюнктное подразбиение данных функций.
\end{remark}

\subsection{Включаем бесконечность}

Пусть у нас, по прежнему, имеется кольцо, и простая функция $f$. Выделим тогда у неё положительную и отрицательную часть ($f^{+}$ и $f^{-}$). Такие, что положительная часть во всех положительных значениях остаётся таковой, а при отрицательных - обнуляется. Почти аналогично с отрицательной, только мы рассмотриваем модуль того, что останется. Таким образом,

\[
    f = f^{+} - f^{-}. 
\]

Определим тогда $I_{+}(f) = \int f_{+}$, и аналогично $I_{-}$. Мы хотим определить интеграл от функции, как $I_{+}(f) - I_{-}(f)$. Но нам мешает то, что обе эти функции могут быть бесконечными. Так что в случае, когда оба интеграла равны бесконечности, у нас ничего не получится, и этот случай мы попросу запрещаем. И рассмотриваем мы теперь только функции, который могут быть бесконечны максимум в одну сторону.

\begin{remark}
    Монотонность и линейность останутся при данном определении (последнее, конечно, опять таки при конечности хотя бы одного из интегралов). 
\end{remark}

\subsection{Произведение мер}
\se{Произведение мер}{}
Пусть $\GA$, $\GB$ - полукольца с мерами $\mu$ и $\nu$ соответственно. Определим функцию $\lambda: \GA \times \GB: \RR_{\geq 0} \cup \{+\infty\}$ по правилу $\lambda(A \times B)=\mu(A)\nu(B)$
\begin{stat}
$\lambda$ - мера на полукольце $\GA \times \GB$, т.е. для любых попарно дизъюнктных $C_1$, ... $C_n$, $C_i=A_i \times B_i$ и таких, что $\bigsqcup_{i=1}^n C_i =C = A \times B \in \GA \times \GB$, верно равенство $\lambda(\bigsqcup_{i=1}^n C_i)=\sum_{i=1}^n \lambda(C_i)$
\end{stat}
\begin{proof}
По определению мер $\lambda(\bigsqcup_{i=1}^n C_i)=\mu(A)\nu(B)$, $\sum_{i=1}^n \lambda(C_i)=\sum_{i=1}^n \mu(A_i)\nu(B_i)$, поэтому мы будем доказывать равенство $\mu(A)\nu(B)=\sum_{i=1}^n \mu(A_i)\nu(B_i)$. Так как все $C_i$ попарно дизъюнктны, верно равенство $\chi_{C}(x, y) \sum_{i=1}^n \chi_{C_i}(x, y)$. Зафиксируем $x$, тогда функция-индикатор $\chi_{C_i}(x, y)$ на $\GA \times \GB$ превращается в функцию индикатор $\chi_{A_i}(x)\chi_{B_i}(y)$ на $\GB$. Проинтегрируем равенство по $y$, получим: $\chi_A(x)\nu(B)=\sum_{i=1}^n \chi_{A_i} \nu(B_i)$. Интегрируя теперь по $x$, получаем $\mu(A)\nu(B)=\sum_{i=1}^n \mu(A_i)\nu(B_i)$, что и требовалось.
\end{proof}

\subsection{Счётная аддитивность (она же $\sigma$-аддитивность)}
\begin{defn}
Пусть даны $\mathcal{D} \subseteq \Rho(X)$ - набор подмножеств множества $X$, и функция $\mu: \mathcal{D} \ra \RR_{\geq 0} \cup \{+\infty\}$. Эта функция называется \se{Счётно-аддитивная функция}{счётно-аддитивной} (или \se{$\sigma$-аддитивная функция}{$\sigma$-аддитивной}), если для любого не более чем счётного набора попарно дизъюнктных множеств $\{B_i\}$ таких, что их объединение $B=\bigsqcup B_i$ лежит в $\mathcal{D}$, верно равенство $\mu(B)=\sum \mu(B_i)$
\end{defn}
\begin{exl}
\begin{itemize}
    \item $\mathcal{D}=\Rho{X}$, и для любого $B \in \mathcal{D} \mu(B)=|B|$ - \textit{считающая функция}
    \item Вероятностное пространство
    \item $X=\RR$, $\mathcal{D}=P(\mathcal{\RR})$, $\mu([a, b))=b-a$
    \item Модификация предыдущего примера: $\mu([a, b))=f(b)-f(a)$, где $f$ - монотонно возрастающая непрерывная функция
    \item $X=\RR$, $\mathcal{D}=\{<a, b> | a, b \in \RR \cup \{\pm \infty\}\}$, $f$ - просто монотонно возрастающая функция. Тогда мера $\mu(<a, b>)=f(b)-f(a)$ не будет счётно-аддитивной. Но если мы определим меру так:
    \begin{itemize}
        \item $\mu([a, b))=\lim_{x \ra b_-} f(x)-\lim_{y \ra a_-} f(y)$
        \item $\mu([a, b])=\lim_{x \ra b_+} f(x)-\lim_{y \ra a_-} f(y)$
        \item $\mu((a, b])=\lim_{x \ra b_+} f(x)-\lim_{y \ra a_+} f(y)$
        \item $\mu((a, b))=\lim_{x \ra b_-} f(x)-\lim_{y \ra a_+} f(y)$
    \end{itemize}
    то она уже будем счётно-аддитивной.
\end{itemize}
\end{exl}
\begin{stat}
Не существует "универсальной меры", т.е. функции $\mu: \Rho(\RR) \ra \RR_{\geq 0} \cup \{+\infty\}$, обладающей следующими свойствами:
\begin{itemize}
    \item $\mu(\emptyset)=0$
    \item $\mu$ - счётноаддитивна
    \item $\mu([0, 1])=1$
    \item Для любых $A \subseteq \RR$ и $x \in \RR$ верно равенство $\mu(A+x)=\mu(A)$
\end{itemize}
\end{stat}
\begin{proof}
Предположим противное: такая функция существует. Определим на $\RR$ бинарное отношение $a \sim b \iff a-b \in \QQ$. Легко видеть, что это отношение эквивалентности. Воспользуемся аксиомой выбора и выберем по одному представителю из каждого класса так, чтобы они все лежали на отрезке $[0, 1]$. Образуем из них множество $A$. С одной стороны, $\mu(A)=\mu([0, 1])-\mu([0, 1] \backslash A) \geq 1 < \infty$. Рассмотрим множества $A_q=\{A+q\}$ для всех $q \in [0, 1] \cap \QQ$. Они попарно не пересекаются, их мера равна мере $A$, а их объединение лежит в отрезке $[-1, 2]$. Тогда $|0, 1] \cap \QQ| \cdot \mu(A) = \sum_{q \in [0, 1] \cap \QQ} \mu(A_q) = \mu(\bigsqcup_{q \in [0, 1] \cap \QQ}A_q) \leq \mu([-1, 2]) < \infty$, откуда $\mu(A)=0$. Но $\bigsqcup_{\lambda \in \QQ} A_{\lambda}=\RR$ $\implies \infty = \mu(\RR)=\sum_{\lambda \in \QQ} \mu(A_{\lambda})=\sum_{\lambda \in \QQ} 0 = 0$, противоречие.
\end{proof}
\begin{defn}
Мера $\mu$, определённая на полукольце (кольце, алгебре и т.д.) $\GA \subseteq \Rho(X)$, называется \se{Регулярная мера}{регулярной}, если для любого $A \in \GA$ :
\begin{itemize}
    \item $\mu(A)=\inf_{G \in \GA, A \subseteq G, G \text{ - открытое}} \mu(G)$
    \item $\mu(A)=\sup_{K \in \GA, K \subseteq A, K \text{ - компакт}} \mu(K)$
\end{itemize}
\end{defn}
\begin{theorem}
Регулярная мера $\mu$, определённая на кольце, счётноаддитивна.
\end{theorem}
\begin{proof}
Пусть $\{A_i\}$ - попарно дизъюнктные элементы кольца, и $A=\bigsqcup A_i \in \GA$. Хотим доказать, что $\mu(A)=\sum \mu(A_i)$.
\\
В одну сторону это практически очевидно: для любого натурального $n$ $A_1 \cup ... \cup A_n \subseteq A$ $\implies$ $\sum_{i=1}^n \mu(A_i) = \mu(A_1 \cup ... \cup A_n ) \leq \mu(A)$. Переходя к пределу по $n$, получаем неравенство в одну сторону. 
\\
Теперь докажем, что для любого $\epsilon > 0$ верно неравенство $\sum \mu(A_i) \geq \mu(A)-2 \epsilon$, откуда и будет следовать неравенство во вторую сторону. Для этого выберем компакт $K \subseteq A$ такой, что $\mu(K) \geq \mu(A)-\epsilon$, а для каждого $A_i$ - такое $G_i$, что $\mu(G_i) \leq \mu(A_i)+\frac{\epsilon}{2^i}$. Так как $\bigsqcup A_i = A \supset K$, то и $\bigcup G_i \supset K$, а тогда можно выбрать конечное подпокрытие $G_{i_1}$, ... $G_{i_s}$. В итоге $\mu(K) \leq \sum_{j=1}^s \mu(G_{i_j}) \leq \sum_{j=1}^s \mu(A_i)+ \frac{\epsilon}{2^{i_j}} < \sum_{j=1}^{\infty} \mu(A_i) + \epsilon$ $\implies $ $\sum \mu(A_i) \geq \mu(K)-\epsilon \geq \mu(A)-2\epsilon$, что и требовалось.
\end{proof}
\subsection{Счётно-аддитивные структуры}
\begin{defn}
Непустое $\GA \subseteq \Rho(X)$ называется \se{$\sigma$-алгебра}{$\sigma$-алгеброй}, если для любого не более чем счётного набора множеств $\{A_i\}$ их объединение и пересечение и $X \backslash A_i$ также лежат в $\GA$
\end{defn}
\begin{remark}
$\emptyset = A \cap (X \backslash A)$, $X= A \cup (X \backslash A)$, $A \backslash B = A \cap (X \backslash B)$ также лежат в $\GA$.
\end{remark}
\begin{remark}
Если $\{A_i\}_{i \in I}$ - произвольный набор $\sigma$-алгебр над каким-то множеством, то $\bigcap_{i \in I} A_i$ - тоже $\sigma$-алгебра.
\end{remark}
\begin{defn}
Пусть $\mathcal{D} \subseteq \Rho(X)$. \se{Порождённая $\sigma$-алгебра}{$\sigma$-алгебра, порождённая $\mathcal{D}$} - это наименьшая $\sigma$-алгебра, содержащая $\mathcal{D}$. мы будем обозначать её $\overline{\mathcal{D}}$
\end{defn}
\begin{stat}
Для любого $\mathcal{D} \subseteq \Rho(X)$ порождённая $sigma$-алгебра существует и единственна.
\end{stat}
\begin{proof}
Хотя бы одна $\sigma$-алгебра, содержащая $\mathcal{D}$, существует: это просто $\Rho(X)$. Но тогда если $\{A_i\}_{i \in I}$ - все такие $\sigma$-алгебры, то $\bigcap_{i \in I} A_i$ - наименьшая.
\end{proof}
\begin{stat}
Любое открытое и замкнутое множество на прямой содержится в $\overline{P(\RR)}$
\end{stat}
\begin{proof}
Заметим, что интервал $(a, b)$ представляется в виде счётного объединения ячеек $\bigcup_{n \in \NN} [a+\frac{1}{n}, b)$, а любое открытое подмножество прямой является объединением не более чем счётного объединения попарно непересекающихся открытых интервалов и лучей. Если же какое-то $A$ замкнуто, то $\RR \backslash A$ открыто и представляется в виде $\bigcup P_i$, $P_i \in P(\RR)$. Тогда $A=X \backslash (\bigcup P_i) = \bigcap (X \backslash P_i)$ тоже представимо в виде не более, чем счётного объединения элементов из $P(\RR)$, а потому лежит в $\overline{P(\RR)}$.
\end{proof}
\begin{stat}
Пусть $\GA \subseteq \Rho(X)$ - алгебра, и известно, что для любых $\{E_i\}_{i=1}^{\infty} \in \GA$, $\bigcap_{i=1}^{\infty} E_i$ также принадлежит $\GA$. Тогда $A$ - $\sigma$-алгебра. 
\end{stat}
\begin{proof}
Надо проверить, что если $\{F_i\}_{i=1}^{\infty} \in \GA$, то $\bigcup_{i=1}^{\infty} F_i$ также принадлежит $\GA$. Но $\bigcup_{i=1}^{\infty} F_i = X \backslash (\bigcap (X \backslash F_i))$, т.е. лежит в $\GA$.
\end{proof}
\begin{remark}
Можно доказать и в обратную сторону (т.е. из счётного объединения вывести счётное пересечение), причём дополнительно можно наложить условие попарной дизъюнктности рассматриваемых множеств - доказательство будет аналогичным (только во втором случае придётся ввести новую последовательность множеств $\{G_i\}$, определённую по индукции $G_1=E_1$, $G_k=E_k \backslash (E_1 \cup ... \cup E_{k-1})$)
\end{remark}
\subsection{Внешняя мера}
\begin{defn}
Пусть $\GA \subseteq \Rho(X)$ - полукольцо с (конечно-аддитивной) мерой $\mu$. Определим функцию $\mu^*: \Rho(X) \ra \RR_{\geq 0} \cup \{+\infty\}$ по правилу $\mu^*(A) = \inf(\sum \mu(A_i) | \{A_i\} \in \GA, \bigcup A_i \supset A)$ (т.е. инфимум по всем покрытиям множества $A$ элементами полукольца) и назовём её \se{Внешняя мера}{внешней мерой}.
\end{defn}
\begin{stat}
\begin{enumerate}
\
    \item $\mu^*(A) \leq \mu(A)$
    \item Монотонность: если $A \subseteq B$, то $\mu^*(A) \leq \mu^*(B)$
    \item \se{Счётная полуаддитивность}{Счётная полуаддитивность}: Если $\{A_i\} \in \GA$, и $\bigcup A_i \in \GA$ то $\mu^*(\bigcup A_i) \leq \sum \mu^*(A_i)$
    \item Если $\mu$ - счётно-аддитивна, то $\mu^*_{\upharpoonright \GA} = \mu$
\end{enumerate}
\end{stat}
\begin{proof}
\

\begin{enumerate}
    \item $A$ - это одно из покрытий самого себя.
    \item $B$ - это одно из покрытий множества $A$.
    \item Обозначим $A=\bigcup A_i$. Если для какого-то $i$ $\mu^*(A_i)=\infty$, то неравенство очевидно, поэтому далее считаем, что все $\mu^*(A_i)< \infty$. Зафиксируем произвольное $\epsilon > 0$. Для каждого $A_n$ существует покрытие $\{B_{n_k}\}_{k \geq 1}$ элементами полукольца, для которого $\mu^*(A_i)> \sum_{k \geq 1} \mu(B_{n_k}) - \frac{\epsilon}{2^n}$. Тогда $\bigcup_{n, k} B_{n_k}$ - покрытие $A$, и $\mu^*(A) \leq \sum_{n, k}\mu(B_{n_k}) < \sum \mu(A_i) + \epsilon $. Значит,  $\mu^*(A) \leq \sum \mu^*(A_i)$, что и требовалось.
    \item Введём вспомогательную функицю $\overline{\mu}$, которая определяется так же, как и $\mu^*$, только теперь мы на каждое рассматриваемое покрытие дополнительно наложили ограничение попарной дизъюнктности составляющих его множеств. 
    \\
    Докажем для начала, что $\overline{\mu}=\mu*$. То, что $\overline{\mu}(A) \geq \mu^*(A)$, очевидно - во втором случае инфимум берётся по большему множеству. Зафиксируем теперь $\epsilon > 0$ и будем доказывать, что $\overline{\mu}(A) \leq \mu^*(A)+\epsilon$. Для этого рассмотрим покрытие $A$ такими множествами $\{A_i\} \in \GA$, что $\sum \mu(A_i) \leq \mu^*(A)+\epsilon$. Определим последовательность множеств $\{B_i\}$ по правилу $B_1:=A_1$ и $B_k = A_k \backslash (A_1 \cup ... A_{k-1})$ при $k>1$. Все $B_i$, во-первых, попарно дизъюнктны, а во-вторых, представляются в виде конечного объединения попарно дизъюнктных элементов полукольца (см. утверждение из раздела "алгебраические структуры подмножеств"). Для определённости, пусть $B_i= \bigsqcup_j B_{i, j}$. Тогда $\{C_{i, j}\}$ - покрытие множества $A$ попарно непересекающимися элементами полукольца, откуда мы заключаем, то $\overline{\mu}(A) \leq \sum_{i, j} \mu(C_{i, j}) = \sum \mu(A_i) \leq \mu^*(A)+\epsilon$. В последнем равенстве мы воспользовались счётной аддитивностью меры $
    \mu$ и тем, что $\bigsqcup_{i, j} C_{i, j} = \bigcup A_i \supset A$
    \\ 
    Вернёмся к исходному утверждению. Пусть $A \in \GA$. Так как $A$ - само себе дизъюнктное покрытие, то $\overline{\mu}(A) \leq \mu(A)$. С другой стороны, для любого $\epsilon > 0$ существует покрытие $A$ попарно дизъюнктными элементами полукольца $\{A_i\} \in \GA$, для которого $\sum \mu(A_i) \leq \overline{\mu}(A)+\epsilon$. Собирая два последних предложения вместе и пользуясбь счётной аддитивностью $\mu$, получаем: $\mu(A) = \sum \mu(A \cap A_i) \leq \sum \mu(A_i) \leq \overline{\mu}(A)+\epsilon$. Так как это выполнено для любого $\epsilon>0$, то $\mu(A) \leq \overline{\mu}(A)=\mu^*(A)$. Но всегда верно обратное неравенство $\mu(A) \geq \mu^*(A)$, откуда мы и получаем требуемое равенство мер.
\end{enumerate}
\end{proof}
\subsection{Теорема Лебега-Каратеодори}
\begin{defn}
Пусть $X$ - множество произвольной природы. Монотонную и счётно-полуаддитивную функцию $\gamma: \Rho(X) \ra \RR_{\geq 0} \cup \{\infty\}$, такую, что $\gamma(\emptyset)=0$, мы назовём \se{Предмера}{предмерой} на множестве $X$.
\end{defn}
\begin{defn}
Множество $E \subseteq X$ называется \se{$\gamma$-измеримое множество}{$\gamma$-измеримым}, если для любого $A \subseteq X$ верно равенство $\gamma(A) = \gamma(A \cap E) + \gamma(A \backslash E)$ или, что равносильно, $\gamma(A)=\gamma(A \cap E) + \gamma (A \cap E^{c})$
\end{defn}
\begin{remark}
Внешняя мера - это предмера
\end{remark}
\begin{theorem}
\se{Теорема Лебега-Каратеодори}{Теорема Лебега-Каратеодори}
\\
Пусть $\gamma$ - предмера на множестве $X$, и $\Sigma \subseteq \Rho(X)$ - набор всех $\gamma$- измеримых подмножеств. Тогда:
\begin{enumerate}
    \item $\Sigma$ - $\sigma$-алгебра
    \item $\gamma_{\upharpoonright \Sigma}$ - счётно-аддитивная мера на $\Sigma$.
    \item Пусть $\GA$ - полукольцо на $X$, и $\mu$ - (конечно) аддитивная мера на нём. Если мы определим $\gamma:=\mu^*$, то $\Sigma \supset \overline{\GA}$.
\end{enumerate}
\end{theorem}

\begin{proof}
\
\begin{itemize}

    \item Сначала докажем, что $\Sigma$ - это (обычная) алгебра.
    \\
    $\gamma(A)=\gamma(A)+\gamma(\emptyset) = \gamma(A \backslash \emptyset)+\gamma(A \cap \emptyset)$ $\implies$ $\emptyset \in \Sigma$. Аналогично, $X \in \Sigma$.
    \\
    Если $E \in \Sigma$, то $E^{c} \in \Sigma$ - следует из симметричного определения измеримой функции. 
    \\
    Так как $A \cup B = X \backslash ((X \backslash A) \cap (X \backslash B))$, то достаточно проверить только, что если $E_1$, $E_2$ $\in \Sigma$, то $E_1 \cap E_2 \in \Sigma$. Хотим: $\gamma(A)=\gamma(A \cap (E_1 \cap E_2))+\gamma(A \backslash (E_1 \cap E_2))$. Воспользуемся теперь определением $\gamma$-измеримого множества и подставим туда различные пары множеств:
    \\
    \begin{equation*}
\begin{cases}
  \gamma(A)=\gamma(A \cap E_1)+\gamma(A \backslash E_1),  & \mbox{ - подставили пару } (A, E_1) \\
  \gamma(A \cap E_1) = \gamma(A \cap E_1 \cap E_2)+ \gamma ((A \cap E_1) \backslash E_2)  & \mbox{ - подставили пару } (A \cap E_1, E_2) \\
  \gamma(A \backslash(E_1 \cap E_2)) = \gamma(A \backslash E_1)+\gamma((A \cap E_1) \backslash E_2) & \mbox{ - подставили пару } (A \backslash (E_1 \cap E_2), E_1)
\end{cases}
\end{equation*}
\

Выражая $\gamma(A \cap E_1)$ из первого уравнения во второе, получаем равенство $\gamma(A) = \gamma(A \cap E_1 \cap E_2) + \gamma (A \backslash E_1)+\gamma((A \cap E_1) \backslash E_2)$, но правая часть по третьему равенству равна в точности $\gamma(A \cap E_1 \cap E_2)+\gamma(A \backslash (E_1 \cap E_2))$. Мы доказали, что множество $E_1 \cap E_2$ тоже $\gamma$-измеримо.
\item Теперь покажем, что $\gamma_{\upharpoonright \Sigma}$ - аддитивна.
\\
Пусть $E_1, E_2 \in \Sigma$ - дизъюнктные множества. Тогда $\gamma(E_1 \cup E_2) = \gamma((E_1 \cup E_2) \backslash E_2)+\gamma((E_1 \cup E_2) \cap E_2) = \gamma(E_1)\cap \gamma(E_2)$, что и требовалось.
\item Следующий шаг - доказать, что $\Sigma$ - это $\sigma$-алгебра.
\\
Мы помним, что достаточно доказывать утверждение про объединение попарно дизъюнктных множеств: если $\{E_i\} \in \Sigma$ - попарно дизъюнктны, то $E=\bigsqcup E_i \in \Sigma$, т.е. что для любого $A \subseteq X$ верно равенство $\gamma(A) = \gamma(A \cap E)+\gamma(A \backslash E)$. Как и раньше, нам достаточно вместо равенства доказать неравенство в обе стороны. Неравенство $LHS \leq RHS$ верно в силу полуаддитивности $\gamma$. Будем доказывать неравенство в обратную сторону. Сразу отметим, что если $\gamma(A)=\infty$, то оно верно, поэтому далее мы считаем, что $\gamma(A)< \infty$. Для любого натурального $n$: $\gamma(A) = \gamma(A \cap \bigcup_{i=1}^n E_i)+\gamma(A \backslash \bigcup_{i=1}^n E_i) \geq \gamma(A \cap \bigcup_{i=1}^n E_i)+\gamma(A \backslash E)$. 
\\
Докажем, что для любого натурального $n$ верно соотношение $\gamma(A \cap \bigsqcup_{i=1}^n E_i) = \sum_{i=1}^n \gamma(A \cap E_i)$. Переход практически очевиден, поэтому сосредоточим наше внимание на базе: $\gamma(A \cap (E_1 \cup E_2)) = \gamma(A \cap E_1) + \gamma(A \cap E_2)$. Но это ни что иное, как определение измеримости для пары $(A \cap (E_1 \cup E_2), E_1)$. 
\\
Комбинируя результаты двух последних абзацев, получаем неравенство $\gamma(A) \geq \sum_{i=1}^n \gamma(A \cap E_i)+\gamma(A \backslash E)$. Так как $\gamma(A) < \infty$, мы можем перейти к пределу по $n$ и получить неравенство $\gamma(A) \geq \sum_{i=1}^{\infty} \gamma(A \cap E_i)+\gamma(A \backslash E) \geq \gamma(A \cap E) + \gamma(A \backslash E)$ (в последнем переходе мы воспользовались счётной полуаддитивностью $\gamma$).
\item $\gamma_{\upharpoonright \Sigma}$ - счётно-аддитивная функция.
\\
Пусть есть счётный набор $\{E_i\} \subseteq \Sigma$ попарно дизъюнктных множеств. Мы уже доказали, что $E = \bigsqcup E_i \in \Sigma$. Хотим доказать, что $\sum_{i=1}^{\infty} = \gamma(E)$. Неравенство $LHS \geq RHS$ выполняется в силу полуаддитивности, поэтому мы будем доказывать неравенство $LHS \leq RHS$.
\\
Для любого натурального $n$ верно соотношение $\gamma(E)=\gamma(E \cap (E_1 \cup ... \cup E_n))+\gamma(E \backslash (E_1 \cup ... \cup E_n)) \geq \gamma(E \cap (E_1 \cup ... \cup E_n)) = \sum_{i=1}^n \gamma(E_i)$. переходя к пределу по $n$, получаем требуемое неравенство.
\item Достаточно показать, что $\GA \subseteq \Sigma$. Пусть $E \in \GA$. Надо доказать, что для любого $A \subseteq X$ $\mu^*(A)=\mu^*(A \cap E)+\mu^*(A \backslash E)$. Опять-таки, в силу полуаддитивности $\mu^*$ достаточно доказать только неравенство $\mu^*(A) \geq \mu^*(A \cap E)+\mu^*(A \backslash E)$ и, как и в пункте 3, нетривиальным будет только случай $\mu^*(A) < \infty$. 
\\
Для любого $\epsilon > 0$ докажем, что $\mu^*(A)+\epsilon \geq \mu^*(A \cap E)+\mu^*(A \backslash E)$, из этого будет следовать требуемое. Можно выбрать $\{C_i\}_{i\geq 1}$ - такое покрытие $A$ попарно дизъюнктными элементами полукольца, что $\sum \mu(C_j) \leq \mu^*(A)+\epsilon$. Тогда $\{C_i \cap E\}_{i \geq 1} \subseteq \GA$ - покрытие $A \cap E$, откуда $\mu^*(A \cap E) \leq \sum_{i \geq 1} \mu(C_i \cap E)$. Также $C_i \backslash E=\bigsqcup_{j=1}^{n_i} D_{i, j}$ - конечное объединение попарно дизъюнктных элементов полукольца, а тогда $\{D_{i, j}\}$ - покрытие $A \backslash E$ $\implies$ $\mu^*(A \backslash E) \leq \sum_{i, j} \mu(D_{i, j}) = \sum_{i \geq 1} \mu(C_i \backslash E)$. Складывая два последних неравенства, получаем, что $\mu^*(A \cap E) + \mu^*(A \backslash E) \leq \sum_{i \geq 1} (\mu(C_i \cap E)+\mu(C_i \backslash E)) = \sum_{i \geq 1} \mu(C_i) \leq \mu^*(A)+\epsilon$. 
\end{itemize}
\end{proof}
\subsection{Борелевские множества и мера Лебега}
\begin{defn}
Пусть $P(\RR^n)$ - полукольцо ячеек с естественной мерой $\mu$ (которая, как мы помним, счётно-аддитивна). Множества, измеримые относительно внешней меры $\mu^*$, образуют $\sigma$-алгебру (будем обозначать её $\Sigma$) и называются \se{Множества, измеримые по Лебегу}{измеримыми по Лебегу}, а $\mu^*$ от них обозначается буквой $\lambda$ и называется \se{Мера Лебега}{мерой Лебега}. 
\end{defn}
\begin{defn}
Рассмотрим $\GB = \overline{P(\RR^n)}$ - $\sigma$-алгебра, натянутая на полукольцо ячеек $P(\RR^n)$. Она состоит из всевозможных счётных объединений и пересечений элементов $P(\RR^n)$ и называется \se{Борелевская $\sigma$-алгебра}{Борелевской $\sigma$-алгеброй}. Эта алгебра содержит, например, все открытые множества (так как любое открытое множество в $\RR^n$ можно представить в виде дизъюнктного объединения ячеек).
\end{defn}
\begin{remark}
Любое измеримое по Борелю множество также измеримо и по Лебегу (в силу п.3 теоремы Лебега-Каратеодори), но обратное неверно. 
\end{remark}
\begin{remark}
Мощность Борелевской алгебры - континуум, так как все её элементы получаются из изначального континуального набора $P(\RR^n)$ применением счётного числа пересечений и объединений.
\end{remark}
\begin{stat}
Пусть $\gamma$ - предмера на $X$. Если $E \subseteq X$, и $\gamma(E)=0$, то $E$ - $\gamma$-измеримо. Как следствие, любое подмножество $\gamma$-измеримого и имеющего предмеру ноль множества также измеримо.
\end{stat}
\begin{proof}
Пусть $A \subseteq X$ - произвольное подмножество. Пользуясь монотонностью и полуаддитивностью предмеры, напишем цепочку неравенств: $\gamma(A \backslash E) \leq \gamma(A) \leq \gamma(A \cap E)+ \gamma(A \backslash E) \leq \gamma(E)+\gamma(A \backslash E) = \gamma(A \backslash E)$. Значит, все неравенства обращаются в равенство, и $\gamma(A)= \gamma(A \cap E)+ \gamma(A \backslash E)$.
\end{proof}
\begin{exl}
\begin{enumerate}
    \item Отрезок в $\RR^n$, где $n \geq 2$, измерим (так как замкнут) и имеет меру Лебега, равную нулю, так как его можно зажать в прямоугольники сколь угодно малого объёма. По утверждению выше всего его подмножества, коих $2^{\text{КОНТИНУУМ}}$ штук, также измеримы. Значит, в $\RR^n$ множество измеримых по Лебегу функций имеет мощность $2^{\text{КОНТИНУУМ}}$ (больше не может, так как $|\RR^n| = |\RR|$).
    \item На плоскости надо действовать хитрее. То же рассуждение пройдёт, если мы придумаем какое-нибудь континуальное множество, имеющее меру ноль. Утверждается, что нам подойдёт Канторово множество.
\end{enumerate}
\begin{stat}
Канторово множество имеет мощность континуум, измеримо по Борелю (а, значит, и по Лебегу) и имеет меру Лебега, равную нулю.
\end{stat}
\begin{proof}
Первое утверждение следует из того, что число из отрезка $[0, 1]$ принадлежит Канторову множеству, если и только если оно записывается в троичной записи с помощью цифр $0$ и $2$ (по модулю обработки предельных случаев вида $0,22222...$).
\\
Второе утверждение верно, так как мы получили Канторово множество путём выкидывания из отрезка $[0, 1]$ счётного числа открытых интервалов.
\\
Посчитаем меру дополнения к Канторову множеству. Мы имеем один отрезок длины $\frac{1}{3}$, два отрезка длины $\frac{1}{9}$, ... $2^{k-1}$ отрезков длины $\frac{n}{3^k}$. Сумма их длин (мер) равна единице (несложно просуммировать ряд), а тогда мера Канторова множества равна $\lambda([0, 1])-\sum_{k=1}^{\infty} \frac{2^{k-1}}{3^k} = 1-1=0$
\end{proof}
\end{exl}
\begin{defn}
Мера на полукольце $\GA\subseteq \Rho(X)$ называется \se{$\sigma$-конечная мера}{$\sigma$-конечной}, если исходное множество $X$ представляется в виде счётного объединения $\bigcup A_n$, где $A_i \in \GA$, и $\mu(A_i) < \infty$.
\end{defn}
\begin{remark}
Мера Лебега является $\sigma$-конечной.
\end{remark}
Измеримые по Борелю множества устроены просто, однако измеримых по Лебегу множеств, как мы увидели, значительно больше, и про их структуру мы пока ещё ничего не знаем. Но это ситуация поправимая, ведь существует
\begin{stat} Белов называл его гордым словосочетанием \se{Теорема о структуре измеримых множеств}{<<теорема о структуре измеримых множеств>>}
\\ 
Пусть $A \in \Sigma$ - (измеримое по Лебегу) множество. Тогда оно представимо в виде разности $B \backslash E$, где $B \in \GB$, а $\lambda(E)=0$
\end{stat}
\begin{proof}
Для начала рассмотрим случай $\lambda(A)< \infty$. Для произвольного $\epsilon>0$ рассмотрим покрытие $A$ попарно дизъюнктными элементами полукольца ячеек $\{c_j\}$ такое, что $\lambda(A)=\mu^*(A)+\epsilon \geq \sum \mu(C_j)$ (здесь мы пользуемся конечностью $\lambda(A)$). Если $C^{\epsilon}=\bigcup C_j$, то $\mu(C^{\epsilon})=\sum \mu(C_j) \leq \lambda(A)+\epsilon$. $D = \bigcap C^{\epsilon} \in \GB$ (хоть написано объединение по всем $\epsilon>0$, достаточно рассмотреть счётную подпоследовательность, стремящуюся к нулю). $\mu(D) = \lim_{\epsilon \ra 0} \mu(C^{\epsilon}) = \lambda(A)$. Также $A \subseteq D$. Тогда $\lambda(A \backslash A) = \mu^*(D \backslash A) = 0$ (в этом месте мы воспользовались измеримостью $A$ - в произвольном случае мы не могли бы использовать аддитивность $mu^*$). Положим теперь $B=D$, $E=A \backslash D$ и получим требуемое.
\\
Чтобы свести случай $\lambda(A)=\infty$ к предыдущему, достаточно рассмотреть по отдельности множества $A \cap A_i$ (они также измеримы и имеют конечную меру Лебега в силу $\sigma$-конечности последней), объединить соответствующие им $B_i$ и $E_i$ и воспользоваться тем, что объединение счётного числа множеств меры ноль также имеет меру ноль (по счётной аддитивности $\lambda$).
\end{proof}


Что на самом деле произошло? Мы придумали счётно-аддитивную функцию $\lambda$ на Борелевских множествах, а потом продлили её на $\Sigma$. Но единственно ли это продолжение? Ответ положительный.
\begin{stat}
Пусть $P(\RR^n)$ - полукольцо ячеек, $\Sigma$ - измеримые по Лебегу подмножества, $\lambda$ - мера Лебега, и $\Delta$ ($\GB \subseteq \Delta \subseteq \Sigma$) - какая-то другая $\sigma$-алгебра со своей мерой $\nu$ такая, что $\nu_{\upharpoonright \GB} = \lambda_{\upharpoonright \GB}$. Тогда $\nu_{\upharpoonright \Delta} = \lambda_{\upharpoonright \Delta}$
\end{stat}
\begin{proof}
Во-первых, $\nu(E)=0 \iff \lambda(E)=0$, так как множество нулевой меры получается аппроксимацией Борелевскими множествами нулевой меры.
\\
Во-вторых, если $A \in \Delta$, то можно найти $E \in \Delta$ такое, что $\mu(E)=\nu(E)=0$, и $A \sqcup E \in \GB $. Но тогда $\mu(A)=\mu(A \sqcup E) -\mu(E)= \nu(A \sqcup E) - \nu(E) = \nu(A)$.
\end{proof}
\begin{stat}
\
\begin{itemize}
    \item Мера Лебега инвариантна относительно сдвига. А именно, если $E \in \Sigma$, и $r \in \RR^n$, то $\lambda(E+r)=\lambda(E)$
    \item Пусть $\mu$ - какая-то счётно-аддитивная мера на $\GB$, инвариантная относительно сдвига. Тогда $\mu=c \lambda$ для некоторой константы $c$.
\end{itemize}
\end{stat}
\begin{proof}
\begin{itemize}
\
    \item Для полуинтервалов это очевидно, а если $\{X_i\}$ - покрытие $E$, то $\{X_i+r\}$ - покрытие $E+r$.
    \item Для простоты ограничимся одномерным случаем, хотя в случае произвольной размерности доказательство будет таким же. 
Пусть $c=\mu([0, 1))$. Тогда $\mu(a, b) = c(b-a)$. Действительно, если $b-a = \frac{p}{q} \in \QQ$, то $\mu(a, b) = \mu(0, \frac{p}{q}) = p \cdot \mu(0, \frac{1}{q})=\frac{c}{q}$. А если $b-a \notin \QQ$, то можно приблизить рациональными. Значит, на полуинтервалах меры $\lambda$ и $c \cdot \mu$ совпадают, а, значит, они совпадают везде, так как мера продолжается единственным образом.
\end{itemize}
\end{proof}


\subsection{Измеримость.}
 
На прошлой лекции у нас было множество $X$, в котором есть $\GA \subset 2^X$ - полукольцо, а также полуаддитивная мера $\mu : \GA \ra \overline{\RR_+}$, и мы научились 
 
\begin{itemize}
\item определить сигма-алгебру $\Sigma \subset \GA$ измеримых множеств относительно $\mu$; 
\item продолжить меру до сигма-аддитивной на $\GA$; 
\item доказывать единственность этого продолжения; 
\item и если исходное $\mu|_{\GA}$ - $\sigma$-аддитивно, то продолжение совпадает с исходной на $\GA$. 
\end{itemize}
 
В дальнейшем будем использовать эти факты, обозначив тройку $(X, \Sigma, \mu)$ как \se{Пространство-мера}{пространство-мера}, причём обычно считают $\mu$ счётноаддитивной.
 
\begin{defn}
Пусть у нас есть $(X, \Sigma, \mu)$ и $(Y, \Delta, \nu)$. $f: X \ra Y$ \se{Измеримость}{измеримо}, если $\forall A \in \Delta$, $f^{-1}(A) \in \Sigma$. 
\end{defn}
 
\begin{exl}
Пусть $X$, $Y$ - топологические пространства. Тогда там есть естественные $\sigma$-алгебры, «натянутые» на все открытые множества: $B(X)$, $B(Y)$ - Борелевские $\sigma$-алгебры. Тогда если $f: X \ra Y$ - непрерывно, то и измеримо по Борелю.
\end{exl}
 
Пусть у нас есть $(Y, \Delta, \mu)$ и множество $\GB \subset \Rho(Y)$. Расширение, наименьшая сигма-алгебра, которая это $\GB$ содержит - $\overline{\GB}$. Если она совпадает с $\Delta$, то $\GB$ - образующее множество в $\Delta$. Это множество можно и желательно выбирать как можно меньше.
 
\begin{exl}
Пусть $X = \RR^n$, $\Sigma = B(X)$, что можно выбрать поменьше? Можно рассмотреть \se{Диадическое разбиение}{диадические разбиения}, то есть, все такие кубики, вершины которых лежат в двоично-рациональных точках. То есть, набор кубиков, устроеный как
 
\[
[\frac{p_1}{2^k}, \frac{p_1+1}{2^k}) \times[\frac{p_2}{2^k}, \frac{p_2+1}{2^k}) \times ... \times [\frac{p_n}{2^k}, \frac{p_n+1}{2^k}).
\]
 
Обозначим это разбиение как $D$. Ясно, что любое открытое множество $G$ в $\RR^n$ представимо в виде объединения $G=\bigcup D_i$ кубиков из $D$. Как следствие, $D$ порождает Борелевскую $\sigma$-алгебру. При этом, если есть два кубика, то они либо не пересекаются, либо один находится внутри другого. Таким образом, можно считать, что все $D_i$ попарно дизъюнктны. 
\end{exl}
 
 
 
\begin{stat}
Пусть $G_1$ и $G_2$ - области в $\RR^n$, и $f: G_1 \ra G_2$ - гомеоморфизм, и дополнительно $f \in Lip(G_1)$. Пусть также есть измеримое по Лебегу множество $B \subset \Sigma_\lambda$, $B\subset G_1$, тогда $f(B)$ тоже измеримо по Лебегу (Лебегово)
\end{stat}
 
\begin{proof}
Воспользуемся тем, что $B$ имеет вид $\tilde{B} \backslash E$, где $\tilde{B} \in B(\RR^n)$, а  $\lambda(E)=0$. Тогда $f(B) = f(\tilde{B}) \backslash f(E)$. Уменьшаемое - борелевское, но тогда нужно доказать, что $\lambda(f(E)) = 0$, чтобы заключить, что $f(E)$ измеримо, а, значит, тогда и $f(B)$ будет измеримым. $\lambda(E) = 0$ $\iff $ внешняя мера множества $E$ равна нулю, т.е. существует набор диадических кубиков такой, что их объединение содержит $E$, а сумма объёмов этих кубиков меньше $\varepsilon$. Тогда $f(E) \subset \bigcup f(Q_j)$, а $\diam f(Q_j) \leq C(\diam Q_j)^n$ (где $Q_j$ - кубики, $C$ - константа липшицевости, делённая на $n$-ую степень отношения длины главной диагонали к стороне куба), а следовательно, $\lambda^*(f(E))\leq \Const \cdot \varepsilon$. То есть, если гомеоморфизм липшицев, то образ измеримого измерим.
\end{proof}
 
\subsection{Небольшое отступление.}
 
Давайте для простоты рассмотрим $n=2$. Тогда если у нас есть $A$, покрываемое кубиками, то существует его покрытие кубиками такое, что сумма их площадей $\sum (diam Q_j)^2$ конечна. Если мы рассмотрим отрезок на плоскости, то ситуация аналогична, только $\diam$ не во второй степени а в первой. И если мы обратим особое внимание на эти показатели степени (отображающие размерность), то это приведёт нас к хаусдорфовой (дробной) размерности:  
 
\begin{defn}
Пусть $B \subset \RR^n$, $\{Q_j\}$. Мы выбираем показатели $s$ такие, что для любого $\varepsilon > 0$ можно выбрать семейство кубиков $\{Q_j\}$ со следующими свойствами:
 
\begin{itemize}
\item $B \subset \bigcup Q_j$.
\item $diam Q_j < \varepsilon$
\item $\sum (\diam Q_j)^s < \infty$.
\end{itemize}
 
Тогда $\inf\{s : \exists \:\text{такое разбиение}\} = \dim_H(B)$ - \se{Хаусдорфова размерность множества}{Хаусдорфова размерность} множества $B$. Более того, после того, как размерность определена, $s_0 = \dim_H(B)$, можно говорить о \se{Мера Хаусдорфа}{мере хаусдорфа}: $\mu_s(B) = \inf_{\text{по всем покрытиям } B} \{ \sum (diam Q_j)^{s_0} \}$.
\end{defn}
 
\subsection{Измеримые функции.}
\begin{defn}
Функция
\[
f: (X, \Sigma, \mu) \ra \RR \: (или \CC),
\]
называется \se{Измеримая по Лебегу функция}{измеримой по Лебегу}, если она
измерима в вышеупомянутом смысле. 
\end{defn}
\begin{defn}
Пусть $f: X \ra \RR$, тогда 
\[
E_a(f) = \{x \in X : f(x) < a\} 
\]
- множества Лебега.
\end{defn}
\begin{remark}
Множества $\{x \in X : f(x) > a\}$, $\{x \in X : f(x) \geq a\}$ и $\{x \in X : f(x) \leq a\}$ также иногда называются множествами Лебега.
\end{remark}
\begin{stat}
$f: X \ra \RR$ измеримо тогда и только тогда, когда $E_a(f) \in X$ для любого $a$. 
\end{stat}
\begin{proof}
$[a, b) = E_b(f) \backslash E_a(f)$
\end{proof}
\begin{stat}
Если у нас есть измеримые функции $f_1$ и $f_2$, то их сумма $f_1+f_2$ и произведение $f_1f_2$ тоже измеримы. Если $X$ - метрическое пространство, и $f_2$ непрерывна, то $\frac{f_1}{f_2}$ также измерима там, где знаменатель не обращается в ноль.
\end{stat}
\begin{proof}
Рассмотрим комбинацию отображений: $X \ra \RR^2 \ra \RR$, устроеную следующим образом: $x \mapsto (f_1(x), f_2(x))$, $(x, y) \mapsto x+y$. Оба этих отображения измеримы. Сквозное отображение сопоставляет точке $x$ число $f_1(x)+f_2(x)$ и будет измеримым как композиция измеримых;
\end{proof}
\begin{stat}
Пусть теперь $\{f_i\}_{i=1}^{\infty}$ - измеримые фукнции. Тогда их поточечный супремум $f(x)=\sup_{i} \{f_i(x)\}$ также измерим.
\end{stat}
\begin{proof}
Выберем число $a \in \RR$ и посмотрим на $E_a(f)$. Хотим доказать, что $E_a(f)$ измеримо по Борелю.  
Условие $x \in E_a(f)$ равносильно тому, что для любого $i$, $f_i(x) \leq a$, которое, в свою очередь, можно записать в виде $x \in \bigcap_i \{x | f_i(x) \leq a\}$. Значит, $E_a(f)$ в точности равно $ \bigcap_i \{x | f_i(x) \leq a\}$. Каждое из написанных в фигурных скобках множеств измеримо, значит, и $E_a(f)$ тоже измеримо.
\end{proof}
\begin{remark}
Аналогично, измеримы фукнции $\inf_i \{f_i(x)\}$,  $\limsup{f_i(x)} = \inf_m \sup_{k>m} f_i(x)$ и $\lim_i f_i(x)$ (если существует).
\end{remark}
Если мы захотим рассматривать функции $f: X \ra [-\infty, \infty]$, то дополнительно в определении измеримости надо потребовать, чтобы $f^{-1}(\infty)$ и $f^{-1}(-\infty)$ были измеримы.
 
\subsection{Интеграл Лебега и теоремы Леви}
Пусть есть функция $f: X \ra \RR$.
\begin{enumerate}
    \item Разобъём её на положительную и отрицательную части: $f=f_-f_-$, где, напомним,  $f_+(x)=\max(f(x), 0)$ и $f_-(x)=-\min(f(x), 0)$ (заметим, что $f_+$ и $f_-$ - наотрицательные функци).
    \item Если мы определим интеграл Лебега $I(f)$ для неотрицательных функций, то сможем определить и для произвольной функции $g=g_+-g_-$: $I(g)=I(g_+)-I(g_-)$ при условии, что хотя бы один из интегралов $I(g_+)$ и $I(g_-)$ меньше бесконечности. Если же оба интеграла равны бесконечности, то определить интеграл Лебега от функции $g$ мы не можем.
    \item Таким образом, наша текущая цель - определить интеграл Лебега от неотрицательной измеримой функции $f: X \ra \RR$. Для этого мы будем пользоваться определёнными ранее простыми функциями.
    
    Пусть $f(x) = \sum_{k=1}^na_k \chi_{E_k}(x)$ - простая функция, $E_k$ - измеримые множества. Для неё мы уже определяли $I(f)=\sum_{k=1}^na_k\mu(E_k)$.
    
    \textbf{Идея:} Приблизить произвольную функцию простыми.
\end{enumerate}
 
\begin{theorem}
(\se{Малая теорема Леви}{Малая теорема Леви})
 
Даны неотрицательные простые функции $f$ и $\{g_i\}_{i=1}^{\infty}$. Также $g_i(x) \leq g_{i+1}(x)$, и для почти любого $x$ есть предел $\lim_{i \ra \infty} g_i(x) = f(x)$. Тогда $\lim_{i \ra \infty} I(g_i)=I(f)$
\end{theorem}
\begin{proof}
 
 
Предположим для простоты, что мера конечна.
 
$f(x)=\sum_{k=1}^n a_k\chi_{E_k}(x)$, где $E_k$ попарно дизъюнктны, а $g_i(x)=\sum_{j=1}^{n_i} b_j \chi_{\tilde{E_j}}(x)$. Для каждого $k \in [1, n]$ рассмотрим функцию $g_i \chi_{E_k}$. Она почти всюду сходится к $a_k \chi_{E_k}$ (при $i \ra \infty$).
Докажем, что $I(\chi_{E_k}g_i) \ra I(a_k\chi_{E_k}) = a_k\mu(E_k)$.
 
Во-первых, $I(\chi_{E_k}g_i)$ возрастает при $i \ra \infty$.
 
Во-вторых, $I(\chi_{E_k}g_i) \leq I(a_k\chi_{E_k})$, а тогда $\lim_{i \ra \infty} I(\chi_{E_k}g_i) \leq I(a_k\chi_{E_k})=a_k\mu(E_k)$, поэтому дальше мы будм доказывать неравенство в другую сторону. Для этого докажем, что $\lim_{i \ra \infty} I(\chi_{E_k}g_i) \geq a_k\mu(E_k)-\epsilon$ для любого $\epsilon>0$.
 
Для любого $\delta>0$ верно неравенство $I(\chi_{E_k}g_i) \geq (a_k-\delta)\mu\{x\in E_k: g_i > a_k-\delta \}$. Нужно доказать, что при достаточно больших $i$ (и фиксированных $\delta$ и $\epsilon$) $\mu\{x \in E_k: g_i > a_k-\delta\} \geq \mu(E_k)-\epsilon$ , потому что в этом случае на искомый интеграл получится оценка $(a_k-\delta)(\mu(E_k)-\epsilon)$.
 
 
Но $\{x\in E_k : g_i(x)> a-\delta\} \subseteq \{x \in E_k: g_{i+1}(x)>a-\delta\}$, а их объединение $\bigcup_i \{x\in E_k : g_i(x)> a-\delta\}$ - это в точности множество тех точек $x$, где $\{g_i(x)\} \ra f(x)$. Значит, $\lim_{i \ra \infty} \mu\{x\in E_k : g_i(x)> a-\delta\} =\mu(E_k)$ 
 
Выберем $n$ достаточно большим, чтобы $\mu(\{x \in E_k: g_n(x)>a_k-\delta\})>\mu(E_k)-\epsilon$, а это нам и требовалось.
\end{proof}
\begin{lemma}
Дана неотрицательная измеримая функция $f$. Тогда существует последовательность неотрицательных простых функций $\{f_i\}$, почти всюду монотонно возрастающих (по $i$) к $f$.
\end{lemma}
\begin{proof}
Для начала разберём случай, когда функция $f$ ограничена: $0 < f(x) < c$. Рассмотрим разбиение $[0, c] = \bigcup_{0 \leq i <c2^n } \underbrace{\left [ \frac{i}{2^n}, \frac{i+1}{2^n} \right ]}_{I^i_n}$, а также множества $E^{(n)}_i = \{x : f(x) \in I^i_n\}$. 
 
Возьмём теперь простую функцию $f_n(x) = \sum_{i=1}^{c \cdot 2^n} \frac{i}{2^n} \chi_{E^{n}_i(x)}$. Говоря иначе, мы нарезали область значений функции $f$, и в каждой "полосочке" огрубили функцию вниз. Тогда понятно, что Последовательность $\{f_n\}$ монотонно возрастает к $f$.
 
Если же функция $f$ не ограничена, то для любого $N$ разделим функцию на две области: там, где она меньше либо равна $N$ и там, где она больше, чем $N$. Первая область приближается как в предыдущем абзаце, а на второй области мы оценим функцию как $N \chi_{\text{эта область}}$. Опять-таки, с ростом $N$ получившаяся последовательность функций будет монотонно возрастать к $f$.
\end{proof}
\begin{defn}
Дана неотрицательная измеримая функция $f: X \ra \RR$. Тогда величина $I(f):=\sup \{I(h), h - \text{простая функция, и } 0 \leq h \leq f\}$ называется \se{Интеграл Лебега}{интегралом Лебега}
\end{defn}
\begin{stat}
\textbf{Свойства интеграла Лебега:}
 
\begin{itemize}
 
    \item \textbf{Монотонность:} Если $0 \leq f_1 \leq f_2$, то $I(f_1) \leq I(f_2)$
    \item \textbf{Аддитивность с простой функцией: } $f$ - измеримая функция, и $0 \leq \phi \leq f$ - простая функция. Тогда $I(f)=I(f-\phi)+I(\phi)$.
    \item \se{Неравенство Чебышёва}{Неравенство Чебышёва:} Даны неотрицательная измеримая функция $f$, вещественное число $a$ и соответствующее множество Лебега $E_a=\{x: f(x) \geq a\}$. Тогда $f \geq a \chi(E_a)$, и $I(f) \geq I(a \chi(E_a)) = a \cdot \mu\{x: f(x) \geq a\}$.
\end{itemize}
\end{stat}
\begin{theorem}
$f$ и $\{f_n\}$ - измеримые неотрицательные функции на пространстве с конечной мерой. Известно, что $\{f_n\}$ почти всюду монотонно возрастает к $f$. Тогда $I(f_n)$ монотонно возрастает к $I(f)$.
\end{theorem}
\begin{proof}
Ясно, что предел $\lim_{n \ra \infty} I(f_n)$ существует и не превосходит $I(f)$. Как и раньше, будем доказывать, что для любого $\epsilon>0$ верно неравенство $\lim_{n \ra \infty} I(f_n) \geq I(f)-\epsilon$.
 
Можно выбрать простую $h \leq f$ так, чтобы выполнялось неравенство $I(f)-\epsilon < I(h)$, поэтому будем доказывать, что $\lim_{n \ra \infty} I(f_n) \geq I(h)$.
 
Для каждого $n$ обозначим через $h_n$ простую функцию, приближающую $f_n$ с погрешностью не более $\frac{1}{2^n}$: $I(f_n)-\frac{1}{2^n} \leq I(h_n)$ и $h_n \leq f_n$. Также пусть $\Tilde{h_n} = \max_{i \leq n}h_i$. Тогда $\Tilde{h_n} \leq f_n$ и $I(f_n)-\frac{1}{2^n} \leq I(\tilde{h_n}) \leq I(f)$, так как $h_i \leq f_i \leq f_n \leq f$.
 
Заметим, что $\{\tilde{h_n}\}$ - возрастающая последовательность простых функций. Докажем, что она почти везде сходится к $f$.
 
Множество точек, где $\tilde{h_n}$ не сходится к $f$ - это в точности $\bigcup_{\epsilon} \{x: \lim_{n \ra \infty} h_n(x)<f(x)-\epsilon\}$. Хотим показать, что это объединение имеет меру ноль.
 
$f(x)-\tilde{h_n}(x) = (f(x)-f_n(x))+(f_n(x)-\tilde{h_n}(x))$. Если вдруг оказалось, что эта разность больше, чем $\epsilon$, то одна из скобок больше, чем $\frac{\epsilon}{2}$.
 
$\mu\{x: |f(x)-f_n(x)|>\frac{\epsilon}{2}\} \ra 0$ при $n \ra \infty$ - это утверждение мы доказали в малой теореме Леви для простых функций, но на самом деле не пользовались их простотой.
 
Чтобы оценить меру $\mu\{x: |f_n(x)-\tilde{h}_n(x)|>\frac{\epsilon}{2}\}$, применим неравенство Чебышёва: $\mu\{x: |f_n(x)-\tilde{h}_n(x)|>\frac{\epsilon}{2}\} \leq \frac{2}{\epsilon} I \left ( f_n(x)-\tilde{h}_n(x) \right ) \leq \frac{2}{\epsilon 2^n} \ra 0$ при $n \ra \infty$.
 
Значит, мера множества точек, где $\tilde{h}_n$ не стремится к $f$, равна нулю.
 
Чтобы завершить доказательство, рассмотрим простые функции $g_n=\min\{\tilde{h}_n, h\}$. Они почти везде монотонно возрастают к $h$, потому что $\tilde{h}_n$ почти везде возрастает к $f$, а $f \geq h$. Тогда по малой теореме Леви $I(g_n)$ монотонно возрастает к $I(h)$. Имеем неравенство $I(f_n) \geq I(\tilde{h}_n)-\frac{1}{2^n} \geq I(g_n)-\frac{1}{2^n}$. Переходя к пределу по $n$, получаем, что $\lim_{n \ra \infty} I(f_n) \geq \lim_{n \ra \infty} I(g_n) = I(h) \geq I(f)-\epsilon$ - мы воспользовались малой теоремой Леви.
\end{proof}
\textbf{Схема доказательства: }
\begin{enumerate}
    \item Доказать в тривиальную сторону
    \item Ослабить (с помощью $\epsilon$) и заменить $f$ на простую функцию $h$
    \item Заменить $f_n$ на простые $h_n$, чтобы они хорошо приближали интеграл.
    \item (\textbf{Типичная идея}) Огранизовать монотонную последовательность: $\tilde{h}_n = \max_{1 \leq i \leq n} h_i$
    \item Доказать, что $\tilde{h}_n$ монотонно возрастают к $f$
    \item Доказать, что $\min\{\tilde{h}_n, h\}$ монотонно возрастают к $h$
    \item Предельный переход по простым функциям с помощью малой теоремы Леви
\end{enumerate}
\begin{stat}
\textbf{Продолжение свойств интеграла Лебега для положительных функций:}
\begin{itemize}
    \item Интеграл Лебега от функции $f$ можно определить не как супремум по всем простым функция, а как предел интеграла простых функций, стремящихся к $f$
    \item \textbf{Линейность:} Если $f_1$, $f_2$ - измеримые неотрицательные, то $I(f_1+f_2)=I(f_1)+I(f_2)$
    \item Если $I(f)=0$, то $f=0$ почти везде.
\end{itemize}
\end{stat}
\begin{proof}
\
\begin{itemize}
    \item Следствие теоремы Леви
    \item Пусть $\{\phi^1_n\}$ приближает $f_1$, $\{\phi^2_n\}$ приближает $f_2$, тогда $\{\phi^1_n+\phi^2_n\}$ приближает $f_1+f_2$
    \item $\{x: f(x)>0\} = \bigcup_{\epsilon} \{x: f(x)>\epsilon\}$, а по неравенству Чебышёва $\mu\{x: f(x)>\epsilon\} \leq \frac{1}{\epsilon}I(f)=0$ 
\end{itemize}
\end{proof}
\begin{defn}
Множество $E$ называется \se{$\sigma$-конечное множество}{$\sigma$-конечным}, если оно представляется в виде счётного объединения множеств конечной меры.
\end{defn}
\begin{stat}
Пусть $f \geq 0$ - измерима, $I(f) < \infty$. Тогда её носитель $\supp(f) = \{x : f(x) \neq 0\}$ - $\sigma$-конечное множество.
\end{stat}
\begin{proof}
$\supp(f) = \bigcup_{\epsilon} \{x : f(x) \geq  \epsilon\}$, и по неравенству Чебышёва $\{x : f(x) \geq  \epsilon\} \leq \frac{1}{\epsilon}I(f) < \infty$.
\end{proof}
 
\subsection{Интеграл как функция множеств}
Пусть задана измеримая функция $f \geq 0$. Для любого измеримого множества $E \in \Sigma$ можно рассмотреть функцию от множества $E$, определённую по правилу $I(f, E) = I(f \cdot \chi_E)$
\\
\begin{theorem}
Дана последовательность вложенных друг в друга множеств $\{E_i\}$, $E_{i+1} \subseteq E_i$, $\mu\{E_1\} < \infty$, $O(f, E_1)<\infty$ и $E=\bigcap_i E_i$. Тогда $$I(f, E) = \lim_{i \ra \infty} I(f, E_i)$$
\end{theorem}
\begin{proof}
Нам надо доказать, что $I(f \cdot \chi_E) = \lim_{i \ra \infty} I(f \cdot \chi_{E_i})$. Хочется применить теорему Леви о монотонной сходимости, но вот незадача: функции монотонно убюывают, а нам нужно возрастание. Для этого мы каждое множество $E_i$ заменяем на $F_i = E_1 \backslash E_i$, тогда $I(f, F_i) = I(f, E_1)-I(f, E_i)$, и можно применять теорему Леви.
\end{proof}
\subsection{Другие предельные переходы под знаком интеграла}
Теорема Леви говорит нам, что если $f_n \nearrow f$, то $I(f_n) \nearrow I(f)$ (какой классный символ, почему я не узнал о его существовании раньше и писал слова <<монотонно возрастает к>>?). Но что, если последовательность $f_n$ вообще не имеет предела?
\begin{defn}
Далее мы будем обозначать $I(f) $ через $\int f d\mu$
\end{defn}
\begin{theorem}
\se{Лемма Фату}{Лемма Фату} Пусть $\{f_n\}$ - последовательность неотрицательных измеримых функций. Тогда $$\liminf_{n \ra \infty} \int f_n d \mu \geq \int \liminf_{n \ra \infty} f_n d \mu$$
\end{theorem}
\begin{proof}
$\liminf_{n \ra \infty} f_n(x) = \sup_{n}\underbrace{ \inf_{k \geq  n} f_k(x)}_{g_n(x)} $. Когда $n$ возрастает, то инфимум берётся по всё меньшему множеству, и поэтому $g_n$ становится всё больше. Значит, $g_n(x)$ возрастает, и по теореме Леви $\int \liminf_{n \ra \infty} f_n d\mu =\int \lim_{n \ra \infty} g_n d\mu = \lim_{n \ra \infty} \int g_n d \mu$, что не больше, чем $\liminf_{n \ra \infty} \int f_n d \mu$
\end{proof}
\begin{defn}
\textbf{Окончательное определение интеграла Лебега}. Дана измеримая функция $f: X \ra \overline{\RR}$. Определим функции $f_+=\max\{f, 0\}$ и $f_- = \max\{-f, 0\}$. Тогда $f_+$и $f_-$ измеримы и неотрицательны. Мы уже умеем определять $\int f_+ d \mu$ и $\int f_- d \mu$. Если оба эти интеграла равны бесконечности, то определить $\int f d \mu$ мы не можем, в противном же случае положим $\int f d \mu = \int f_+ d \mu - \int f_- d \mu$ 
\end{defn}
\begin{defn}
Функция $f$ называется \se{Суммируемая функция}{суммируемой}, если оба интеграла $\int f_+ d \mu$ и $\int f_- d \mu$ конечны или, что равносильно, конечен и $\int |f| d \mu$
\end{defn}
\begin{stat}
\textbf{Свойства интеграла Лебега, в большей степени повторяющие то, что уже было написано ранее:} 
\begin{itemize}
    \item\textbf{ Монотонность: } $f_1 \leq f_2$ $\implies $ $\int f_1 d \mu \leq \int f_2 d \mu$
    \item \textbf{Линейность для суммируемых функций:} Если $f_1$, $f_2$ - суммируемые функции, то $\int (f_1+f_2) d \mu = \int f_1 d \mu + \int f_2 d \mu$
\end{itemize}
\end{stat}
\begin{theorem}
\textbf{Теоремы о предельных переходах под знаком интеграла:}
\begin{enumerate}
    \item \textbf{Монотонный предельный переход} Пусть $(X, \Sigma, \mu)$ - пространство с мерой, $\{f_n\}$ - последовательность функций, $f_n \nearrow f$ почти всюду и  $\int_X f_1 d \mu < \infty$. Тогда существует $$\lim_{n \ra \infty} \int f_n d \mu = \int f d \mu$$
    \item То же самое, только теперь $f_n \searrow f$.
    \item \textbf{Лемма Фату} Пусть $(X, \Sigma, \mu)$ - пространство с мерой, $\{f_n\}$ - последовательность неотрицательных измеримых функций, и $\int \inf_{k \geq 1} f_k d \mu < \infty$. Тогда $$\liminf_{n \ra \infty} \int f_n d \mu \geq \int \liminf_{n \ra \infty} f_n d \mu$$
    \item \se{Теорема о мажорируемой сходимости}{Теорема о мажорируемой сходимости} Пусть $(X, \Sigma, \mu)$ - пространство с мерой, $\{f_n\}$ - последовательность измеримых функций, почти всюду сходящаяся к $f$ (но, возможно, не монотонно). Предположим, есть суммируемая функция $g \geq 0$ такая, что $|f_n|<g$ и $|f|<g$. Тогда $$\int f_n d \mu \ra \int f d \mu$$ при $n \ra \infty$
\end{enumerate}
\end{theorem}
\begin{proof}
\
\begin{enumerate}
    \item Рассмотрим последовательность $g_n=f_n-f_1$. Это неотрицательные функции, монотонно возрастающие к $f-f_1$, а тогда по теореме Леви всё получается.
    \item Рассмотрим последовательность $\{g_n\}$, $g_n=f-f_n$. Она монотонно возрастает (хоть все эти функции и отрицательны), и $\int g_1 d 
\mu < \infty$, а тогда можно применить теорему о монотонной сходимости и получить, что $$\lim_{n \ra \infty} \int (f-f_n) d \mu = \int \lim_{n \ra \infty} (f-f_n) d \mu = \int (f-f) d \mu = 0$$
    \item Определяем функции $g_i$, как в оригинальном доказательстве, а потом рассматриваем функции $h_i = g_i-g_1$ и применяем для них предыдущую версию леммы Фату.
    \item Положим $h_n = |f_n-f|$ и будем доказывать, что $\int   h_n d \mu \underset{n \ra \infty}{\longrightarrow} 0$. Заметим, что написанный интеграл вообще существует, так как $h_n=|f_n-f|<2g$, а функция $g$ суммируема.
    
    Обозначим через $\tilde{h}_n = \sup_{j \geq n} h_j$. Верно неравенство $|\tilde{h}_n| \leq 2g$, и, как следствие, $\int \tilde{h}_n d \mu  < \infty$. Более того, последовательность $\{\tilde{h}_n(x)\}$ поточечно и почти всюду стремится к нулю. Значит, $\lim_{n \ra \infty} \int \tilde{h}_n d \mu = \int 0 d \mu = 0$. Но, разумеется, $0 \leq \lim_{n \ra \infty} |f_n-f| = \lim_{n \ra \infty} h_n \leq \lim_{n \ra \infty} \tilde{h}_n = 0$, откуда $\lim_{n \ra \infty} h_n = 0$, и мы получаем требуемое.
\end{enumerate}
\end{proof}
\begin{remark}
Без требования конечности $\int f_1 d \mu$ (или $\int f_k d \mu$ для некоторого $k$) утверждение пункта 2 становится неверным. Пример: $$f_n(x) = \begin{cases} 0 &\mbox{если } 0 \leq x \leq n \\ 
1 & \mbox{если } x>n \end{cases}$$ Очевидно, что $\infty = \lim_{n \ra \infty} \int f_n d \mu \neq \int \lim_{n \ra \infty} f_n d \mu = 0$
\end{remark}
Пусть задана неотрицательная суммируемая функция $f$. Для любого $E \in \Sigma$
можно определить $I_f(E) = \int_E f d \mu$. Легко проверить, что это $\sigma$-аддитивная мера.
\\
\begin{defn}
Пусть $\mu$, $\nu$ - две меры на одной и той же $\sigma$-алгебре пространства $X$. Мы говорим, что $\nu$ - \se{Абсолютно непрерывная мера}{абсолютно непрервна} относительно $\mu$, если для любого $\epsilon > 0$ существует $\delta > 0$ такое, что из того, что $\mu(E)< \delta$ следует, что $\nu(E)<\epsilon$. В частности, из того, что $\mu(E)=0$, следует, что $\nu(E)=0$.
\end{defn}
\begin{stat}
Если $\mu$ -  $\sigma$-конечная мера, то $I_f(E)$ абсолютно непрерывна относительно неё
\end{stat}
\begin{proof}
Допустим, нет: существует $\epsilon>0$ такое, что для любого $\delta > 0$ есть множество $E_{\delta}$, $\mu\{E_{\delta}\}< \delta$ и $\int_{E_{\delta}} f d \mu > \epsilon$. Выберем последовательность $\{\delta_n\}$, $\delta_n = \frac{1}{2^n}$. Обозначим через $E_n$ множество, соответствующее $\delta_n$, т.е. такое, что $\mu(E_n)<\frac{1}{2^n}$ и $\int_{E_n} f d \mu > \epsilon$. Множества $\{E_n\}$ никак между собой не связаны, поэтому сделаем их монотонными: $\overline{E}_n = \bigcup_{k \geq n} E_k$. Из определения следует, что $\overline{E}_{k+1} \subset \overline{E}_k$, и $\chi_{\overline{E}_k} \searrow \chi_{\bigcap_k \overline{E}_n}$. Оценим меру множества $\overline{E}_n$:  $\mu(\overline{E}_n) \leq \sum_{k=n}^{\infty} \frac{1}{2^k} \underset{n \ra \infty}{\longrightarrow} 0$. Значит, $\mu(\bigcap_k \overline{E}_k) = 0$, и $\chi_{\bigcap_k \overline{E}_n} = 0$ почти всюду. Из всего вышесказанного следует, что $f \chi_{\overline{E}_n}$ монотонно убывает к $f \chi_{\bigcap_k \overline{E}_n} = 0$. Тогда можно поменять предел и интегрирование местами: $$\lim_{n \ra \infty} \int_{\overline{E}_n} f d \mu = \int_X \chi_{\overline{E}_n}f d \mu = \int_X \lim_{n \ra \infty} \chi_{\overline{E}_n}f d \mu = 0$$. С другой стороны, для любого $n$ есть неравенства $$\epsilon < \int_{E_n} f d \mu \leq \int_{\overline{E}_n} f d \mu$$
Противоречие.
\end{proof}
\subsection{Теоремы Тонелли и Фубини}
Пусть $(\GA, \Sigma, \mu)$ и $(\GB, \Delta, \nu)$ - пространства с мерами. Можно построить полукольцо $R = \GA \times \GB = \{X \times Y | X \in \GA, Y \in \GB\}$ и определить на нём $\sigma$-аддитивную меру $\mu \otimes \mu(X \times Y) = \mu(X)\nu(Y)$. По теореме Лебега-Каратеодори в $\GA \times \GB$ есть $\sigma$-алгебра $\Theta$ множеств, измеримых относительно $\mu \otimes \nu$.
\begin{exl}
Пусть $\GA=\RR^n$ с мерой Лебега $\lambda_n$, $\GB=\RR^m$ с мерой Лебега $\lambda_m$. В $\RR^{m+n}$ есть мера Лебега $\lambda_{m+n}$, которая, конечно, совпадает с $\lambda_n \otimes \lambda_m$
\end{exl}
Пусть $(\GA, \Sigma, \mu)$, $(\GB, \Delta, \nu)$ - пространства с мерами, $(\GA \times \GB, \Theta, \mu \otimes \nu)$ - их произведение. Если у нас есть функция $F(x, y): \GA \times \GB \ra \RR$, то она, с одной стороны, может быть измеримой относительно $\mu \otimes \nu$, а, с другой стороны, при фиксированном $x \in \GA$ быть измеримой относительно $\nu$. Хотелось бы понять, как все эти махинации связаны между собой.
\begin{theorem}
\se{Теорема Тонелли}{Теорема Тонелли} (пока что без доказательства)
\\Пусть $F: \GA \times \GB \ra \RR$ - неотрицательная измеримая функция, меры $\mu$ и $\nu$ $\sigma$-конечны.
Тогда "всё можно": 
\begin{enumerate}
\item При почти всех $x \in \GA$ функция $\phi_x(y)=F(x, y ): \GB \ra \RR$ измерима
\item При почти всех $y \in \GB$ функция $\psi_y(x)=F(x, y ): \GA \ra \RR$ измерима
\item $\Phi(x) = \int_{\GB} \phi_x(y) d \nu$ - измерима
\item $\Psi(y) = \int_{\GA} \psi_y(x) d \mu$ - измерима
\item $\int_{\GA} \Phi(x) d \mu = \int_{\GB} \Psi(y) d \nu = \int_{\GA \times \GB} F(x, y) d \mu \otimes \nu$
Альтернативная запись: $$\int_{\GA} \Big ( \int_{\GB} F(x, y) d \nu \Big ) d \mu = \int_{\GB} \Big ( \int_{\GA} F(x, y) d \mu \Big ) d \nu = \int_{\GA \times \GB} F(x, y) d \mu \otimes \nu$$
\end{enumerate}
\end{theorem}
\begin{theorem}
\se{Теорема Фубини}{Теорема Фубини} 
\\
$F(x, y)$ - суммируемая (но уже, возможно, не положительная) относительно $\mu \otimes \nu$ функция. Тогда "всё можно"
\end{theorem}
\begin{proof}
$F(x, y) = F_+(x, y)-F_-(x, y)$. К каждому слагаемому применим теперь теорему Тонелли.
\end{proof}
\subsection{Пространства суммируемых функций}
Пусть $(\GA, \Sigma, \mu)$ - пространство с мерой. Как обычно, на всякий случай считаем меру $\sigma$-конечной.
\begin{defn}
$L^1(\GA, \Sigma, \mu) = \{f : \int_{\GA} |f| d \mu < \infty\}$. Хотелось бы определить норму $||f||_{L^1}:= \int_{\GA} |f| d \mu$, но вот незадача: норма может быть равна нулю, когда функция отлична от нуля на непустом множестве нулевой меры. Поэтому мы будем подразумевать, что наши функции определены с точностью до множества меры нуль, а, если быть точным, введём отношение эквивалентности $f \sim g \iff f-g=0$ почти везде, и будем подразумевать не сами функции, а их классы.
\end{defn}
\begin{exl}
$L^1(\RR^n)$, $l^1 = L^1(\ZZ, \text{считающая мера})$, $L^1(0, 1)$ - функции, сумируемые на отрезке $[0, 1]$.
\end{exl}
\begin{stat}
$L^1(0, 1)$ - нормированное пространство:
\begin{enumerate}
    \item $||f|| \geq 0$, $f=0 \iff ||f||=0$
    \item $||\alpha f|| = |\alpha| \cdot ||f||$
    \item $||f_1+f_2|| \leq ||f_1||+||f_2||$
\end{enumerate}
\end{stat}
\begin{stat}
$L^1(0, 1)$ - полное пространство: если $\{f_n\} \in L^1$ - последовательность Коши, то существует $f \in L^1$ такая, что $||f_n-f|| \ra 0$ при $n \ra \infty$ 
\end{stat}
\begin{proof}
\begin{enumerate}
    \item Строим кандидата на функцию $f$.
    \\
     Хочется рассмотреть ряд $f_1+(f_2-f_1)+(f_3-f_2)+...$. Если бы он сошёлся, то предельная функция нам бы подошла. К сожалению, он сходится не всегда. Но в силу того, что $\{f_n\}$ - последовательность Коши, можно выбрать подпоследовательность $\{f_{n_k}\}$ такую, что $||f_{n_k}-f_{n_{k+1}}||\leq \frac{1}{2^k}$. Так как $\sum ||f_{n_k}-f_{n_{k+1}}||<\infty$, мы можем переставить порядки сумирования и интегрирования: $\infty > \sum_k \int |f_{n_k}-f_{n_{k+1}}| d \mu = \int \sum_k |f_{n_k}-f_{n_{k+1}}| d \mu$. Значит, подынтегральный ряд сходится почти всюду. Определим $f(x) = \lim_{k \ra \infty} f_{n_k}(x)$.
    \item Доказываем, что найденная функция подходит, т.е. что $||f_n-f|| \ra 0$ при $n \ra \infty$.
    \\
    Применим неравенство треугольника: $||f_n-f|| = ||f_n-f_{n_k}||+||f_{n_k}-f||$. Если $n$ и $k$ достаточно велики, то первая норма мала из-за того, что $\{f_n\}$ - последовательность Коши, а вторая норма мала, так как $\{f_{n_k}\}$ приближают $f$.
\end{enumerate}
\end{proof}
Обозначим через $\GC_0(\RR)$ множество всех непрерывных функций из $\RR$ и $\RR$ с компактным носителем.
\begin{stat}
$\GC_0(\RR)$ плотно в $L^1(\RR)$
\end{stat}
\begin{proof}
\
\begin{itemize}
    \item Для любой функции $f \in L^1(\RR)$ обозначим $$f_N(x) = \begin{cases} 0 &\mbox{если } |x|\geq N \\ 
f(x) & \mbox{если } |x|<N \end{cases}$$ Все $f_N$ - функции с компактным носителем, и $||f-f_N|| = \int_N^{\infty} f d \mu \ra 0$ при $n \ra \infty$
\item Мы умеем приближать $f_N^+$ и $f_N^-$, а, значит, и $f_N$, простыми функциями (которые тоже имеют компактный носитель), поэтому достаточно доказать утверждение лишь для них. А на самом деле даже только для характеристических, так как линейные комбинации последних - это и есть простые функции.
\item Пусть $E$ - измеримое множество, являющееся подмножеством какого-то конечного интервала. Его можно покрыть дизъюнктным набором интервалов $\{I_k\}$ причём таким, что $\mu(\bigcup I_k \backslash E) < \epsilon$. Тогда $\chi_E$ приближается функцией $\sum_k \chi_{I_k}$, а характеристическая функция интервала уж точно приближается непрерывной функцией. 
\end{itemize}
\end{proof}
\begin{remark}
То же самое верно для функций из $\RR^n$
\end{remark}
\begin{cons}
Пусть всё происходит на отрезке $[0, 1]$. Тогда любую измеримую функцию $f$ можно приблизить непрерывной. Но по теореме Вейерштрасса любую непрерывную функцию на $[0, 1]$ можно приблизить полиномом. Как следствие, любая измеримая функция на отрезке также приближается (по мере) полиномом.
\end{cons}
\begin{theorem} \se{Теорема Мюнца}{Теорема Мюнца}
Рассмотрим последовательность функций $\{t^{\lambda_n}\}$, где $0=\lambda_0<\lambda_1 < \lambda_2 <...$. Следующие утверждения эквивалентны:
\begin{enumerate}
    \item Любую функцию $f \in C[0, 1]$ можно равномерно приблизить <<обобщёнными полиномами>> $\sum_{k=0}^N \alpha_k t^{\lambda_k}$
    \item Ряд $\sum_{k=1}^{\infty} \frac{1}{\lambda_k}=\infty$
\end{enumerate} 
\end{theorem}
\textbf{Общий случай:} Пусть $(\GA, \Sigma, \mu)$ - пространство с мерой, $\GA$ - топологическое пространство. Предположим, $\GA$ хаусдорфово, а мера $\mu$ регулярна (неформально говоря, любое множество $E$ можно <<снизу подпереть компактами>> и <<сверху подпереть открытыми множествами>>; формальное определение см. в начале конспекта). 
\begin{stat}
Непрерывные суммируемые функции плотны в $L^1(\GA, \Sigma, \mu)$
\end{stat}
Для доказательства потребуется
\begin{lemma}
\se{Лемма Урысона}{Лемма Урысона} Пусть $X$ - Хаусдорфово пространство, $K \subset G \subset X$, $K$ - компакт, $G$ - открытое. Тогда существует непрерывное отображение $f: X \ra [0, 1]$ такое, что $f_{\upharpoonright_K}=1$ и $f_{\upharpoonright_{X \backslash G}}=0$
\end{lemma}
\subsection{Свёртка}
\begin{defn}
Пусть $f, g \in L^1(\RR)$. Их \se{Свёртка функций}{свёрткой} называется функция $h(t)=(f*g)(t) = \int_{\RR} f(t-\tau)g(\tau) d \tau$. Очень похоже на умножение полиномов.
\end{defn}
\begin{stat}
\textbf{Свойства свёртки: } \begin{enumerate}
    \item \textbf{Коммутативность: }$f*g = g*f$
    \item \textbf{Дистрибутивность: }  $f*(g_1+g_2)=f*g_1+f*g_2$
    \item $||f*g|| \leq ||f|| \cdot ||g||$
\end{enumerate}
\end{stat}
\begin{proof}
\
\begin{enumerate}
    \item Очевидно из определения
    \item Очевидно из определения
    \item $||f*g|| = \int_{\RR} |\int_{\RR}f(t-\tau)g(\tau) d \tau | dt \leq \int_{\RR} \int_{\RR}|f(t-\tau)|\cdot |g(\tau)| d \tau  dt \overset{\text{ Тонелли}}{=}\int_{\RR} \int_{\RR}|f(t-\tau)|\cdot |g(\tau)| d t  d \tau = \int_{\RR} |g(\tau)| \int_{\RR}|f(t-\tau)|\cdot  d t  d \tau =||f|| \cdot ||g||$. Заметим, что заодно мы доказали существование свёртки.
\end{enumerate}
\end{proof}
Вопрос: Как себя ведёт функция из $L^1$ при сдвиге?
\begin{stat}
$f \in L^1$, тогда $||f(t)-f(t+\tau)|| \underset{\tau \ra 0}{\longrightarrow} 0$
\end{stat}
\begin{proof}
Если бы $f$ была непрерывной и имела компактный носитель, то утверждение бы следовало из теоремы Кантора о равномерной непрерывности.
Пусть теперь $f \in L^1(\RR)$, $\epsilon>0$, хотим найти $\delta=\delta(\epsilon)$ такое, что при любом $\tau$, $|\tau|<\delta$, верно неравенство $||f(t)-f(t+\tau)||<\epsilon$. Мы уже знаем, что функцию $f$ можно приблизить непрерывной функцией с компактным носителем: $||g-f||<\frac{\epsilon}{3}$. Тогда $||f(t)-f(t+\tau)|| \leq ||f(t)-g(t)||+||g(t)-g(t+\tau)||+||g(t+\tau)-f(t+\tau)||$. Первое и третье слагаемые меньше, чем $\frac{\epsilon}{3}$, а второе слагаемое тоже будет маленьким, если $\tau$ достаточно мало (теорема Кантора о равномерной непрерывности).
\end{proof}











\hypertarget{dex}
    \printindex


%staryi_variant
%\hypertarget{uk}{Основные понятия.}

%\begin{multicols}{2}
%    \hyperlink{}{} \ 
%\end{multicols}



%novyi_variant


\end{document}
\documentclass[a4paper]{article}

\usepackage[12pt]{extsizes}
\usepackage[utf8]{inputenc}
\usepackage[unicode, pdftex]{hyperref}
\usepackage{cmap}
\usepackage{mathtext}
\usepackage{multicol}
\setlength{\columnsep}{1cm}
\usepackage[T2A]{fontenc}
\usepackage[english,russian]{babel}
\usepackage{amsmath,amsfonts,amssymb,amsthm,mathtools}
\usepackage{icomma}
\usepackage{euscript}
\usepackage{mathrsfs}
\usepackage[dvipsnames]{xcolor}
\usepackage[left=2cm,right=2cm,
    top=2cm,bottom=2cm,bindingoffset=0cm]{geometry}
\usepackage[normalem]{ulem}
\usepackage{graphicx}
\usepackage{makeidx}
\makeindex
\graphicspath{{pictures/}}
\DeclareGraphicsExtensions{.pdf,.png,.jpg}
%\usepackage[usenames]{color}
\hypersetup{
     colorlinks=true,
     linkcolor=magenta,
     filecolor=magenta,
     citecolor=black,      
     urlcolor=magenta,
     }
\usepackage{fancyhdr}
\pagestyle{fancy} 
\fancyhead{} 
\fancyhead[LE,RO]{\thepage} 
\fancyhead[CO]{\hyperlink{uk}{к списку объектов}}
\fancyhead[LO]{\hyperlink{sod}{к содержанию}} 
\fancyfoot{}
\newtheoremstyle{indented}{0 pt}{0 pt}{\itshape}{}{\bfseries}{. }{0 em}{ }

\renewcommand\thesection{}
\renewcommand\thesubsection{}

%\geometry{verbose,a4paper,tmargin=2cm,bmargin=2cm,lmargin=2.5cm,rmargin=1.5cm}

\title{Дифференциальные уравнения \\ 
и динамические системы}
\author{Алешин Артем \\ 
    на основе лекций Пилюгина С. Ю. \\
    под редакцией @keba4ok}
\date{5 сентября 2021.}


%envirnoments
    \theoremstyle{indented}
    \newtheorem*{theorem}{Теорема}
    \newtheorem*{lemma}{Лемма}
    \newtheorem*{alg}{Алгоритм}

    \theoremstyle{definition} 
    \newtheorem*{defn}{Определение}
    \newtheorem*{exl}{Пример(ы)}
    \newtheorem*{prob}{Задача}

    \theoremstyle{remark} 
    \newtheorem*{remark}{Примечание}
    \newtheorem*{cons}{Следствие}
    \newtheorem*{exer}{Упражнение}
    \newtheorem*{stat}{Утверждение}
%esli ne hochetsa numeracii - nuzhno prisunut' zvezdochku-pezsochku

\definecolor{brilliantrose}{rgb}{1.0, 0.33, 0.64}

%declarations
        %arrows_shorten
            \DeclareMathOperator{\la}{\leftarrow}
            \DeclareMathOperator{\ra}{\rightarrow}
            \DeclareMathOperator{\lra}{\leftrightarrow}
            \DeclareMathOperator{\llra}{\longleftrightarrow}
            \DeclareMathOperator{\La}{\Leftarrow}
            \DeclareMathOperator{\Ra}{\Rightarrow}
            \DeclareMathOperator{\Lra}{\Leftrightarrow}
            \DeclareMathOperator{\Llra}{\Longleftrightarrow}

        %letters_different
            \DeclareMathOperator{\CC}{\mathbb{C}}
            \DeclareMathOperator{\ZZ}{\mathbb{Z}}
            \DeclareMathOperator{\RR}{\mathbb{R}}
            \DeclareMathOperator{\NN}{\mathbb{N}}
            \DeclareMathOperator{\HH}{\mathbb{H}}
            \DeclareMathOperator{\LL}{\mathscr{L}}
            \DeclareMathOperator{\KK}{\mathscr{K}}
            \DeclareMathOperator{\GA}{\mathfrak{A}}
            \DeclareMathOperator{\GB}{\mathfrak{B}}
            \DeclareMathOperator{\GC}{\mathfrak{C}}
            \DeclareMathOperator{\GD}{\mathfrak{D}}
            \DeclareMathOperator{\GN}{\mathfrak{N}}
            \DeclareMathOperator{\Rho}{\mathcal{P}}
            \DeclareMathOperator{\FF}{\mathcal{F}}

        %common_shit
            \DeclareMathOperator{\Ker}{Ker}
            \DeclareMathOperator{\Frac}{Frac}
            \DeclareMathOperator{\Imf}{Im}
            \DeclareMathOperator{\cont}{cont}
            \DeclareMathOperator{\id}{id}
            \DeclareMathOperator{\ev}{ev}
            \DeclareMathOperator{\lcm}{lcm}
            \DeclareMathOperator{\chard}{char}
            \DeclareMathOperator{\codim}{codim}
            \DeclareMathOperator{\rank}{rank}
            \DeclareMathOperator{\ord}{ord}
            \DeclareMathOperator{\End}{End}
            \DeclareMathOperator{\Ann}{Ann}
            \DeclareMathOperator{\Real}{Re}
            \DeclareMathOperator{\Res}{Res}
            \DeclareMathOperator{\Rad}{Rad}
            \DeclareMathOperator{\disc}{disc}
            \DeclareMathOperator{\rk}{rk}
            \DeclareMathOperator{\const}{const}
            \DeclareMathOperator{\grad}{grad}
            \DeclareMathOperator{\Aff}{Aff}
            \DeclareMathOperator{\Lin}{Lin}
            \DeclareMathOperator{\Prf}{Pr}
            \DeclareMathOperator{\Iso}{Iso}

        %specific_shit
            \DeclareMathOperator{\Tors}{Tors}
            \DeclareMathOperator{\form}{Form}
            \DeclareMathOperator{\Pred}{Pred}
            \DeclareMathOperator{\Func}{Func}
            \DeclareMathOperator{\Const}{Const}
            \DeclareMathOperator{\arity}{arity}
            \DeclareMathOperator{\Aut}{Aut}
            \DeclareMathOperator{\Var}{Var}
            \DeclareMathOperator{\Term}{Term}
            \DeclareMathOperator{\sub}{sub}
            \DeclareMathOperator{\Sub}{Sub}
            \DeclareMathOperator{\Atom}{Atom}
            \DeclareMathOperator{\FV}{FV}
            \DeclareMathOperator{\Sent}{Sent}
            \DeclareMathOperator{\Th}{Th}
            \DeclareMathOperator{\supp}{supp}
            \DeclareMathOperator{\Eq}{Eq}
            \DeclareMathOperator{\Prop}{Prop}


%env_shortens_from_hirsh            
    \newcommand{\bex}{\begin{example}\rm}
    \newcommand{\eex}{\end{example}}
    \newcommand{\ba}{\begin{algorithm}\rm}
    \newcommand{\ea}{\end{algorithm}}
    \newcommand{\bea}{\begin{eqnarray*}}
    \newcommand{\eea}{\end{eqnarray*}}
    \newcommand{\be}{\begin{eqnarray}}
    \newcommand{\ee}{\end{eqnarray}}
    \newcommand{\abs}[1]{\lvert#1\rvert}
        \newcommand{\bp}{\begin{prob}}
        \newcommand{\ep}{\end{prob}}
    
\begin{document}
%ya_ebanutyi
\newcommand{\resetexlcounters}{%
  \setcounter{exl}{0}%
} 
\newcommand{\resetremarkcounters}{%
  \setcounter{remark}{0}%
} 
\newcommand{\reseconscounters}{%
  \setcounter{cons}{0}%
} 
\newcommand{\resetall}{%
    \resetexlcounters
    \resetremarkcounters
    \reseconscounters%
}
\newcommand{\cursed}[1]{\textit{\textcolor{brilliantrose}{#1}}}
\newcommand{\de}[3][2]{\index{#2}{\textbf{\textcolor{brilliantrose}{#3}}}}
\newcommand{\re}[3][2]{\hypertarget{#2}{\textbf{\textcolor{brilliantrose}{#3}}}}
\newcommand{\se}[3][2]{\index{#2}{\textit{\textcolor{brilliantrose}{#3}}}}
\maketitle 
\newpage
\hypertarget{sod}
\tableofcontents
\newpage

\subsection{Литература}
\begin{itemize}
    \item В. И. Арнольд Обыкновенные дифференциальные уравнения
    
    \item Ю. Н. Бибиков Общий курс дифференциальных уравнения
    \item С. Ю. Пилюгин Пространства динамических систем
\end{itemize}   
\begin{defn}
  \se{Дифференциальное уравнение}{Дифференциальное уравнение} -- уравнение от неизвествной фукции $y(x)$, где $x \in \RR $ -- независимая переменная, вида
    \[f(x,y,y',\ldots,y^{(n)}) = 0\]
\end{defn}
\section{Дифференциальные уравнения 1-го порядка, разрешенные относительно производной}
\begin{defn}
  \se{Дифференциальное уравнение!1-го порядка}{Дифференциальное уравнение 1-го порядка}, разрешенное относительно производной -- уравнение вида $y' = f(x,y), f \in C(G)$, где $G$ -- область (открытое связное множество) в $\RR_{x,y}^2$
\end{defn}
\begin{defn}
  $y : (a,b) \to \RR$ -- \se{Решение дифференциального уравнения}{решение} на $(a,b)$, если  
  \begin{itemize}
    \item $y$ -- дифференцируема;
    \item $(x,(y(x)) \in G, x \in (a,b)$;
    \item $y'(x) \equiv f(x,y(x))$ на $(a,b)$.
  \end{itemize} 
\end{defn}
\begin{exl} \ 
  \begin{itemize}
    \item $y' = ky, k > 0, \  G = \RR^2$;
    \item $\forall c \in \RR \  y(x) = ce^{kx}$ -- решение на $\RR$.
  \end{itemize}
\end{exl}
\begin{defn}
  \se{Интегральная кривая}{Интегральная кривая} -- график решения.
\end{defn}
\subsection{Задача Коши}
\begin{defn}
$y(x)$ -- решение \se{Задача Коши}{задачи Коши} с начальным условем $(x_0,y_0)$, если
\begin{itemize}
    \item $y(x)$ -- решение дифференциального уравнения на $(a,b)$;
    \item $y(x_0) = y_0$.
\end{itemize}
\end{defn}
\subsection{Единственность}
\begin{defn}
  $(x_0,y_0)$ -- \se{Точка единственности}{точка единственности} для задачи Коши, если $\forall y_1, y_2$ -- решения $\exists (\alpha, \beta) \ni x_0: y_1|_{(\alpha,\beta)} = y_2|_{(\alpha,\beta)}$.
\end{defn}
\begin{exl}
  \[y' = 3 \sqrt[3]{y^2}\]
  Если $(x_0,y_0) = 0$, то возможны следующие решения:
  \begin{itemize}
  \item \[y_1 = 0\]
  \item \[y_2 = \begin{cases}
        0 & x \leqslant 0 \\
        x^3 & x > 0
      \end{cases}\]
  \item \[y_3 = \begin{cases}
        x^3 & x \leqslant 0 \\
        0 & x > 0
      \end{cases}\]
  \end{itemize}
  Точка $(0,0)$ не является точкой единственности, но при этом $(1,1)$ уже будет точкой единственности
\end{exl}
\subsection{Поле направлений}
\begin{defn}
  Из уравнения $y' = f(x,y)$ мы можем вычислить \se{Коэффициент наклона}{коэффициент наклона} в каждой точке $(x,y)$
  \[k = y'(x) = f(x,y)\]
  Если в каждой точке $(x,y)$ области $G$ провести отрезок с угловым коэффициентом равным $f(x,y)$, то получится \se{Поле направлений}{поле направлений}. Любая интегральная кривая в каждой своей точке касается соответствующего отрезка.
\end{defn}
\subsection{Основные теоремы}
\begin{theorem}[\cursed{О существовании}]
  Если $y' = f(x,y), \ f \in C(G)$, то $\forall (x_0,y_0) \in G \ \exists $ решение задачи Коши с начальными данными $(x_0,y_0)$
  $G$ называется \se{Область!существования}{областью существования}.
\end{theorem} \ 
\begin{theorem}[\cursed{О единственности}]
  Если $y' = f(x,y), \ f, \frac{\partial f}{\partial y} \in C(G)$, то $\forall (x_0,y_0) \in G \ \exists $ единственное решение задачи Коши с начальными данными $(x_0,y_0)$
  $G$ называется \se{Область!единственности}{областью единственности}.
\end{theorem}
\section{Интегрируемые типы дифференциальных уравнений 1-го порядка}
\begin{exl}
  $y' = f(x)$ -- из анализа знаем, что единнственным решение при данном условии $(x_0,y_0)$ будет \[y(x) = y_0 + \int_{x_0}^xf(t)dt\]
\end{exl}
\subsection{Интеграл}
Пусть $H \subset G$ -- область
\begin{defn}
  Функция $U \in C^1(H,\mathbb{R})$ называется \se{Интеграл уравнения}{интегралом уравнения} $y' = f(x,y)$ в $H$, если выполнены следующие условия:
  \begin{itemize}
  \item $\frac{\partial U}{\partial y} \not = 0$;
  \item если $y(x), x \in (a,b)$ -- решение с $(x,y(x)) \in H$, то $U(x,y(x)) = \const$.
  \end{itemize}
\end{defn}
\begin{theorem}[Напоминание \cursed{теоремы о неявной функции}]
  \[F : H \subset \mathbb{R}^2 \to \mathbb{R}, F \in C^1  \]
  Если
  \begin{itemize}
    \item
      \[F(x_0,y_0) = 0\]
    \item
      \[ \frac{\partial F}{\partial y}\bigg|_{(x_0,y_0)} \not  = 0\]
  \end{itemize}
  тогда $\exists I, J$ -- открытые интервалы $x_0 \in I, y_0 \in J$, $\exists z(x) \in C^1(I)$ такая, что
  \begin{itemize}
  \item $z(x_0) = y_0$;
  \item $F(x,y) = 0 \leftrightarrow y = z(x)$ при $(x,y) \in I \times J$.
  \end{itemize}
\end{theorem} \ 
\begin{theorem}[\se{Теорема!об интеграле для дифференциальных уравнений первого порядка}{Об интеграле для дифференциальных уравнений первого порядка}]
  Пусть $U$ -- интеграл $y' = f(x,y)$ в $H \subset G$. Тогда $\forall (x_0,y_0) \in H \ \exists H_0 \subset H, H_0 = I \times J \ni (x_0,y_0)$ и $\exists y(x) \in C^1(I)$ такая что:
  \begin{itemize}
    \item $y(x)$ -- решение задачи Коши с начальными данными $(x_0,y_0)$
    \item $(x,y) \in  H$ и $U(x,y) = U(x_0,y_0) \Rightarrow y = y(x)$
  \end{itemize}
\end{theorem}
\begin{proof} \ 
  Фиксируем произвольную точку $(x_0,y_0)$. Рассмотрим $F(x,y) = U(x,y) - U(x_0,y_0)$.
  $F$ удовлетворяет условию теоремы о неявной функции, так как $\frac{\partial F}{\partial y} = \frac{\partial U}{\partial y} \not  = 0$, поэтому существуют $I_0, J_0 \ I_0 \times J_0 \subset H$ и $\exists y(x) \in  C^1(I_0), \ y(x_0) = y_0$.
  По теореме существования $\exists $ решение $z(x)$ задачи Коши с начальными условиями $(x_0,y_0)$ на некотором промежутке $I \ni x_0$ такое что $(x,z(x)) \in I_0 \times J_0$.
  Тогда по определению интеграла $U(x,z(x)) = \const$ $\Rightarrow F(x,z(x)) = 0 \Rightarrow z(x) = y(x)$.
\end{proof}
\subsection{Дифференциальные уравнения с разделяющимися переменными}
\begin{equation*}
  \begin{aligned}
    & y' = m(x) \cdot n(y) \\
    & m \in C((a,b)), n \in C((\alpha, \beta)) \\
    & G = (a,b) \times (\alpha, \beta)
  \end{aligned}
\end{equation*}

\begin{itemize}
\item $y_0 \in (\alpha, \beta) n(y_0) = 0 \Rightarrow y \equiv y_0$
  
  Проверяется подставнкой
\item $I \subset (\alpha, \beta), n(y) \not  = 0$ при $y \in  I$
  Подсказка:
  Рассмотрим $y(x) : (x,y(x)) \in (a,b) \times I$ и отличную от $0$
  $y' = m(x) n(y)$, на $n(y)$ можно поделить
  \[\frac{y'}{n(y(x))} = m(x), \:
  \int\limits_{x_0}^x\frac{y'(t)dt}{n(y(t))} = \int\limits_{x_0}^x m(t) dt. \]
    Замена $z = y(t)$
  \[\int\limits_{y(x_0)}^{y(x)}\frac{dz}{n(z)} = \int\limits_{x_0}^x m(t) dt,
  \]
 
  Обозначим за $N(y)$ и $M(x)$ некоторые первообразные $\frac{1}{n(y)}$ и $m(x)$ соответственно
  \begin{equation*}
    \begin{aligned}
      N(y(x)) - N(y(x_0)) & = M(x) - M(x_0)\\
      U(x,y) :& = N(y) - M(x).
    \end{aligned}
  \end{equation*}
  Если $y(x)$ -- решение, то $U(x,y(x)) = N(y(x_0)) - M(x_0)$
  \[\frac{\partial U}{\partial y} = \frac{1}{n(y)} \not  = 0. \]
  
\end{itemize}
Это была некоторая эвристика для того, чтобы найти формулу для интеграла.
  
Сформулируем некоторое утверждение, которое позволит нам проверять, является ли  $U$ интегралом.

\begin{stat}
  (\cursed{Критерий интеграла})

  $U$ -- интеграл для уравнения $y' = f(x,y)$ $\Longleftrightarrow$
  \begin{itemize}
  \item \[\frac{\partial U}{\partial y} \not = 0\]
  \item \[\frac{\partial U}{\partial x} + \frac{\partial U}{\partial y} \cdot f \equiv 0\]
  \end{itemize}
\end{stat}

\begin{proof}
  Если $y(x)$ -- решение, то $U(x,y(x)) = \const$

  \[\frac{dU}{dx} \equiv 0
  \]

  \[\frac{d}{dx} U(x,y(x)) = \frac{\partial U}{\partial x}(x, y(x))  + \frac{\partial U}{\partial y} \cdot y'(x) =  \frac{\partial U}{\partial x} + \frac{\partial U}{\partial y} \cdot f \equiv 0
  \]
\end{proof}


Применяя это утверждение к нашему уравнению $y' = m(x)n(y)$ и $U = N(y) - M(x)$ имеем:

\begin{equation}
  \begin{gathered}
    \frac{d}{dx} U = \frac{d}{dx} (N(y) - M(x)) = -m(x) + \frac{1}{n(y)} \cdot m(x)n(y) \equiv 0
  \end{gathered}
\end{equation}

\section{Замена переменных}

\begin{exl}
  \begin{enumerate}
  \item $y' = f(ax +by)$
    
    Новая независимая переменная -- $x$

    Новая искомая функция -- $v = ax+by$

    \[\frac{dv}{dx} = a+by' = a + bf(v)\]

  \item $y' = m(x) n(y)$,  Пусть $n(y) \neq 0 $

    Новая переменная -- $x$

    Новая функция -- $v = N(y)$

    \[\frac{dv}{dx} = \frac{1}{n(y(x))} \cdot y'(x) = m(x)\]
    
    Все сводится к уравнению, решение которого мы уже умеем находить
    \[\frac{dv}{dx} = m(x)\]
    


  \end{enumerate}
\end{exl}


\subsection{Линейное дифференциальное уравнение первого порядка}

\[y' = p(x) y + q(x), \ p,q \in C((a,b))\]

$f(x,y)$ определена на $G = (a,b) \times \mathbb{R} $, $f$ и $\frac{\partial f}{\partial y}$ непрерывны на $G$, поэтому $G$ -- область существования и единственности.


\begin{enumerate}
\item Для начала научимся решать \se{Однородное линейное уравнени}{однородное линенйное уравнение} ($q \equiv 0$)

  \[y' = p(x) y\]

  Есть решение $y \equiv 0, x \in (a,b)$


  Если $y > 0$, то

  \[U = \int \frac{dy}{y} - \int p(x) dx = \log (y) - \int p(x) dx = \log(C)
  \]

  \[y = c e^{\int p(x) dx}\]

  Для $y < 0$ то же самое


\item \se{Метод вариации произвольной переменной}{Метод вариации произвольной переменной} (Лагранж)

  Воспользуемся заменой переменной:

  Новая независимая переменная -- $x$

  Новая функция -- $v(x)$

  Будем искать решение $y(x)$ в виде $y(x) = v(x) e^{\int_{}^{}p(x) dx}$
  \begin{equation*}
    \begin{aligned}
      y'  & = v' e^{\int_{}^{}p(x) dx} + v \cdot p(x) e^{\int_{}^{}p(x) dx} \\
      p(x)y + q(x) & =  p(x) v(x) e^{\int_{}^{}p(x) dx} + q(x) \\
      v' \cdot e^{\int p(x) dx} & = q(x) \\
      v'& = q(x) \cdot e^{-\int p(x) dx} \\
      v  &= \int q(x) e^{-\int p(x) dx} dx \\
      y & = e^{\int p(x) dx} \left( \int q(x) e^{-\int p(x) dx} dx \right)
  \end{aligned}
\end{equation*}

Заметим, что первообразная для $p(x)$ берется одна и та же

Для задачи Коши с начальным условием $(x_0,y_0)$ имеем


\[y  = e^{\int_{x_0}^x p(t) dt} \left( y_0 + \int\limits_{x_0}^x q(s) e^{-\int_{x_0}^x p(t) dt} ds \right)\]

\end{enumerate}

\subsection{Уравнения, сводящиеся к линейным}

\se{Уравнение Бернулли}{Уравнение Бернулли} $y' = p(x)y + q(x) y^m, m = \const$

Исключения -- $m = 0, m = 1$, так как тогда это будет обычное линейное уравнение

Если $m > 0$, то есть решение $y \equiv 0$

Если $y \neq 0$, то возпользуемся заменой переменных
$v = y^{1-m}$

\begin{equation*}
  \begin{aligned}
    \frac{y'}{y^m}& = p(x) y^{1-m} + q(x)\\
    v'& = (1-m) y' y^{-m} \\
    \frac{v'}{(1-m)} & = p(x) v + q(x)
  \end{aligned}
\end{equation*}

Получилось линейное уравнение, которое мы уже умеем решать.

\se{Уравнение Рикатти}{Уравнение Рикатти}
\[y' = ay^2 + bx^{\alpha}, ab \neq 0 \]

Бернулли показал, что при $\alpha = \frac{4k}{2k-1}, k \in \mathbb{Z}$ это уравнение имеет решения.

Луивилль(1841) доказал, что если $\alpha$ -- не число Бернулли и $\alpha \neq 2$, то уравнение Рикатти не интегрируемо.

\subsection{Дифференциальные уравнения первого порядка в симметричной форме}

\se{Уравнение Пфаффа}{Уравнение Пфаффа}
\[m(x,y) dx + n(x,y) dy = 0\]

\begin{defn}
  \se{Дифференциальная 1-форма}{Дифференциальная 1-форма}

  \[F = m(x,y) dx + n(x,y) dy, m,n \in C^1(G), m^2 + n^2 \neq 0\]

\end{defn}

\begin{defn}
  \se{Интегральная кривая дифференциальной формы}{Интегральная кривая дифференциальной формы} $F$ -- гладкая кривая $\gamma(t) = (\gamma_1(t), \gamma_2(t)), t \in (a,b)$

  \[m(\gamma(t)) \dot \gamma_1(t) + n(\gamma(t)) \dot \gamma_2(t) = 0 \text{ на } (a,b)
  \]
\end{defn}

\begin{remark}
  Кривая называется гладкой, если $\exists $ непрерывные $\dot \gamma_1, \dot \gamma_2$ и $(\dot \gamma_1, \dot \gamma_2) \neq 0$
\end{remark}


\cursed{Связь уравнения Пфаффа с обыкновенным дифференциальным уравнением}

Пусть $\gamma(t) = (\gamma_1(t), \gamma_2(t))$ -- интегральная кривая $F$

Выберем $t_0 \in (a,b)$, пусть $\dot \gamma_1(t_0) \neq 0$

Тогда $\exists (\alpha, \beta) \ni t_0: \dot \gamma_1(t)|_{(\alpha, \beta)} \neq 0$

Положим $x = \gamma_1(t)$

Так как $\dot \gamma_1$ -- непрерывна и не обращается в ноль на $(\alpha, \beta)$,  то существует обратная функция.

Тогда $x = \gamma_1(t) \Longleftrightarrow t = \gamma_1^{-1}(x)$

Положим $y = \gamma_2(\gamma_1^{-1})$

Дифференциальное уравнение для $y$:

\[\frac{dy}{dx} = \dot \gamma_2(t) \cdot \frac{d}{dx} (\gamma_1^{-1}(x)) = \frac{\dot \gamma_2(t)}{\dot \gamma_1(\gamma_1^{-1}(x))} = \frac{\dot \gamma_2(t)}{\dot \gamma_1(t)}
\]

$\gamma$ была интегральной кривой формы $F$, то есть выполнялось равенство:

\[m(\gamma(t)) \dot \gamma_1(t) + n(\gamma(t)) \dot \gamma_2(t) = 0\]

Тогда понятно, что
\[\frac{dy}{dx} = \frac{\dot \gamma_2(t)}{\dot \gamma_1(t)}  = - \frac{m(\gamma(t))}{n(\gamma(t))} = - \frac{m(x,y)}{n(x,y)}\]

Мы получили, что если у нас есть интегральная кривая $\gamma$ уравнения $F = 0$, то в локальных координатах они решают уравнение $y' = \frac{m(x,y)}{n(x,y)}$

Значит  интегральные кривые уравнения Пфаффа $m dx + n dy = 0$ локально совпадают с интегральными кривыми уравнения $y' = \frac{m(x,y)}{n(x,y)}$

Верно и обратное: пусть $y(x)$ -- решение уравнения $y' = - \frac{m}{n}, n(x,y(x)) \neq 0$

Как тогда получить из этого уравнения интегральную кривую уравнения Пфаффа?

Берем $\gamma_1(t) = x, \gamma_2(t) = y(x)$

\[ \dot \gamma_1(t) = 1, \dot \gamma_2(t) = \frac{dy}{dt} = \frac{dy}{dx} = - \frac{m(x,y)}{n(x,y)} = - \frac{m(\gamma(t))}{n(\gamma(t))}\]

Мы получили интегральную кривую уравнения Пфаффа.

Вывод: $F = m dx + n dy = 0$  -- запись совокупности двух обыкновенных дифференциальных уравнений:

\begin{equation*}
  \left[
    \begin{gathered}
      \frac{dy}{dx} = - \frac{m}{n}\\
      \frac{dx}{dy} = - \frac{n}{m}
    \end{gathered}
  \right.
\end{equation*}
  
  
\section{Уравнение в полных дифференциалах}

\begin{defn}
  Форма $F$ -- \se{Точная форма}{точная}, если $\exists U \in C^2(\mathbb{R}_{x,y}^2)$
  \[F = \frac{\partial U}{\partial x} dx + \frac{\partial U}{\partial y} dy  \]

  Если $F$ -- точная, то $F = 0$ называется \se{Уравнение полных дифференциалов}{уравнением полных дифференциалов}
\end{defn}

\begin{theorem}
  Если $F$ -- точная, то в окрестности произвольной точки $(x_0,y_0) \in G$ $U$ -- интеграл одного из уравнений:
  \[\frac{dy}{dx} = - \frac{m}{n} \text{ или } \frac{dx}{dx} = - \frac{n}{m}  \]
\end{theorem}
\begin{proof}
  $(x_0,y_0) \in  G$ можно считать, что $n(x_0,y_0) \neq 0$, тогда $n(x,y) \neq 0$ в некоторой окрестности

  Рассмотрим уравнение $y' =- \frac{m}{n}$

  Пусть $y(x)$ -- решение

  \[\frac{d}{dx}U(x,y(x)) = \frac{\partial U}{\partial x} + \frac{\partial U}{\partial y} \frac{dy}{dx}  = m + n \cdot(- \frac{m}{n}) \equiv 0  \]

  \[\frac{\partial U}{\partial y} = n \neq 0\]

  Получаем, что $U$ -- интеграл
  
\end{proof}

\subsection{Условие точности 1-формы}

\[U \in C^2 \Rightarrow \frac{\partial U}{\partial x}, \frac{\partial U}{\partial y} \in C^1\]

\[\frac{\partial m}{\partial y} = \frac{\partial^2 U}{\partial x \partial y}\]
\[\frac{\partial n}{\partial x} = \frac{\partial^2 U}{\partial y\partial x}\]
Из курса матанализа знаем, что если производные непрерывны, то они совпадают

\[F \text{ точна } \Rightarrow \frac{\partial m}{\partial y} = \frac{\partial n}{\partial x}\]

\begin{stat}
  \[G = (a,b) \times (\alpha, \beta)\]
  Тогда из равенства частных производных $m$ и $n$ следует, что $F$ -- точна
\end{stat}
\begin{proof}
  Фиксируем $(x_0,y_0) \in G$

  Хотим построить $U$
  \[\frac{\partial U}{\partial x} = m, \frac{\partial U}{\partial y} = n \]

  \[U = \int_{x_0}^{x}m(s,y)ds + \varphi (y) \text{ удовлетворяет первому уравнению}\]
  Нужно только найти $\varphi $

  \begin{equation*}
    \begin{gathered}
      \frac{\partial U}{\partial y} = \int_{x_0}^{x}\frac{\partial m}{\partial y}(s,y) ds + \varphi' (y) =
      \\
     =  \int_{x_0}^{x} \frac{\partial n}{\partial x}(s,y) ds + \varphi'(y) = n(x,y) - n(x_0,y) + \varphi'(y)
    \end{gathered}
  \end{equation*}
  Хотим
  \[n(x,y) = n(x,y) - n(x_0,y) + \varphi'(y)\]
  Тогда можно взять в качестве $\varphi(y) = \int_{y_0}^{y}n(x_0,t)dt$
  \[U(x,y) = \int_{x_0}^{x}m(s,y)ds + \int_{y_0}^{y}n(x_0,t)dt\]
\end{proof}
\begin{remark}
  Это утверждение верно не для любой области $G$, хотя верно, если $G$ -- звездчатое множество
\end{remark}


\subsection{Интегрирующий множитель}
\begin{defn}
  $\mu \in C^1, \mu \neq 0$ называется \se{Интегрирующий множитель}{интегрирующим множителем}, если $\mu F$ -- точная форма
\end{defn}

\begin{exl}
  Уравнение с разделяющимися переменными:

  \[m(x)n(y) dx + dy = 0\]
  Интегрирующий множитель -- $\frac{1}{n(y)}$
  \[m(x)dx + \frac{1}{n(y)} dy = 0\]

  \[ \frac{\partial m}{\partial y} = 0 = \frac{\partial }{\partial x}(\frac{1}{n(y)})\]

  И как мы уже видели интегралом будет

  \[U(x,y) = \int_{}^{}m(x) dx + \int_{}^{}\frac{1}{n(y)} dy \]
\end{exl}
  
\section{Системы дифференциальных уравнений}

Отныне независимая переменная будет обозначаться $t$ и искать мы будем функции $x(t)$

\begin{defn}
  \se{Система дифференциальных уравнений общего вида}{Системы дифференциальных уравенний общего вида} (системы разрешимые относительно старших производных)

  $n$ и $m_1, \ldots , m_n$ -- фиксированные натуральные числа

  Для каждого $i = 1,\ldots, n$ имеем уравнение

  \[\frac{d^mx}{dt^m} = f_i(t,x_1, \dot x_1, \ldots, \frac{d^{m_1-1}}{dt^{m_1-1}}, \ldots , x_n, \dot x_n, \ldots, \frac{d^{m_n-1}}{dt^{m_n-1}})  \]
  $m = \sum_{}^{}m_i$ называется \se{Порядок системы}{порядком системы}
\end{defn}

\subsection{Частные случаи}

\begin{itemize}
\item \se{Нормальная система}{Нормальная система}
  Ищем $x_1(t), \ldots, x_n(t)$, все $m_i = 1$
  \begin{equation*}
      \dot x_i(t) = f_i(t,x_1, \ldots ,x_n)
  \end{equation*}


\item \se{Дифференциальное уравнение порядка $m$}{Дифференциальное уравнение порядка $m$}
  $x(t)$ -- искомая функция

  \[\frac{d^mx}{dt^m} = f(t,x, \dot x, \ldots, \frac{d^{m-1}x}{dt^{m-1}})\]
\end{itemize}

Системы общего вида всегда сводятся к нормальным системам

Покажем, что дифференциальное уравнение сводится к нормальной системе

\begin{equation*}
  \begin{aligned}
    \begin{cases}
      \dot y_1 = y_2 &\\  
      \dot y_2 = y_3 & \\ 
      \vdots & \Longleftrightarrow  \frac{d^mx}{dt^m} = f(t,x, \dot x, \ldots, \frac{d^{m-1}x}{dt^{m-1}}) \\        
      \dot y_{m-1} = y_m & \\ 
      y_m = f(t,y_1,\ldots, y_{m-1}) & 
    \end{cases}
  \end{aligned}
\end{equation*}

Если $x$ решение уравнения, то очевидно, что $y_1 = x, y_2 = \dot x, \ldots y_m =  \frac{d^{m-1}x}{dt^{m-1}}$ решения системы и наоборот, если $y_1, y_2, \ldots y_m$ решения системы, то $x = y_1$ решение уравнения.

\subsection{Векторная запись нормальных систем}

Сейчас мы введем некоторые обозначения и соглашения, с которыми будем работать в дальнейшем

\begin{equation*}
  \begin{cases}
    \dot x_1 = f_1(t,x_1, \ldots, x_n)\\
    \vdots \\
    \dot x_n = f_n(t,x_1, \ldots, x_n)
  \end{cases}
\end{equation*}

\begin{equation*}
  \text{Вектор } x = 
  \begin{pmatrix}
    x_1 \\
    \vdots \\
    x_n 
  \end{pmatrix}
  \in \mathbb{R}^n, \
  \dot x = 
  \begin{pmatrix}
    \dot x_1 \\
    \vdots \\
    \dot x_n
  \end{pmatrix}
\end{equation*}

\begin{equation*}
  \text{Векторная функция } f(t,x) = 
  \begin{pmatrix}
    f_1(t,x) \\
    \vdots \\
    f_n(t,x) 
  \end{pmatrix}
\end{equation*}



Тогда исходная система принимает вид
\[\dot x = f(t,x)\]

\[ \text{Для функции $f(t)$ под записью } \int_{}^{}f(t) dt \text{ будем подразумевать }
  \begin{pmatrix}
    \int_{}^{}f_1(t)dt \\
    \vdots \\
    \int_{}^{}f_n(t) dt
  \end{pmatrix}\]


В качестве нормы на $\mathbb{R}^n$ зафиксируем $||x|| = \max\limits_{1 \leqslant i \leqslant n}^{}|x_i|$

\begin{defn}
  Для уравнения $\dot x = f(t,x), x \in \mathbb{R}^n$ ($f \in C(G) G \subset \mathbb{R}^{n+1}_{t,x}$) функция $x : (a,b) \to \mathbb{R}^n$ называется \cursed{решением}, если
  \begin{itemize}
  \item $\exists \dot x $ на $(a,b)$
  \item $(t,x(t)) \in G$
  \item $\dot x(t) = f(t,x(t)), t \in (a,b)$
  \end{itemize}
\end{defn}

\begin{defn}
  $x : (a,b) \to \mathbb{R}^n $ называется решением задачи Коши с начальным условием $(t_0,x_0)$, если
  \begin{itemize}
  \item $x$ -- решение
  \item $x(t_0) = x_0$
  \end{itemize}
\end{defn}

\newpage
\hypertarget{dex}
    \printindex  

\end{document}

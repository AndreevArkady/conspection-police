\documentclass[a4paper]{article}

\usepackage[12pt]{extsizes}
\usepackage[utf8]{inputenc}
\usepackage[unicode, pdftex]{hyperref}
\usepackage{cmap}
\usepackage{mathtext}
\usepackage{multicol}
\setlength{\columnsep}{1cm}
\usepackage[T2A]{fontenc}
\usepackage[english,russian]{babel}
\usepackage{amsmath,amsfonts,amssymb,amsthm,mathtools}
\usepackage{icomma}
\usepackage{euscript}
\usepackage{mathrsfs}
\usepackage[dvipsnames]{xcolor}
\usepackage[left=2cm,right=2cm,
    top=2cm,bottom=2cm,bindingoffset=0cm]{geometry}
\usepackage[normalem]{ulem}
\usepackage{graphicx}
\usepackage{makeidx}
\makeindex
\graphicspath{{pictures/}}
\DeclareGraphicsExtensions{.pdf,.png,.jpg}
%\usepackage[usenames]{color}
\hypersetup{
     colorlinks=true,
     linkcolor=brilliantrose,
     filecolor=brilliantrose,
     citecolor=black,      
     urlcolor=brilliantrose,
     }
\usepackage{fancyhdr}
\pagestyle{fancy} 
\fancyhead{} 
\fancyhead[LE,RO]{\thepage} 
\fancyhead[CO]{\hyperlink{uk}{к списку объектов}}
\fancyhead[LO]{\hyperlink{sod}{к содержанию}} 
\fancyfoot{}
\newtheoremstyle{indented}{0 pt}{0 pt}{\itshape}{}{\bfseries}{. }{0 em}{ }

\renewcommand\thesection{}
\renewcommand\thesubsection{}

%\geometry{verbose,a4paper,tmargin=2cm,bmargin=2cm,lmargin=2.5cm,rmargin=1.5cm}

\title{Дифференциальные уравнения \\ 
и динамические системы}
\author{Алёшин Артём \\ 
    на основе лекций Пилюгина С. Ю. \\
    под редакцией @keba4ok}
\date{5 сентября 2021.}


%envirnoments
    \theoremstyle{indented}
    \newtheorem*{theorem}{Теорема}
    \newtheorem*{lemma}{Лемма}
    \newtheorem*{alg}{Алгоритм}

    \theoremstyle{definition} 
    \newtheorem*{defn}{Определение}
    \newtheorem*{exl}{Пример(ы)}
    \newtheorem*{prob}{Задача}

    \theoremstyle{remark} 
    \newtheorem*{remark}{Примечание}
    \newtheorem*{cons}{Следствие}
    \newtheorem*{exer}{Упражнение}
    \newtheorem*{stat}{Утверждение}
%esli ne hochetsa numeracii - nuzhno prisunut' zvezdochku-pezsochku

\definecolor{brilliantrose}{rgb}{1.0, 0.33, 0.64}

%declarations
        %arrows_shorten
            \DeclareMathOperator{\la}{\leftarrow}
            \DeclareMathOperator{\ra}{\rightarrow}
            \DeclareMathOperator{\lra}{\leftrightarrow}
            \DeclareMathOperator{\llra}{\longleftrightarrow}
            \DeclareMathOperator{\La}{\Leftarrow}
            \DeclareMathOperator{\Ra}{\Rightarrow}
            \DeclareMathOperator{\Lra}{\Leftrightarrow}
            \DeclareMathOperator{\Llra}{\Longleftrightarrow}

        %letters_different
            \DeclareMathOperator{\CC}{\mathbb{C}}
            \DeclareMathOperator{\ZZ}{\mathbb{Z}}
            \DeclareMathOperator{\RR}{\mathbb{R}}
            \DeclareMathOperator{\NN}{\mathbb{N}}
            \DeclareMathOperator{\HH}{\mathbb{H}}
            \DeclareMathOperator{\LL}{\mathscr{L}}
            \DeclareMathOperator{\KK}{\mathscr{K}}
            \DeclareMathOperator{\GA}{\mathfrak{A}}
            \DeclareMathOperator{\GB}{\mathfrak{B}}
            \DeclareMathOperator{\GC}{\mathfrak{C}}
            \DeclareMathOperator{\GD}{\mathfrak{D}}
            \DeclareMathOperator{\GN}{\mathfrak{N}}
            \DeclareMathOperator{\Rho}{\mathcal{P}}
            \DeclareMathOperator{\FF}{\mathcal{F}}

        %common_shit
            \DeclareMathOperator{\Ker}{Ker}
            \DeclareMathOperator{\Frac}{Frac}
            \DeclareMathOperator{\Imf}{Im}
            \DeclareMathOperator{\cont}{cont}
            \DeclareMathOperator{\id}{id}
            \DeclareMathOperator{\ev}{ev}
            \DeclareMathOperator{\lcm}{lcm}
            \DeclareMathOperator{\chard}{char}
            \DeclareMathOperator{\codim}{codim}
            \DeclareMathOperator{\rank}{rank}
            \DeclareMathOperator{\ord}{ord}
            \DeclareMathOperator{\End}{End}
            \DeclareMathOperator{\Ann}{Ann}
            \DeclareMathOperator{\Real}{Re}
            \DeclareMathOperator{\Res}{Res}
            \DeclareMathOperator{\Rad}{Rad}
            \DeclareMathOperator{\disc}{disc}
            \DeclareMathOperator{\rk}{rk}
            \DeclareMathOperator{\const}{const}
            \DeclareMathOperator{\grad}{grad}
            \DeclareMathOperator{\Aff}{Aff}
            \DeclareMathOperator{\Lin}{Lin}
            \DeclareMathOperator{\Prf}{Pr}
            \DeclareMathOperator{\Iso}{Iso}

        %specific_shit
            \DeclareMathOperator{\Tors}{Tors}
            \DeclareMathOperator{\form}{Form}
            \DeclareMathOperator{\Pred}{Pred}
            \DeclareMathOperator{\Func}{Func}
            \DeclareMathOperator{\Const}{Const}
            \DeclareMathOperator{\arity}{arity}
            \DeclareMathOperator{\Aut}{Aut}
            \DeclareMathOperator{\Var}{Var}
            \DeclareMathOperator{\Term}{Term}
            \DeclareMathOperator{\sub}{sub}
            \DeclareMathOperator{\Sub}{Sub}
            \DeclareMathOperator{\Atom}{Atom}
            \DeclareMathOperator{\FV}{FV}
            \DeclareMathOperator{\Sent}{Sent}
            \DeclareMathOperator{\Th}{Th}
            \DeclareMathOperator{\supp}{supp}
            \DeclareMathOperator{\Eq}{Eq}
            \DeclareMathOperator{\Prop}{Prop}


%env_shortens_from_hirsh            
    \newcommand{\bex}{\begin{example}\rm}
    \newcommand{\eex}{\end{example}}
    \newcommand{\ba}{\begin{algorithm}\rm}
    \newcommand{\ea}{\end{algorithm}}
    \newcommand{\bea}{\begin{eqnarray*}}
    \newcommand{\eea}{\end{eqnarray*}}
    \newcommand{\be}{\begin{eqnarray}}
    \newcommand{\ee}{\end{eqnarray}}
    \newcommand{\abs}[1]{\lvert#1\rvert}
        \newcommand{\bp}{\begin{prob}}
        \newcommand{\ep}{\end{prob}}
    
\begin{document}
%ya_ebanutyi
\newcommand{\resetexlcounters}{%
  \setcounter{exl}{0}%
} 
\newcommand{\resetremarkcounters}{%
  \setcounter{remark}{0}%
} 
\newcommand{\reseconscounters}{%
  \setcounter{cons}{0}%
} 
\newcommand{\resetall}{%
    \resetexlcounters
    \resetremarkcounters
    \reseconscounters%
}
\newcommand{\cursed}[1]{\textit{\textcolor{brilliantrose}{#1}}}
\newcommand{\de}[3][2]{\index{#2}{\textbf{\textcolor{brilliantrose}{#3}}}}
\newcommand{\re}[3][2]{\hypertarget{#2}{\textbf{\textcolor{brilliantrose}{#3}}}}
\newcommand{\se}[3][2]{\index{#2}{\textit{\textcolor{brilliantrose}{#3}}}}
\maketitle 
\newpage
\hypertarget{sod}
\tableofcontents
\newpage

\subsection{Литература}

\begin{itemize}
    \item В. И. Арнольд Обыкновенные дифференциальные уравнения
    
    \item Ю. Н. Бибиков Общий курс дифференциальных уравнения
    \item С. Ю. Пилюгин Пространства динамических систем
\end{itemize}   

\begin{defn}
  \se{Дифференциальное уравнение}{Дифференциальное уравнение} -- уравнение от неизвествной фукции $y(x)$, где $x \in \RR $ -- независимая переменная, вида
    \[f(x,y,y',\ldots,y^{(n)}) = 0\]
\end{defn}

\section{Дифференциальные уравнения 1-го порядка, разрешенные относительно производной}

\begin{defn}
  \se{Дифференциальное уравнение!1-го порядка}{Дифференциальное уравнение 1-го порядка}, разрешенное относительно производной -- уравнение вида $y' = f(x,y), f \in C(G)$, где $G$ -- область (открытое связное множество) в $\RR_{x,y}^2$
\end{defn}

\begin{defn}
  $y : (a,b) \to \RR$ -- \se{Решение дифференциального уравнения}{решение} на $(a,b)$, если  

  \begin{itemize}
    \item $y$ -- дифференцируема;
    \item $(x,(y(x)) \in G, x \in (a,b)$;
    \item $y'(x) \equiv f(x,y(x))$ на $(a,b)$.
  \end{itemize} 

\end{defn}

\begin{exl} \ 

  \begin{itemize}
    \item $y' = ky, k > 0 G = \RR^2$;
    \item $\forall c \in \RR \  y(x) = ce^{kx}$ -- решение на $\RR$.
  \end{itemize}
\end{exl}

\begin{defn}
  \se{Интегральная кривая}{Интегральная кривая} -- график решения.
\end{defn}

\subsection{Задача Коши}

\begin{defn}
$y(x)$ -- решение \se{Задача Коши}{задачи Коши} с начальным условем $(x_0,y_0)$, если

\begin{itemize}
    \item $y(x)$ -- решение дифференциального уравнения на $(a,b)$;
    \item $y(x_0) = y_0$.
\end{itemize}
\end{defn}

\subsection{Единственность}
\begin{defn}
  $(x_0,y_0)$ -- \se{Точка единственности}{точка единственности} для задачи Коши, если $\forall y_1, y_2$ -- решения $\exists (\alpha, \beta) \ni x_0: y_1|_{(\alpha,\beta)} = y_2|_{(\alpha,\beta)}$.
\end{defn}

\begin{exl}
  \[y' = 3 \sqrt[3]{y^2}\]

  Если $(x_0,y_0) = 0$, то возможны следующие решения:

  \begin{itemize}
  \item \[y_1 = 0\]
  \item \[y_2 = \begin{cases}
        0 & x \leqslant 0 \\
        x^3 & x > 0
      \end{cases}\]
  \item \[y_3 = \begin{cases}
        x^3 & x \leqslant 0 \\
        0 & x > 0
      \end{cases}\]
  \end{itemize}
  Точка $(0,0)$ не является точкой единственности, но при этом $(1,1)$ уже будет точкой единственности
\end{exl}

\subsection{Поле направлений}

\begin{defn}
  Из уравнения $y' = f(x,y)$ мы можем вычислить \se{Коэффициент наклона}{коэффициент наклона} в каждой точке $(x,y)$
  \[k = y'(x) = f(x,y)\]
  Если в каждой точке $(x,y)$ области $G$ провести отрезок с угловым коэффициентом равным $f(x,y)$, то получится \se{Поле направлений}{поле направлений}. Любая интегральная кривая в каждой своей точке касается соответствующего отрезка.
\end{defn}

\subsection{Основные теоремы}

\begin{theorem}[\cursed{О существовании}]

  Если $y' = f(x,y), \ f \in C(G)$, то $\forall (x_0,y_0) \in G \ \exists $ решение задачи Коши с начальными данными $(x_0,y_0)$

  $G$ называется \se{Область!существования}{областью существования}.

\end{theorem} \ 

\begin{theorem}[\cursed{О единственности}]

  Если $y' = f(x,y), \ f, \frac{\partial f}{\partial y} \in C(G)$, то $\forall (x_0,y_0) \in G \ \exists $ единственное решение задачи Коши с начальными данными $(x_0,y_0)$

  $G$ называется \se{Область!единственности}{областью единственности}.
\end{theorem}

\section{Интегрируемые типы дифференциальных уравнений 1-го порядка}

\begin{exl}
  $y' = f(x)$ -- из анализа знаем, что единнственным решение при данном условии $(x_0,y_0)$ будет \[y(x) = y_0 + \int_{x_0}^xf(t)dt\]
\end{exl}

\subsection{Интеграл}

Пусть $H \subset G$ -- область
\begin{defn}
  Функция $U \in C^1(H,\mathbb{R})$ называется \se{Интеграл уравнения}{интегралом уравнения} $y' = f(x,y)$ в $H$, если выполнены следующие условия:
  \begin{itemize}
  \item $\frac{\partial U}{\partial y} \not = 0$;
  \item если $y(x), x \in (a,b)$ -- решение с $(x,y(x)) \in H$, то $U(x,y(x)) = \const$.
  \end{itemize}
\end{defn}

\begin{theorem}[Напоминание \cursed{теоремы о неявной функции}]

  \[F : H \subset \mathbb{R}^2 \to \mathbb{R}, F \in C^1  \]
  Если
  \begin{itemize}
    \item
      \[F(x_0,y_0) = 0\]
    \item
      \[ \frac{\partial F}{\partial y}\bigg|_{(x_0,y_0)} \not  = 0\]
  \end{itemize}

  тогда $\exists I, J$ -- открытые интервалы $x_0 \in I, y_0 \in J$, $\exists z(x) \in C^1(I)$ такая, что
  \begin{itemize}
  \item $z(x_0) = y_0$;
  \item $F(x,y) = 0 \leftrightarrow y = z(x)$ при $(x,y) \in I \times J$.
  \end{itemize}
\end{theorem} \ 

\begin{theorem}[\se{Теорема!об интеграле для дифференциальных уравнений первого порядка}{Об интеграле для дифференциальных уравнений первого порядка}]

  Пусть $U$ -- интеграл $y' = f(x,y)$ в $H \subset G$. Тогда $\forall (x_0,y_0) \in H \ \exists H_0 \subset H, H_0 = I \times J \ni (x_0,y_0)$ и $\exists y(x) \in C^1(I)$ такая что:

  \begin{itemize}
    \item $y(x)$ -- решение задачи Коши с начальными данными $(x_0,y_0)$
    \item $(x,y) \in  H$ и $U(x,y) = U(x_0,y_0) \Rightarrow y = y(x)$
  \end{itemize}

\end{theorem}

\begin{proof} \ 

  Фиксируем произвольную точку $(x_0,y_0)$. Рассмотрим $F(x,y) = U(x,y) - U(x_0,y_0)$.

  $F$ удовлетворяет условию теоремы о неявной функции, так как $\frac{\partial F}{\partial y} = \frac{\partial U}{\partial y} \not  = 0$, поэтому существуют $I_0, J_0 \ I_0 \times J_0 \subset H$ и $\exists y(x) \in  C^1(I_0), \ y(x_0) = y_0$.

  По теореме существования $\exists $ решение $z(x)$ задачи Коши с начальными условиями $(x_0,y_0)$ на некотором промежутке $I \ni x_0$ такое что $(x,z(x)) \in I_0 \times J_0$.

  Тогда по определению интеграла $U(x,z(x)) = \const$ $\Rightarrow F(x,z(x)) = 0 \Rightarrow z(x) = y(x)$.
\end{proof}

\subsection{Дифференциальные уравнения с разделяющимися переменными}

\begin{equation*}
  \begin{aligned}
    & y' = m(x) \cdot n(y) \\
    & m \in C((a,b)), n \in C((\alpha, \beta)) \\
    & G = (a,b) \times (\alpha, \beta)
  \end{aligned}
\end{equation*}

\begin{itemize}
\item $y_0 \in (\alpha, \beta) n(y_0) = 0 \Rightarrow y \equiv y_0$
  
  Проверяется подставнкой
\item $I \subset (\alpha, \beta), n(y) \not  = 0$ при $y \in  I$

  Подсказка:

  Рассмотрим $y(x) : (x,y(x)) \in (a,b) \times I$ и отличную от $0$

  $y' = m(x) n(y)$, на $n(y)$ можно поделить

  \[\frac{y'}{n(y(x))} = m(x), \:
  \int\limits_{x_0}^x\frac{y'(t)dt}{n(y(t))} = \int\limits_{x_0}^x m(t) dt. \]


    Замена $z = y(t)$

  \[\int\limits_{y(x_0)}^{y(x)}\frac{dz}{n(z)} = \int\limits_{x_0}^x m(t) dt,
  \]

 
    Обозначим за $N(y)$ и $M(x)$ некоторые первообразные $\frac{1}{n(y)}$ и $m(x)$ соответственно

\begin{equation*}
  \begin{aligned}
    N(y(x)) - N(y(x_0)) & = M(x) - M(x_0)\\
    U(x,y) :& = N(y) - M(x).
  \end{aligned}
\end{equation*}


Если $y(x)$ -- решение, то $U(x,y(x)) = N(y(x_0)) - M(x_0)$

  \[\frac{\partial U}{\partial y} = \frac{1}{n(y)} \not  = 0. \]

Поэтому $U(x,y)$ -- интеграл.
  
\end{itemize}
  
\newpage
\hypertarget{dex}
    \printindex


\end{document}
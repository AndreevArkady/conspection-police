\documentclass[a4paper,100pt]{article}

\usepackage[utf8]{inputenc}
\usepackage[unicode, pdftex]{hyperref}
\usepackage{cmap}
\usepackage{mathtext}
\usepackage{multicol}
\setlength{\columnsep}{1cm}
\usepackage[T2A]{fontenc}
\usepackage[english,russian]{babel}
\usepackage{amsmath,amsfonts,amssymb,amsthm,mathtools}
\usepackage{icomma}
\usepackage{euscript}
\usepackage{mathrsfs}
\usepackage{geometry}
\usepackage[usenames]{color}
\hypersetup{
     colorlinks=true,
     linkcolor=magenta,
     filecolor=green,
     citecolor=black,      
     urlcolor=cyan,
     }
\usepackage{fancyhdr}
\pagestyle{fancy} 
\fancyhead{} 
\fancyhead[LE,RO]{\thepage} 
\fancyhead[CO]{\hyperlink{t2}{к списку объектов}}
\fancyhead[LO]{\hyperlink{t1}{к содержанию}} 
\fancyhead[CE]{текст-центр-четные} 
\fancyfoot{}
\newtheoremstyle{indented}{0 pt}{0 pt}{\itshape}{}{\bfseries}{. }{0 em}{ }

%\geometry{verbose,a4paper,tmargin=2cm,bmargin=2cm,lmargin=2.5cm,rmargin=1.5cm}

\title{Первая контрольная 2 сем.}
\author{Кабашный Иван (@keba4ok)\\ \\ (по материалам лекций Белова Ю. С.,\\ а также других источников)}
\date{31 марта 2021 г.}

\theoremstyle{indented}
\newtheorem{theorem}{Теорема}
\newtheorem{lemma}{Лемма}

\theoremstyle{definition} 
\newtheorem{defn}{Определение}
\newtheorem{exl}{Пример(ы)}

\theoremstyle{remark} 
\newtheorem{remark}{Примечание}
\newtheorem{cons}{Следствие}
\newtheorem{stat}{Утверждение}

\DeclareMathOperator{\Ker}{Ker}
\DeclareMathOperator{\Tors}{Tors}
\DeclareMathOperator{\Frac}{Frac}
\DeclareMathOperator{\Imf}{Im}
\DeclareMathOperator{\Real}{Re}
\DeclareMathOperator{\cont}{cont}
\DeclareMathOperator{\id}{id}
\DeclareMathOperator{\ev}{ev}
\DeclareMathOperator{\lcm}{lcm}
\DeclareMathOperator{\chard}{char}
\DeclareMathOperator{\CC}{\mathbb{C}}
\DeclareMathOperator{\ZZ}{\mathbb{Z}}
\DeclareMathOperator{\RR}{\mathbb{R}}
\DeclareMathOperator{\NN}{\mathbb{N}}
\DeclareMathOperator{\PP}{\mathbb{P}}
\DeclareMathOperator{\FF}{\mathcal{F}}
\DeclareMathOperator{\Rho}{\mathcal{P}}
\DeclareMathOperator{\codim}{codim}
\DeclareMathOperator{\rank}{rank}
\DeclareMathOperator{\ord}{ord}
\DeclareMathOperator{\adj}{adj}
\DeclareMathOperator{\const}{const}
\DeclareMathOperator{\grad}{grad}

\begin{document}

\newcommand{\resetexlcounters}{%
  \setcounter{exl}{0}%
} 

\newcommand{\resetremarkcounters}{%
  \setcounter{remark}{0}%
} 

\newcommand{\reseconscounters}{%
  \setcounter{cons}{0}%
} 

\newcommand{\resetall}{%
    \resetexlcounters
    \resetremarkcounters
    \reseconscounters%
}

\maketitle 

\newpage

\hypertarget{t1}{Темы} задач. Кстати, конкретно про асимптотику суммы тут ничего нет, там всё-таки как-то трудно пером общий случай описать, ну и про определённые интегралы - тоже пусто, из фактов только таблица первообразных и интегрирование по частям, думаю уж, упустить можно. Пока есть время - можете предложить, что стоит добавить помимо всего, что есть.
\tableofcontents

\newpage

\section{Несобственный интеграл. \\
Равномерная сходимость интегралов.}

\begin{defn}
    Пусть функция $x \mapsto f(x)$ определена на промежутке $[a, \omega)$ (где $\omega$ может быть действительным числом или $+\infty$) и интегрируема на любом промежутке $[a, b]$, содержащемся в этом промежутке. Величина 
    \[
        \int_a^\omega f(x)dx:=\lim_{b \rightarrow \omega} \int_a^b f(x)dx, 
    \] 
    если указанный предел существует, называется \textit{несобственным интегралом} от функции $f$ по промежутку $[a, \omega)$. Если указанный предел существует, то говорят, что интеграл \textit{сходится}, и \textit{расходится} в противном случае.
\end{defn}

\begin{stat}
    (Критерий Коши сходимости несобственного интеграла). Если функция $x\mapsto f(x)$ определена на промежутке $[a, \omega)$ и интегрируема на любом отрезке $[a, b]$, лежащем внутри, то интеграл $\int_a^\omega f(x)dx$ сходится тогда и только тогда, когда для любого $\varepsilon>0$ можно указать $B\in [a, \omega)$ так, что при любых $b_1, b_2\in [a, \omega)$ таких, что $B<b_1, b_2$ имеет место соотношение 
    \[
        \bigg| \int_{b_1}^{b_2}f(x)dx\bigg| < \varepsilon. 
    \]
\end{stat}

\begin{defn}
    Про несобственный интеграл $\int_a^\omega f(x)dx$ говорят, что он \textit{сходится абсолютно}, если сходится интеграл $\int_a^\omega |f|(x)dx$. 
\end{defn}

\begin{remark}
    Нетрудно заметить, что если интеграл сходится абсолютно, то он сходится.
\end{remark} 

\begin{stat}
    Если функция $f$ удовлетворяет условиям первого определения и неотрицательна на $[a, \omega)$, то несобственный интеграл существует в том и только том случае, когда функция 
    \[
        \mathcal{F}(b)=\int_a^b f(x) dx
    \]
    ограничена на $[a, \omega)$.
\end{stat}

\begin{theorem}
    (Теорема сравнения). Пусть функции $f(x)$ и $g(x)$ определены на промежутке $[a, \omega)$ и интегрируемы на любом его отрезке. Если на данном промежутке выполнено $0\leq f(x) \leq g(x)$, то из сходимости несобственного интеграла по $g$ следует сходимость несобственного интеграла по $f$, и из расходимости несобственного интеграла f следует расходимость несобственного интеграла g. 
\end{theorem}

\begin{remark}
    Если к функциям из теоремы выше можно добавить такое условие: существуют две положительные константы $c_1$, $c_2$, что $c_1f(x)\leq g(x)\leq c_2 f(x)$, то с учётом линейности несобственного интеграла, можно заключит, что интегралы по функциям $f$ и $g$ сходятся или расходятся одновременно.
\end{remark}

\begin{defn}
    Если несобственный интеграл сходится, но не абсолютно, то говорят, что он \textit{сходится условно}.
\end{defn}

\begin{stat}
    (Признак Абеля-Дирихле сходимости интеграла). Пусть $x\mapsto f(x)$, $x\mapsto g(x)$ - функции, определённые на промежутке $[a, \omega)$ и интегрируемые на любом его отрезке. Пусть $g$ - монотонная функция. \ 

    Тогда для сходимости несобственного интеграла
    \[
        \int_a^\omega (f\cdot g)(x)dx
    \]
    достаточно, чтобы выполнилась либо пара условий

    \begin{itemize}
        \item интеграл $\int_a^\omega f(x)dx$ сходится;
        \item функция $g$ ограничена на $[a, \omega)$,
    \end{itemize}

    либо пара условий

    \begin{itemize}
        \item функция $\mathcal{F}=\int_a^b f(x)dx$ ограничена на $[a, \omega)$; 
        \item функция $g(x)$ стремится к нулю при $x\rightarrow \omega$, $x\in [a, \omega)$.
    \end{itemize}
\end{stat}

\begin{defn}
    Несобственный параметрический интеграл $\int_a^\omega f(x, \alpha)dx$ называется \textit{равномерно сходящимся} на множестве $E$, если
    \[
        \forall \varepsilon >0 \: \exists t\in (a, \omega) \: \forall \xi \in [t, \omega) \: \forall \alpha \in E \mapsto \bigg| \int_\xi^\omega f(x, \alpha)dx\bigg|<\varepsilon.
    \]
\end{defn}

\begin{theorem}
    (Признак Вейерштрасса). Если подынтегральная функция в параметрическом интеграле может быть ограничена функцией одной переменной сверху, и интеграл от данной функции сходится, то и изначальный интеграл сходится.
\end{theorem} \

\begin{theorem}
    (Признак Дирихле). Достаточное условие равномерной сходимости интеграла вида $\int_a^{+\infty}f(x)g(x)dx$. Если выполнены следующие условия: 

    \begin{itemize}
        \item первообразная $f(x)$ ограничена; 
        \item $g(x)$ дифференцируема, больше нуля, её производная отрицательна; 
        \item $\lim_{x\rightarrow +\infty} g(x)=0$. 
    \end{itemize}

    Тогда $\int_a^{+\infty}f(x)g(x)dx$ сходится.
\end{theorem} \ 

\begin{theorem}
    (Критерий Коши). Параметрический интеграл сходится тогда и только тогда, когда выполнено следующее условие:
    \[
        \forall \varepsilon >0 \: \exists t\in (a, \omega): \: \forall \xi_1, \xi_2 \in [t, \omega) \: \forall \alpha \in E \mapsto \bigg| \int_{\xi_1}^{\xi_2} f(x, \alpha)dx\bigg|<\varepsilon.
    \]
\end{theorem}

\section{Кривые.}

\begin{defn}
    \textit{Кривые в $\RR^n$} - непрерывное отображение $f:[a, b]\rightarrow \RR^n$.
\end{defn}

\begin{stat}
    (Длина кривой). Если непрерывная кривая $\gamma$ задана параметрически: $x_1=f_1(t)$, $x_2=f_2(t)$, \dots, то можно найти её длину:
    \[
        \int_a^b\sqrt{(f_1')^2(x)+\ldots+(f_n')^2(x)}dx. 
    \]
    Естественно, все $f_i$ должны быть дифференцируемы на нужном промежутке.
\end{stat}

\begin{remark}
    Ну а площадь - сам бог велел использовать интегралы (наверное).
\end{remark}

\section{Многомерные функции.}

\begin{defn}
    $f$ \textit{дифференцируема} в точке $(x_1, \ldots, x_m)$, если $f(y)=f(x)+L(y-x)+o(||x-y||)$, где $L$ - линейное отображение $\RR^m\rightarrow \RR$, причём однородное, то есть, $L(0)=0$.
\end{defn}

\begin{defn}
    Это линейное отображение $L$ называется \textit{дифференциалом} в точке $x$.
\end{defn}

\begin{defn}
    \textit{Частная производная}. Пусть имеется $f:\RR\rightarrow \RR$, $f(x_1, \ldots, x_n)$, $x^0=(x_1^0, \ldots, x_n^0)$. Тогда частная производная по $x_k $, $f(x_1^0, \ldots, x_{k-1}^0, x, x_{k+1}^0, \ldots, x_m^0)=g(x), g'(x_k^0)$. $\frac{\partial f}{\partial x_k}\bigg|_{x^0}:=g'(x_k^0)=\lim_{\varepsilon\rightarrow 0}\frac{f(\ldots, x_k^0+\varepsilon, \ldots)-f(\ldots)}{\varepsilon}$. 
\end{defn}

\begin{defn}
    \textit{Производная по направлению}. Пусть направление задаётся $e\in\RR^n$, $||e||=1$, $f$ - дифференцируема по направлению $e$, если $g(t)=f(x^0+te)$, $t\in\RR$ и существует $g'(0)$, то производная по направлению $e$ - $g'(0)=\lim_{t\rightarrow 0}\frac{f(x^0+te)-f(x^0)}{t}$. 
\end{defn}

\begin{theorem}
    $f: G\rightarrow \RR^m$, $G\subset \RR^n$ - открытое, $f$ - гладкая в окрестности $x^0$ (верхние индексы), $y^0=f(x^0)$, $g:V_{f(x^0)}\rightarrow \RR^k$, гладкая в $f(x^0)$, для $f$ и $g$ существуют линейные операторы $A$ ($x_0)$ и $B$ ($f(x_0)$). Тогда $g(f(x))$ - гладкое отображение в $x_0$ с линейным оператором $BA:\RR^n\rightarrow \RR^k$. 
\end{theorem} \

\begin{theorem}
    Пусть у нас есть отображение $f:\RR^n\rightarrow \RR^m$, $V_{x_0}\rightarrow \RR^m$, причём существуют все частные производные в $V_{x^0}$ и они непрерывные в $x^0$. Тогда $f$ дифференцируема в точке $x^0$. 
\end{theorem} \ 

\begin{theorem}
    Пусть $f:\RR^n\rightarrow \RR$, $f$ - гладкая на $G$ - открытое множестве, причём частные производные сущуствуют и непрерывны в каждой точке (условно говоря, $f$ гладкая). Предположим, что точка $x^0$ - локальный максимум или минимум. Тогда $\grad f|_{x^0}\equiv 0$.
\end{theorem} \

\begin{theorem}
    Если функция $E\rightarrow \RR$, определённая на множестве $E\subset \RR^m$, дифференцируема во внутренней точке $x\in E$ этого множества, то в этой точке функция имеет все частные производные по каждой переменной и дифференциал функции однозначно определяется этими частными производными в виде
    \[
      df(x)h=\frac{\partial f}{\partial x_1}(x)h_1+\ldots+\frac{\partial f}{\partial x_m}(x)h_m. 
    \]
\end{theorem}

\section{Дифференцирование.}

\begin{theorem}
    Если отображение $f_1: E \rightarrow \RR^n$, $f_2:E \rightarrow \RR^n$, определённые на множестве $E\subset \RR^m$, дифференцируемы в точке $x\in E$, то их линейная комбинация также является дифференцируемым в этой точке отображением, причём имеет место равенство
    \[
        (\lambda_1 f_1 + \lambda_2 f_2)'(x) = (\lambda_1 f_1'+\lambda_2 f_2')(x). 
    \]
\end{theorem}

\begin{remark}
    Про произведени и частное - тоже точно так же, как и в случае с одномерными функциями.
\end{remark}

\begin{defn}
    Пусть $x\in \RR^m$. Через $T\RR_x^m$ обозначим совокупность векторов, приложенных к точке $x\in \RR^m$. Это векторное пространство называют \textit{касательным пространством} к $\RR^m$ в точке $x$. 
\end{defn}

\begin{theorem}
    (Дифференцирование композиции). Если отображение $f: X \rightarrow Y$ множества $x\subset \RR^m$ в множество $y\subset \RR^n$ дифференцируемо в точке $x\in X$, а отображение $g: Y \rightarrow \RR^k$ дифференцируемо в точке $y=f(x)\in Y$, то композиция $g\circ f: X \rightarrow \RR^k$ этих отображений дифференцируема в точке $x$, причём дифференциал $d(g\circ f): T\RR_x^m \rightarrow T\RR_{g(f(x))}^k$ композиции равен композиции дифференциалов.
\end{theorem}

\end{document}

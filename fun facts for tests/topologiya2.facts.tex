\documentclass[a4paper,100pt]{article}

\usepackage[utf8]{inputenc}
\usepackage[unicode, pdftex]{hyperref}
\usepackage{cmap}
\usepackage{mathtext}
\usepackage{multicol}
\setlength{\columnsep}{1cm}
\usepackage[T2A]{fontenc}
\usepackage[english,russian]{babel}
\usepackage{amsmath,amsfonts,amssymb,amsthm,mathtools}
\usepackage{icomma}
\usepackage{euscript}
\usepackage{mathrsfs}
\usepackage{geometry}
\usepackage[usenames]{color}
\hypersetup{
     colorlinks=true,
     linkcolor=magenta,
     filecolor=green,
     citecolor=black,      
     urlcolor=cyan,
     }
\usepackage{fancyhdr}
\pagestyle{fancy} 
\fancyhead{} 
\fancyhead[CO]{\hyperlink{t2}{к списку объектов}}
\fancyhead[LO]{\hyperlink{t1}{к содержанию}} 
\fancyfoot{}
\newtheoremstyle{indented}{0 pt}{0 pt}{\itshape}{}{\bfseries}{. }{0 em}{ }

%\geometry{verbose,a4paper,tmargin=2cm,bmargin=2cm,lmargin=2.5cm,rmargin=1.5cm}

\title{Геометрия и топология. Факты 2 сем.}
\author{Кабашный Иван (@keba4ok)\\ \\ (по материалам лекций Фоминых Е. А.,\\ практик, а также других источников)}
\date{26 марта 2021 г.}

\theoremstyle{indented}
\newtheorem{theorem}{Теорема}
\newtheorem{lemma}{Лемма}

\theoremstyle{definition} 
\newtheorem{defn}{Определение}
\newtheorem{exl}{Пример(ы)}

\theoremstyle{remark} 
\newtheorem{remark}{Примечание}
\newtheorem{cons}{Следствие}
\newtheorem{stat}{Утверждение}

\DeclareMathOperator{\la}{\leftarrow}
\DeclareMathOperator{\ra}{\rightarrow}
\DeclareMathOperator{\lra}{\leftrightarrow}
\DeclareMathOperator{\La}{\Leftarrow}
\DeclareMathOperator{\Ra}{\Rightarrow}
\DeclareMathOperator{\Lra}{\Leftrightarrow}
\DeclareMathOperator{\Llra}{\Longleftrightarrow}
\DeclareMathOperator{\Ker}{Ker}
\DeclareMathOperator{\Tors}{Tors}
\DeclareMathOperator{\Frac}{Frac}
\DeclareMathOperator{\Imf}{Im}
\DeclareMathOperator{\Real}{Re}
\DeclareMathOperator{\cont}{cont}
\DeclareMathOperator{\id}{id}
\DeclareMathOperator{\ev}{ev}
\DeclareMathOperator{\lcm}{lcm}
\DeclareMathOperator{\chard}{char}
\DeclareMathOperator{\CC}{\mathbb{C}}
\DeclareMathOperator{\ZZ}{\mathbb{Z}}
\DeclareMathOperator{\RR}{\mathbb{R}}
\DeclareMathOperator{\NN}{\mathbb{N}}
\DeclareMathOperator{\PP}{\mathbb{P}}
\DeclareMathOperator{\FF}{\mathcal{F}}
\DeclareMathOperator{\Rho}{\mathcal{P}}
\DeclareMathOperator{\codim}{codim}
\DeclareMathOperator{\rank}{rank}
\DeclareMathOperator{\ord}{ord}
\DeclareMathOperator{\adj}{adj}
\DeclareMathOperator{\const}{const}
\DeclareMathOperator{\grad}{grad}
\DeclareMathOperator{\Aff}{Aff}

\begin{document}

\newcommand{\resetexlcounters}{%
  \setcounter{exl}{0}%
} 

\newcommand{\resetremarkcounters}{%
  \setcounter{remark}{0}%
} 

\newcommand{\reseconscounters}{%
  \setcounter{cons}{0}%
} 

\newcommand{\resetall}{%
    \resetexlcounters
    \resetremarkcounters
    \reseconscounters%
}

\maketitle 

\newpage

\hypertarget{t1}{Основные} (по моему мнению) факты по топологии.
\tableofcontents

\newpage


\section{Аффинные пространства.}

\subsection{Начальные определения и свойства.}

\begin{defn}
    \textit{Аффинное пространство} - тройка $(X, \vec{X}, +)$, состоящая из непустого множества \textit{точек}, векторного пространства над $\RR$ (\textit{присоединённое}) и операцией $+:X\times \vec{X}\ra X$ \textit{откладывания вектора}. \ 

    Налагаемые условия - для любых точек $x, y\in X$ существует единственный вектор $v\in \vec{X}$ такой, что $x+v=y$ ($\vec{xy}$), а также ассоциативность откладывания вектора.
\end{defn}

\begin{defn}
    \textit{Начало отсчёта} аффинного пространства - произвольная фиксированная точка $o\in X$. 
\end{defn}

\begin{lemma}
    Начало отсчёта $o\in X$ задаёт биекцию $\varphi_o: X\ra \vec{X}$ по правилу:
    \[
        \varphi_o(x)=\vec{ox} \: \forall x \in X. 
    \]
    Такая биекция называется \textit{векторизацией} аффинного пространства.
\end{lemma}

\begin{defn}
    \textit{Линейная комбинация} $\sum t_i p_i$ точек с коэффициентами относительно начала отсчёта $o\in X$ - вектор $v = \sum t_i \vec{op_i}$, или точка $p=o+v$. Комбинация называется \textit{барицентрической}, если сумма коэффициентов равна единице, и \textit{сбалансированной}, если сумма коэффициентов равна нулю.
\end{defn}

\begin{theorem}
    Барицентрическая комбинация точек - точка, не зависящая от начала отсчёта. Сбалансированная комбинация точек - вектор, не зависящий от начала отсчёта.
\end{theorem}

\subsection{Материальные точки.}

\begin{defn}
    Пусть $x$ — некоторая точка аффинного пространства и $m$ — ненулевое число. \textit{Материальной точкой} $(x,m)$ называется пара: точка $x$ с вещественным числом $m$, причем число m называется \textit{массой} материальной точки $(x,m)$, а точка $x$ — носителем этой материальной точки.
\end{defn}

\begin{defn}
    \textit{Центром масс} системы материальных точек $(x_i, m_i)$ называется такая точка $z$ (притом единственная), для которой имеет место равенство
    \[
        m_1 \cdot \vec{zx_1} + \ldots + m_n \cdot \vec{zx_n} = 0. 
    \]
\end{defn}

\subsection{Аффинные подпространства и оболочки.}

\begin{defn}
    Множество $Y\subset X$ - \textit{аффинное подпространство}, если существуют такие линейное подпространство $V\subset \vec{X}$ и точка $p\in Y$, что $Y=p+V$. $V$ называется\textit{направлением} $Y$. Определение подпространства не зависит от выбора точки в нём.
\end{defn}

\begin{defn}
    \textit{Размерность} $\dim X$ афинного пространства есть размерность его присоединённого векторного пространства.
\end{defn}

\begin{defn}
    \textit{Параллельный перенос} на вектор $v\in \vec{X}$ - отображение $T_v:X\ra X$, заданное равенством $T_v(x)=x+v$. 
\end{defn}

\begin{defn}
    Аффинные подпространства одинаковой размерности \textit{параллельны}, если их направления совпадают.
\end{defn}

\begin{defn}
    \textit{Прямая} - аффинное подпространство размерности 1, \textit{гиперплоскость} в $X$ - аффинное подпространство размерности $\sim X - 1$. 
\end{defn}

\begin{stat}
    Две различные гиперплоскости не пересекаются тогда и только тогда, когда они параллельны.
\end{stat}

\begin{defn}
    \textit{Суммой аффинных подпространств} называется наименьшее аффинное подпространство, их содержащее.
\end{defn}

\begin{theorem}
    Пересечение любого набора аффинных подпространств - либо пустое мноежство, либо аффинное подпространство.
\end{theorem}

\begin{defn}
    \textit{Аффинная оболочка} $\Aff{A}$ непустого множества $A\subset X$ - пересечение всех аффинных подпространств, содержащих $A$. Как следствие, это - наименьшее аффинное подпространство, содержащее $A$. 
\end{defn}

\begin{theorem}
    $\Aff(A)$ - множество всех барицентрических комбинаций точек из $A$. 
\end{theorem}

\begin{defn}
    Точки $p_1, \ldots, p_k$ \textit{аффинно зависимы}, если существуют такие коэффициенты $t_i \in \RR$, не все равные нулю, что $\sum t_i = 0$ и $\sum t_ip_i = 0$. Если такой комбинации нет, то точки \textit{аффинно независимы}.
\end{defn}

\begin{theorem}
    (Переформулировки аффинной независимости.) Для $p_1, \ldots, p_k \in X$ следующие свойства эквивалентны: 

    \begin{itemize}
        \item они аффинно независимы; 
        \item векторы $p_1 p_i$, $i\in \{2, 3, \ldots, k\}$, линейно независимы; 
        \item $\dim \Aff (p_1, \ldots, p_k)=k-1$; 
        \item каждая точка из $\Aff(p_1, \ldots, p_k)$ единственным образом представляется в виде барицентрической комбинации $p_i$. 
    \end{itemize}
\end{theorem}

\subsection{Базисы и отображения.}

\begin{defn}
    \textit{Аффинный базис} - набор $n+1$ точке в $X$, пространстве размерности $n$, являющийся аффинно независимым. Или же, это - точке $o\in X$ и базис $e_0, \ldots, e_n$ пространства $\vec{X}$. 
\end{defn}

\begin{defn}
    Каждая точка однозначно записывается в виде барицентрической комбинации $\sum_{i=0}^n t_i e_i$, а числа $t_i$ называют \textit{барицентрическими координатами} этой точки.
\end{defn}

\begin{defn}
    (Говно-определение). Отображение $F:X\ra Y$ называется \textit{аффинным}, если отображение $\tilde{F}_p$ линейно для некоторой точки $p\in X$. Отображение $\tilde{F}_p:\vec{X}\ra\vec{Y}$ индуцируется из любого отображения $F:X\ra Y$ посредством формулы $\forall v\in \vec{X}$ $\tilde{F}_p(v)=\overrightarrow{F(p)F(q)}$, где $q=p+v$. 
\end{defn}

\begin{defn}
    Отображение $\tilde{F}$ называется \textit{линейной частью} аффинного отображения $F$.
\end{defn}

\begin{defn}
    (Нормальное определение.) Отображение $F:X\ra Y$ называется \textit{аффинным}, если существует такое линейное $L:\vec{X}\ra\vec{Y}$, что для любых $q, p\in X$, $\overrightarrow{F(p)F(q)}=L(\vec{pq})$. 
\end{defn}

\begin{theorem}
    Пусть $x\in X$, $y\in Y$, $L:\vec{X}\ra \vec{Y}$ линейно. Тогда существует единственное аффинное отображение $F:X\ra Y$ такое, что $\tilde{F}=L$ и $F(x)=y$. 
\end{theorem} \

\begin{lemma}
    Пусть $p_1, \ldots, p_n$ - аффинно независимые точки в аффинном пространстве $X$, $q_1, \ldots, q_n$ - точки в аффинном пространстве $Y$. Тогда существует такое аффинное отображение $F:X\ra Y$, что $F(p_i)=q_i$ $\forall i$. Кроме того, если $\dim X = n-1$, то такое отображение единственно.
\end{lemma} \ 

\begin{lemma}
    Аффинное отображение сохраняет барицентрические комбинации.
\end{lemma} \ 

\begin{lemma}
    Композиция аффинных отображений - аффинное отображение. При этом линейная часть композиции - композиция линейных частей.
\end{lemma}

\begin{stat}
    Образ и прообраз аффинного подпространства - аффинное подпространство. Образы (прообразы) параллельных подпространств параллельны.
\end{stat}

\begin{theorem}
    Параллельный перенос - аффинное отображение, его линейная часть тождественна. Верно также и обратное.
\end{theorem}

\begin{defn}
    Аффинное отображение $F:X\ra X$ такое, что $\tilde{F}=k \id$ для некоторого $k \in \RR\backslash \{0, 1\}$, называется \textit{гомотетией}, а $k$ называют \textit{коэффициентом растяжения} гомотетии $F$. Такое отображение имеет ровно одну неподвижную точку, называемую центром.
\end{defn}

\begin{theorem}
    (\textit{Основная теорема аффинной геометрии.}) Пусть $X, Y$ - аффинные пространства, $\dim X \geq 2$. Пусть $F: X\ra Y$ - инъективное отображение, и для любой прямой $l\subset X$ её образ $F(l)$ - тоже прямая. Тогда $F$ - аффинное отображение.
\end{theorem}

\end{document}
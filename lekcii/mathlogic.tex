\documentclass[a4paper,100pt]{article}

\usepackage[utf8]{inputenc}
\usepackage[unicode, pdftex]{hyperref}
\usepackage{cmap}
\usepackage{mathtext}
\usepackage{multicol}
\setlength{\columnsep}{1cm}
\usepackage[T2A]{fontenc}
\usepackage[english,russian]{babel}
\usepackage{amsmath,amsfonts,amssymb,amsthm,mathtools}
\usepackage{icomma}
\usepackage{euscript}
\usepackage{mathrsfs}
\usepackage{geometry}
\usepackage[usenames]{color}
\hypersetup{
     colorlinks=true,
     linkcolor=magenta,
     filecolor=magenta,
     citecolor=black,      
     urlcolor=cyan,
     }
\usepackage{fancyhdr}
\pagestyle{fancy} 
\fancyhead{} 
\fancyhead[LE,RO]{\thepage} 
\fancyhead[CO]{\hyperlink{t2}{к списку объектов}}
\fancyhead[LO]{\hyperlink{t1}{к содержанию}} 
\fancyhead[CE]{текст-центр-четные} 
\fancyfoot{}
\newtheoremstyle{indented}{0 pt}{0 pt}{\itshape}{}{\bfseries}{. }{0 em}{ }

%\geometry{verbose,a4paper,tmargin=2cm,bmargin=2cm,lmargin=2.5cm,rmargin=1.5cm}

\title{Матлог. Основные записи}
\author{Мастера конспектов}
\date{22 января 2020 г.}

\theoremstyle{indented}
\newtheorem{theorem}{Теорема}
\newtheorem{lemma}{Лемма}

\theoremstyle{definition} 
\newtheorem{defn}{Определение}
\newtheorem{exl}{Пример(ы)}

\theoremstyle{remark} 
\newtheorem{remark}{Примечание}
\newtheorem{cons}{Следствие}
\newtheorem{exer}{Упражнение}

\DeclareMathOperator{\Ker}{Ker}
\DeclareMathOperator{\Frac}{Frac}
\DeclareMathOperator{\Imf}{Im}
\DeclareMathOperator{\cont}{cont}
\DeclareMathOperator{\id}{id}
\DeclareMathOperator{\ev}{ev}
\DeclareMathOperator{\lcm}{lcm}
\DeclareMathOperator{\chard}{char}
\DeclareMathOperator{\CC}{\mathbb{C}}
\DeclareMathOperator{\ZZ}{\mathbb{Z}}
\DeclareMathOperator{\RR}{\mathbb{R}}
\DeclareMathOperator{\NN}{\mathbb{N}}
\DeclareMathOperator{\codim}{codim}
\DeclareMathOperator{\rank}{rank}
\DeclareMathOperator{\ord}{ord}
\DeclareMathOperator{\adj}{adj}
\DeclareMathOperator{\Prop}{Prop}
\DeclareMathOperator{\LL}{\mathscr{L}}
\DeclareMathOperator{\form}{Form}
\DeclareMathOperator{\Sub}{Sub}

\begin{document}

\newcommand{\resetexlcounters}{%
  \setcounter{exl}{0}%
} 

\newcommand{\resetremarkcounters}{%
  \setcounter{remark}{0}%
} 

\newcommand{\reseconscounters}{%
  \setcounter{cons}{0}%
} 

\newcommand{\resetall}{%
    \resetexlcounters
    \resetremarkcounters
    \reseconscounters%
}

\maketitle 

\newpage

\hypertarget{t1}{Основные моменты}. 
\tableofcontents

\newpage

\section{Лекция 1.}

\begin{defn}
  \textit{Конкатенация} - записали подряд два слова. ($A$ - алфавит, $A^*$ - слова).
\end{defn}

\begin{defn}
  \textit{Подслово} - как есть, \textit{вхождение} - учитываем, где начинается подслово. Если подслово стоит в начале, то мы его и называем \textit{начало}, а обозначаем как $\psi \sqsubseteq \varphi$.
\end{defn}

\begin{defn}
  $w[w'/u, k]$ - замена подслова $w'$ на $u$, начинающегося в позиции $k$.
\end{defn}

\begin{defn}
  Фиксированное счётное множество $\Prop$ - \textit{пропозициональные переменные}. Язык $\LL$ классической пропозициональной логики состоит из переменных, а также символов $\rightarrow, \vee, \wedge, \neg$ и круглых скобочек.
\end{defn}

\begin{defn}
  $\form$ (формулы) - наименьшее множество слов в алфавите, замкнутое относительно следующих порождающих правил:
  \begin{itemize}
    \item если $p\in \Prop$, то $p\in \form$; 
    \item если $\{\varphi, \psi\}\subseteq \form$, то $(\varphi*\psi)\in \form$, где $*$ - любая из операций в определении выше (если отрицание, то отсительно одногой формулы, конечно).
  \end{itemize}
\end{defn}

\begin{lemma}
  Пусть $\{\varphi , \psi\}\subseteq \form$ таковы, что $\psi \sqsubseteq \varphi$. Тогда $\psi = \varphi$. 
\end{lemma}

\begin{proof}
  По индукции по мощности большей формулы. База - переменная, очевидно. Иначе $\psi$ представляется в виде ''композиции'' единственным образом, тогда возьмём первую часть этой композиции и сравним с первой частью того, как $\varphi$ представляется в виде ''композиции''. По предположению индукции они должны совпасть, продолжение тривиально.
\end{proof}

\begin{lemma}
  Каждую $\varphi \in \form \backslash \Prop$ можно единственным способом представить в виде $(\theta \rightarrow \chi)$, $(\theta \vee \chi)$, $(\theta \wedge \chi)$ или $\neg\theta$, где $\{\theta, \chi\}\subseteq \form$ (это я везде безграмотно называю \textit{композицией}). 
\end{lemma}

\begin{proof}
  От противного по лемме 2.
\end{proof}

\begin{defn}
  Для каждой $\varphi\in \form$ определим $\Sub(\varphi):=\{\psi\in \form|\psi \preccurlyeq \varphi\}$ - \textit{подформулы}.
\end{defn}

\begin{lemma}
  Пусть $\varphi \in \form$. Тогда каждое вхождение $\neq$ или $($ является началом вхождения некоторой подформулы.
\end{lemma}

\begin{proof}
  Возвратная индукция по длине формулы.
\end{proof}

\begin{lemma}
  Множество подслов $\varphi$ - объединение множеств подслов элементов его композиции и его самого.
\end{lemma}

\begin{proof}
  Из лемм выше.
\end{proof}

\begin{defn}
  \textit{Оценка} ($v$) - произвольная функция из $\Prop$ в $\{0, 1\}$, которую можно расширить и до $\form$ ($v^*$) посредством применения операций к переменным. Если $v^*(\varphi)=1$, то порой пишут $v \Vdash \varphi$.
\end{defn}

\begin{defn}
  Формулу называют \textit{выполнимой}, если $v \Vdash \varphi$ для некоторой оценки, и \textit{общезначимой} (тождественно истинной или тавтологией), если $v \Vdash \varphi$ для всех оценок.
\end{defn}

\begin{defn}
  Формула \textit{семантически следует} из множества формул и записывается $\Gamma \vDash \varphi$, если для любой оценки $v$, любая формула из множества истина, то $\varphi$ истина.  \ 
  
  Формулы называют \textit{семантически эквивалентны}, и пишут $\varphi \equiv \psi$, если $\vDash \varphi \leftrightarrow \psi$. 
\end{defn}


\section{Лекция 2.}

В Гильбертовском исчислении для классической пропозициональной логики используются следующие схемы аксиом (implication, conjunction, disjunction, negotiation): 

\begin{itemize}
  \item (I1). $\varphi\rightarrow (\psi\rightarrow \varphi)$;
  \item (I2). $\varphi\rightarrow (\psi\rightarrow \chi)\rightarrow((\varphi\rightarrow \psi)\rightarrow (\varphi\rightarrow \chi))$; 
  \item (C1). $\varphi \wedge \psi \rightarrow \varphi$; 
  \item (C2). $\varphi \wedge \psi \rightarrow \psi$; 
  \item (C3). $\varphi\rightarrow (\psi \rightarrow \varphi \wedge \psi)$;
  \item (D1). $\varphi\rightarrow \varphi \vee\psi$;
  \item (D2). $\psi\rightarrow \varphi \vee \psi$; 
  \item (D3). $(\varphi\rightarrow \chi)\rightarrow((\psi\rightarrow \chi)\rightarrow (\varphi\vee\psi\rightarrow\chi))$;
  \item (N1). $(\varphi\rightarrow \psi)\rightarrow((\varphi\rightarrow \neg \psi)\rightarrow \neg \varphi)$; 
  \item (N2). $\neg\varphi \rightarrow (\varphi\rightarrow \psi)$; 
  \item (N3). $\varphi \vee \neg \varphi$, 
\end{itemize}

а также, одно \textit{правило вывода}, которое называется \textit{modus ponents}:

\begin{center}
  \begin{tabular}{c c c c c}
    $\varphi$ & & $\varphi$ & $\rightarrow$ & $\psi$ \\ 
    \hline
    & & $\psi$ & & 
  \end{tabular}
\end{center}

\begin{defn}
  Пусть $\Gamma \subseteq \form$, тогда \textit{выводом} из него в гильбертовском исчислении понимают конечную последовательность $\varphi_0, \ldots, \varphi_n$ ($n\in \NN$) элементов $\form$, что для каждого $i\in\{0, \ldots, n\}$ выполнено одно из следующиъ условий: 

  \begin{itemize}
    \item $\varphi_i$ - аксиома; 
    \item $\varphi_i$ - элемент $\Gamma$; 
    \item $\exists \{j, k\}\subseteq\{0, \ldots, i-1\}$ такие, что $\varphi_k$ есть $\varphi_j\rightarrow \varphi_i$. 
  \end{itemize}
  При этом, $\varphi_n$ - \textit{заключение}, а элементы $\Gamma$ - \textit{гипотезы}. Если $\varphi$ выводится из $\Gamma$, то пишут $\Gamma \vdash \varphi$. 
\end{defn}

Основные свойства $\vdash$:

\begin{itemize}
  \item монотонность; 
  \item транзитивность;
  \item компактность (если $\Gamma \vdash \varphi$, то $\Delta \vdash \varphi$ для некоторого конечного $\Delta \subseteq \Gamma$).
\end{itemize}

\begin{theorem}
  (О дедукции). Для любых $\Gamma \cup \{\varphi, \psi\}\subseteq \form$, 
  \[
    \Gamma \cup \{\varphi\}\vdash \psi \Longleftrightarrow \Gamma \vdash \varphi \rightarrow \psi. 
  \]
\end{theorem}

\begin{proof}
  В одну правую сторону очевидно, в обратную - по индукции по $i\in\{0, 1, \ldots, n\}$ показываем, что $\Gamma\vdash \varphi\rightarrow \psi_i$, там три случая, и все, кроме одного, тривиальны.
\end{proof}

Введём обозначения: $\top:= p\rightarrow p$ и $\perp := \neg \top$, где $p$ - фиксированная пропозициональная переменная.

\begin{cons}
  Для любых $\Gamma\cup \{\varphi\}\subseteq \form$, 
  \[
    \Gamma \vdash \varphi \Longleftrightarrow \vdash \bigwedge_{i=1}^n \psi_i\rightarrow \varphi
  \]
  для некоторых $\{\varphi_1, \ldots, \varphi_n\}\subseteq \Gamma$.
\end{cons}

\begin{proof}
  Влево - очевидно, вправо - очевидно и применяется теорема о дедукции.
\end{proof}

\begin{lemma}
  Всякая аксиома гильбертовского исчисления для классической пропозициональной логики общезначима.
\end{lemma}\ 


\begin{theorem}
  Для любых $\Gamma \cup \{\varphi\}\subseteq \form$, 
  \[
    \Gamma \vdash \varphi \Longrightarrow \Gamma \vDash \varphi. 
  \]
\end{theorem}

\begin{proof}
  Фиксируем вывод $\varphi_0, \ldots, \varphi_n = \varphi$. Затем рассматриваем произаольную оценку $v$ такую что $v\Vdash \psi$ для всех $\psi \in \Gamma$ и покажем по индукции по $i\in\{0, \ldots, n\}$, что $v\Vdash \varphi_i$.
\end{proof}

\begin{defn}
  $\Gamma \subseteq \form$ называется \textit{простой теорией}, если оно обладает следующими свойствами:

  \begin{itemize}
    \item $\Gamma \neq \form$; 
    \item $\{\varphi \in \form |\Gamma \vdash \varphi\}\subseteq \Gamma$; 
    \item для любого $\varphi \vee \psi \in \Gamma$ верно $\varphi \in \Gamma$ или $\psi \in \Gamma$. 
  \end{itemize}
\end{defn}

\begin{lemma}
  Пусть $\Gamma$ - простая теория, тогда для любых её элементов можно переписать действия над ними в рамках принадлежности к теории.
\end{lemma}\

\begin{lemma}
  (О расширении. a.k.a. Линденбаума). Пусть $\Gamma \cup \{\varphi \}\subseteq \form$ таковы, что $\Gamma \nvdash \varphi$. Тогда существует простая теория $\Gamma'\supseteq \Gamma$ такая, что $\Gamma' \nvdash \varphi$.
\end{lemma}

\begin{proof}
  Рекурсивно докидываем к $\Gamma$ элементы $\form$ (их счётно).
\end{proof}




\end{document}
\documentclass[a4paper,100pt]{article}

\usepackage[utf8]{inputenc}
\usepackage[unicode, pdftex]{hyperref}
\usepackage{cmap}
\usepackage{mathtext}
\usepackage{multicol}
\setlength{\columnsep}{1cm}
\usepackage[T2A]{fontenc}
\usepackage[english,russian]{babel}
\usepackage{amsmath,amsfonts,amssymb,amsthm,mathtools}
\usepackage{icomma}
\usepackage{euscript}
\usepackage{mathrsfs}
\usepackage{geometry}
\usepackage[usenames]{color}
\hypersetup{
     colorlinks=true,
     linkcolor=red,
     filecolor=red,
     citecolor=black,      
     urlcolor=cyan,
     }
\usepackage{fancyhdr}
\pagestyle{fancy} 
\fancyhead{} 
\fancyhead[LE,RO]{\thepage} 
\fancyhead[CO]{\hyperlink{t2}{к списку объектов}}
\fancyhead[LO]{\hyperlink{t1}{к содержанию}} 
\fancyhead[CE]{текст-центр-четные} 
\fancyfoot{}
\newtheoremstyle{indented}{0 pt}{0 pt}{\itshape}{}{\bfseries}{. }{0 em}{ }

%\geometry{verbose,a4paper,tmargin=2cm,bmargin=2cm,lmargin=2.5cm,rmargin=1.5cm}

\title{Алгебра. Конспект 2 сем.}
\author{Мастера Конспектов\\ \\ (по материалам лекций В. А. Петрова,\\ а также других источников)}
\date{12 февраля 2021 г.}

\theoremstyle{indented}
\newtheorem{theorem}{Теорема}
\newtheorem{lemma}{Лемма}

\theoremstyle{definition} 
\newtheorem{defn}{Определение}
\newtheorem{exl}{Пример(ы)}

\theoremstyle{remark} 
\newtheorem{remark}{Примечание}
\newtheorem{cons}{Следствие}
\newtheorem{stat}{Утверждение}

\DeclareMathOperator{\Ker}{Ker}
\DeclareMathOperator{\Tors}{Tors}
\DeclareMathOperator{\Frac}{Frac}
\DeclareMathOperator{\Imf}{Im}
\DeclareMathOperator{\cont}{cont}
\DeclareMathOperator{\id}{id}
\DeclareMathOperator{\ev}{ev}
\DeclareMathOperator{\lcm}{lcm}
\DeclareMathOperator{\chard}{char}
\DeclareMathOperator{\CC}{\mathbb{C}}
\DeclareMathOperator{\ZZ}{\mathbb{Z}}
\DeclareMathOperator{\RR}{\mathbb{R}}
\DeclareMathOperator{\NN}{\mathbb{N}}
\DeclareMathOperator{\codim}{codim}
\DeclareMathOperator{\rank}{rank}
\DeclareMathOperator{\ord}{ord}
\DeclareMathOperator{\adj}{adj}
\DeclareMathOperator{\Ann}{Ann}

\begin{document}

\newcommand{\resetexlcounters}{%
  \setcounter{exl}{0}%
} 

\newcommand{\resetremarkcounters}{%
  \setcounter{remark}{0}%
} 

\newcommand{\reseconscounters}{%
  \setcounter{cons}{0}%
} 

\newcommand{\resetall}{%
    \resetexlcounters
    \resetremarkcounters
    \reseconscounters%
}

\maketitle 

\newpage

\hypertarget{t1}{Некоторые} записи по алгебре.
\tableofcontents

\newpage


\section{Лекция 1.}

Пусть $R$ - кольцо главных идеалов, а $M$ - конечно порождённый $R$-модуль (левый). 

\[
    m_1, \ldots, m_n\in M, \:\:M=\{\sum r_im_i \vert r_i\in R\}
\]

Пусть $\varphi: R^n\rightarrow M$ - функция, которая действует по правилу $e_i\mapsto m_i$ (базисные элементы $R^n$ (именно тривиального базиса) в элементы $m_i$). \

Тогдя ядро $\Ker \varphi \leq R^n$ - подмодуль. Причём равен он $\{(r_i)\vert \sum r_im_i=0\}$ - \textit{соотношения} (линейные) между $m_i$. А также он есть \textit{свободный} модуль $R^k, \: k\leq n$. 

\[
    \Ker \varphi =R^k, \:\: R^k\leq R^n 
\]
\[
    \psi: R^k\rightarrow R^n
\]

Подходящей заменой базиса в $R^k$ и $R^n$ можно добиться того, чтобы $\psi$ стала диагональной матрицей (с нижними нулевыми строками, естественно) и числами $d_1\vert d_2\vert \ldots\vert d_k$ на диагонали.

Тогда $M\cong R^{n-k}\oplus R/(d_i)\oplus\ldots \oplus R/(d_k)$ (это планируется доказывать, но перед этим нужно ввести несколько определений).\ 

\begin{defn} 
    Пусть $R$ кольцо (не обязательно коммутативное), тогда $M$ - \textit{циклический}, если он порождён одним элементом ($M=\{rm\vert r\in R\}$).
\end{defn}

Пусть $\theta: R\rightarrow M$ - гомоморфизм $R$-модулей, действующий по правилу $r\mapsto rm$, он сюръективен и $M\simeq R/\Ker \theta$ по теореме о гоморфизме.

\[
    \Ker\theta = \{r\in R\vert rm=0\} \leq R, 
\]
что также является левым идеалом.\ 

А если $R$ - область главных идеалов, то циклический модуль выглядит как $R/(d)$. Если $d = 0$, то $R$ - свободный модуль ранга $1$, а если он не равен нулю, то это есть \textit{модуль кручения} $\forall x\in M \: dx=0$.\\

\begin{theorem}
    Конечнопорождённый модуль над областью главных идеалов - конечная прямая сумма циклических модулей.
\end{theorem}\

Была доказана в прошлом семестре (не у нас). Однаком мы можем сформулировать следствие:

\begin{cons}
    Конечнопорождённая абелева группа - конечная прямая сумма циклических групп.
\end{cons}

Пусть $R$ - область, $M$ - $R$-модуль, тогда подмодуль кручения - 

\[
    \Tors (M) = \{m\in M\vert \exists r\neq 0, \: rm=0\}
\]

\begin{stat}
    $\Tors (M)$ - модмодуль в $M$.
\end{stat}

Нужно выполнить проверку этого утверждения, но для этого достаточно проверить, что всё хорошо с нулём (он там лежит и $1\cdot 0 = 0$), а затем несколько свойств:

\[
    m_1, m_2\in \Tors (M), \: r_1, r_2 \neq 0, \: r_1m_1=r_2m_2=0, 
\]
тогда
\[
    r_1r_2(m_1+m_2)=0, \: r_1r_2\neq 0, 
\]
а также, если
\[
    m\in \Tors (M), \: s\in R, \: rm=0 \Rightarrow r(sm)=rsm=s(rm)=0.
\]

Пусть $r\in R$, $r\neq 0$, $M[r]:=\{m\in M:\: rm=0\}\leq M$ - подмодуль, $p$ - пргстой элемент $R$. Рассмотрим $M[p]\leq M[p^2]\leq M[p^3]\leq \ldots$ - получили цепочку вложенных модулей.\ 

$M_p:=\bigcup_{i\geq 1}M[p^i]$ - подмодуль, $p$-кручение в $M$. \\

Сейчас начнётся пиздец. Наша цель: доказать, что $\Tors (M)\cong \bigoplus_{p-\text{простое}} M_p$.\ 

$N_i$ - модули $i\in I$, $\bigoplus :=\{(n_i)_{i\in I}\vert n_i\in N_i, \text{ почти все }n_i=0\}$, операции покомпонентные. Это, получается, (бесконечная) прямая сумма модулей.\\

\begin{theorem}
    (О примарном разложении). Пусть $R$ - область главных идеалов, $M$ - $R$-модуль. Тогда $\bigoplus M_p \rightarrow \Tors (M)$, дествующий по правилу $(m_p)\mapsto \sum m_p$ (конечная сумма) - изоморфизм модулей.
\end{theorem}

\begin{proof}
    Докажем всё по порядку:\

    \begin{itemize}
        \item Докажем, что это гомоморфизм. $(m_p+n_p)\mapsto \sum m_p+n_p=\sum m_p+\sum n_p$, а также $(rm_p)\mapsto \sum rm_p=r(\sum m_p)$. 
        \item Теперь нужно доказать сюръективность. $m\in \Tors (m)$, $rm=0$, $r=\Pi_{i=1}^np_i^{\alpha_i}$, где $p_i$ - простое. Рассмотрим линейное разложение $\text{НОД}$: 
        \[
            r_1p_2^{\alpha_2}\ldots p_n^{\alpha_n}+\ldots+r_np_1^{\alpha_1}\ldots p_{n-1}^{\alpha_{n-1}}=1.
        \]
        Тогда если мы домножим равенство на $m$, получим, что $r_i=\frac{rm}{p_i^{\alpha_i}}\in M_{p_i}$, тогда получили, что $(r_1p_2^{\alpha_2} \ldots  p_n^{\alpha_n}m, \ldots, r_np_1^{\alpha_1}\ldots p_{n-1}^{\alpha_{n-1}}m)\mapsto m$.
        \item Осталась инъективность. Пусть $0\neq(m_p)\mapsto 0$, возьмём наименьшее число индексов, что $\sum m_p=0$. А теперь начнём его уменьшать. Пусть у нас есть $p_1, \ldots , p_n$, $p_i^{\alpha_i}m_{p_i}=0$. Всё домножим на $p_n^{\alpha_n}$, получим $\sum p_n^{\alpha_n}m_p=0$. Тогда раньше было $m_{p_n}\neq 0$, а теперь $p_n^{\alpha_n}m_{p_m}=0$. Докажем, что ничего, кроме последнего не обнулилось. Предположим противное, $p_1^{\alpha_1}m_1=0$, $p_n^{\alpha_n}m_1=0$, но $p_1^{\alpha_1}, \: p_n^{\alpha_n}$ - взаимно просты, тогда есть линейное разложение $r_1p_1^{\alpha_1}+ r_n p_n^{\alpha_n}=1$, домножим на $m$, получим $r_1p_1^{\alpha_1}m_1+ r_n p_n^{\alpha_n}m_1=m_1$, но оба они не могут быть равны нулю.
    \end{itemize}
\end{proof}

Сейчас будем заниматься в основном кольцом многочленов. Пусть $R=F[t]$, $F$ - поле, $V$ - $R$-модуль. В частности, $V$ - $F$-модуль, то векторное пространство $A:v\rightarrow tv$ - $F$-линейное отображение $V\rightarrow V$ \textit{оператор}. Линейные операторы образуют кольцо (сумма - поточечно, умножение - композиция). $A(v)$ или $Av$.
\[
    (a_0+a_1t+\ldots+a_nt^n)V=a_0v+a_1Av+\ldots+a_nA^nv
\]
$V$ - векторное порстранство с оператором, значит, $F[t]$ - модуль.\ 

Пусть $a$ - матрица $n\times n$ $F^n\rightarrow F^n$, $F[t]$ - модуль на $F^n$. $F[t]$ - как модуль над собой векторное пространство со счётным базисом.\ 


\begin{stat}
    Пусть $V$ возьмём конечнопорождённый модуль над $F[t]$, тогда $V$ - конечномерное векторное пространство над $F$ тогда и только тогда, когда $V=\Tors (V)$ (как $F[t]$-модуль).
\end{stat}

\begin{proof}
    $F[t]^n\oplus F[t]/(f_i)\oplus\ldots\oplus F[t]/(f_k)$, где $f_i\neq 0$. Если $n\neq 0$, то в $V$ есть бесконечномерное подпространство $F[t]$. Если $n=0$, то $\dim _F F[t]/(f_i)=\deg f_i < \infty$.
\end{proof}

Теперь рассмотрим матрицы. Пусть $\dim V =n $, $A: V\rightarrow V$. Если зафиксировать базис в $V$, получается матрица $a$ $n\times n$. Взали другой базис, получим матрицу перехода $c$. $V\rightarrow V$ посредством $A$, причём стороны соответственно изоморфны вот таким вещам (по центру, я не умею так круто чертить, загляните в лекцию) $F^n\xrightarrow{c^{-1}}F^n\xrightarrow{a}F^n\xrightarrow{c}F^n$. И, кстати, $a\sim c^{-1}ac$ (сопряжённая матрица).\\

\textsf{тут надо дописать какую-то ебанину с кучей формул}

\section{Лекция 2.}

Начинаем опять с оператора. Рассматриваем векторной пространство $V$ над каким-то полем $F$ и мы действуем на него оператором $A:V\rightarrow V$. Мы его также рассматривали как $F[t]$-модуль, $t\cdot v=Av$. Мы определили минимальный многочлен $A$ такой, что $\{g(t)\in F[t]\vert g(a)=0\}\triangleleft F[t]$, причём $F[t]=(f(t))$ - идеал унитарного (нуо) многочлена. Такой $f(t)$ и называется минимальным многочленом. \
    
Теперь немного понятнее на языке модулей. Рассмотрим $V$ - $F[t]$-модуль, а также $\Ann(V):=\{r\in V\vert rv=0, \: \forall v\in V\}$. Это - идеал в $R$, причём даже двусторонний (можно будет потом записать проверку). Причём получаем, что $\Ann(V)=(f(t))$, легко заметить, что они совпадают.\ 

$g(A)v=0$, но тогда 

\[
    g=a_0+a_1t+\ldots+a_kt^k
\]
\[
    g=a_0+a_1Av+\ldots+a_kA^tv=0,
\]
что также и равно $g(t)\cdot v$. Тогда $f(A)v=g(t)\cdot v$ как оператор и из структуры модуля соответственно. Тогда $g(A)=0$ $\Leftrightarrow$ $g(A)\cdot v=0$ для любого $v\in V$ $\Leftrightarrow$ $g(t)\cdot v=0$ $\forall v\in V$ $\Leftrightarrow$ $g(t)\in \Ann (v)$.\ 

Мы уже начинали рассматривать такой модуль: $F[t]/(f(t))$ - $F[t]$-модуль, имеем также $V$, $Av=t\cdot v$. Мы хотим придумать базис $V$, в которм матрица $A$ имеет простой вид. Возьмём такой базис: $[1], [t], \ldots, [t^{k-1}]$, тогда $[t^k]=-a_0[1]-\ldots-a_{k-1}[t^{k-1}]$. Как выглядит матрица $A$ в этом базисе?

\begin{equation*}
    \begin{pmatrix}
        0 & 0 & \dots & 0 & -a_0\\
        1 & 0& \dots & 0 & -a_1\\
        \vdots & \vdots &\ddots & \vdots  & \vdots  \\
        \vdots & \vdots & \dots & 0 & -a_{k-2}\\
        0 & 0 & \dots & 1 & -a_{k-1}
    \end{pmatrix}
\end{equation*}

Такая матрица называется \textit{фробениусовой клеткой}. А вообще, в итоге мы получили, что если $V$ - циклический $F[t]$-модуль, то $A$ в некотором базисе записывается фробениусовой клеткой, причём последним столбцом будут коэффициенты минимального многочлена, только со знаком ''минус''.\ 

А если модуль не циклический (произвольный и с конечномерным $V$), то мы можем его разложить в прямую сумма циклических: 

\[
    F[t]/(f_1(t))\oplus F[t]/(f_2(t))\oplus\ldots\oplus F[t]/(f_m(t)), 
\]
причём мы можем даже потребовать, чтобы $f_1\vert f_2\vert\ldots\vert f_n$. \ 

Умножение на $t$ будет действовать поккординатно.

\end{document}
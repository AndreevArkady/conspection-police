\documentclass[a4paper,100pt]{article}

\usepackage[utf8]{inputenc}
\usepackage[unicode, pdftex]{hyperref}
\usepackage{cmap}
\usepackage{mathtext}
\usepackage{multicol}
\setlength{\columnsep}{1cm}
\usepackage[T2A]{fontenc}
\usepackage[english,russian]{babel}
\usepackage{amsmath,amsfonts,amssymb,amsthm,mathtools}
\usepackage{icomma}
\usepackage{euscript}
\usepackage{mathrsfs}
\usepackage{geometry}
\usepackage[usenames]{color}
\hypersetup{
     colorlinks=true,
     linkcolor=magenta,
     filecolor=green,
     citecolor=black,      
     urlcolor=cyan,
     }
\usepackage{fancyhdr}
\pagestyle{fancy} 
\fancyhead{} 
\fancyhead[LE,RO]{\thepage} 
\fancyhead[CO]{\hyperlink{t2}{к списку объектов}}
\fancyhead[LO]{\hyperlink{t1}{к содержанию}} 
\fancyhead[CE]{текст-центр-четные} 
\fancyfoot{}
\newtheoremstyle{indented}{0 pt}{0 pt}{\itshape}{}{\bfseries}{. }{0 em}{ }

%\geometry{verbose,a4paper,tmargin=2cm,bmargin=2cm,lmargin=2.5cm,rmargin=1.5cm}

\title{Матанализ. Конспект 2 сем.}
\author{Мастера Конспектов\\ \\ (по материалам лекций Белова Ю. С.,\\ а также других источников)}
\date{16 февраля 2021 г.}

\theoremstyle{indented}
\newtheorem{theorem}{Теорема}
\newtheorem{lemma}{Лемма}

\theoremstyle{definition} 
\newtheorem{defn}{Определение}
\newtheorem{exl}{Пример(ы)}

\theoremstyle{remark} 
\newtheorem{remark}{Примечание}
\newtheorem{cons}{Следствие}
\newtheorem{stat}{Утверждение}

\DeclareMathOperator{\Ker}{Ker}
\DeclareMathOperator{\Tors}{Tors}
\DeclareMathOperator{\Frac}{Frac}
\DeclareMathOperator{\Imf}{Im}
\DeclareMathOperator{\Real}{Re}
\DeclareMathOperator{\cont}{cont}
\DeclareMathOperator{\id}{id}
\DeclareMathOperator{\ev}{ev}
\DeclareMathOperator{\lcm}{lcm}
\DeclareMathOperator{\chard}{char}
\DeclareMathOperator{\CC}{\mathbb{C}}
\DeclareMathOperator{\ZZ}{\mathbb{Z}}
\DeclareMathOperator{\RR}{\mathbb{R}}
\DeclareMathOperator{\NN}{\mathbb{N}}
\DeclareMathOperator{\PP}{\mathbb{P}}
\DeclareMathOperator{\FF}{\mathcal{F}}
\DeclareMathOperator{\Rho}{\mathcal{P}}
\DeclareMathOperator{\codim}{codim}
\DeclareMathOperator{\rank}{rank}
\DeclareMathOperator{\ord}{ord}
\DeclareMathOperator{\adj}{adj}
\DeclareMathOperator{\const}{const}
\begin{document}

\newcommand{\resetexlcounters}{%
  \setcounter{exl}{0}%
} 

\newcommand{\resetremarkcounters}{%
  \setcounter{remark}{0}%
} 

\newcommand{\reseconscounters}{%
  \setcounter{cons}{0}%
} 

\newcommand{\resetall}{%
    \resetexlcounters
    \resetremarkcounters
    \reseconscounters%
}

\maketitle 

\newpage

\hypertarget{t1}{Некоторые} записи по матанализу.
\tableofcontents

\newpage


\section{Лекция 1.}

В этом чеместре мы будем занимать анализом функций от многих переменных, то есть,  $f:\RR^n\rightarrow \RR^m$,  и если $m=1$, то такая функция называется функцией многих переменных. \ 

\begin{defn}
    \textit{Кривые в $\RR^n$} - непрерывное отображение $f:[a, b]\rightarrow \RR^n$.
\end{defn}

Основная проблема состоит в том, что образ может выглядеть очень и очень сложно, потому нам хотелось бы более точно понять, как всё это устроено. Потому начнём рассматривать \textit{спрямляемые кривые}, то есть, кривые с конечной длиной. Введём следующее определение: 

\begin{defn}
    \textit{Вариация функции} - $V_f([a, b])=\sup_{a=x_0\leq x_2\leq \ldots\leq x_n=b} \sum_{k=0}\vert f(x_{k+1})-f(x_k)\vert$. 
\end{defn}

$(x-y)$ - евклидово расстояние.

\begin{stat}
    Если $f:\RR\rightarrow \RR$ монотонна, то $V_f([a, b])=\vert f(a)-f(b)\vert $.
\end{stat}

\begin{stat}
    $V_f([a, b])=0 \: \Leftrightarrow \: f=\const$.
\end{stat}

\begin{stat}
    $V_{f+g}\leq V_f+V_g$.
\end{stat}

\begin{stat}
    $V_f$ аддитивна на промежутке: $a\leq b\leq c$, тогда $V_f([a, c])=V_f([a, b])+V_f([b, c])$. 
\end{stat}

\begin{defn}
    Вариация ограничена, если $V_f<\infty $ на $[a, b]$.
\end{defn}

\begin{lemma}\
    
    \begin{itemize} 
        \item $\RR\rightarrow \RR$, $f_1$ и $f_2$ монотонны, тогда $f_1-f_2$ имеют ограниченную вариацию.
        \item $f$ имеет ограниченную вариацию тогда и только тогда, когда $f=f_1-f_2$ на отрезке $[a, b]$, причём эти две функции монотонно возрастают.
    \end{itemize}
\end{lemma}

\begin{proof}
    Пусть у нас есть $f$, а также $V_f([a, b])<\infty$. Рассмотрим $\varphi(x)=V_f([a, x])$. $\varphi$ определа корректно, причём возрастает. $f=\varphi-(\varphi-f)$, скажем, что $(\varphi-f)=h$, тогда $h(x)\leq h(y)$ при $x\leq y$. Но это нетрудно показать, $\varphi(x)-f(x)\leq \varphi(y)-f(y)$ равносильно $f(y)-f(x)\leq \varphi(y-\varphi(x))=V_f([x, y])$.
\end{proof}

По сути, если понимать определение вариации геометрически, то это просто длина кривой на отрезке. Перейдём теперь к способам обхода кривой.\\

\begin{lemma}
    Пусть $g:[a, b]\rightarrow [c, d]$ - непрерывная биекция (тогда и монотонная). Тогда $V_f[c, d]=V_{f\circ g}([a,b])$.
\end{lemma}

\begin{proof}
    Левая и правая части равны соответственно $\sup \sum_{k=0}^{n-1}\vert f(x_{k+1})-f(x_k)\vert$ и $\sup \sum_{k=0}^{n-1}\vert f(g(y_{k+1}))-f(g(y_k))\vert$. Это, очевидно, одно и то же.
\end{proof}

Теперь стоит задаться вопросом: а когда же это $V_f$ (или же, длину кривой) можно посчитать. Если $f$ - гладкая функция (гладкая покоординатно $f_k$). $f:=[a, b]\rightarrow \RR^n$, $f=(f_1, \ldots, f_n)$, $f_k:[a, b]\rightarrow \RR$. Тогда
\[
    V_f([a, b])=\int_a^b\sqrt{(f_1')^2(x)+\ldots+(f_n')^2(x)}\: dx.
\]
Рассмотрим
\begin{eqnarray*}
    \sup_{a=x_0, \ldots, x_n=b}\sum_{k=0}^{n-1}\sqrt{(f_1(x_{k+1})-f_1(x_k))^2+\ldots+f_n(x_{k+1})-f_n(x_k))^2} = \\ 
    = \sum_{k=0}^{n-1}(x_k+1-x_k)\sqrt{f_1'^2(\xi_{1, k})+\ldots+f_n'^2(\xi_{n, k}))}
\end{eqnarray*}

Если $f_i$ непрерывна, то $f_i^2$ равномерно непрерывна. $f_i'^2(\xi_{i, k})\leq \min_{[x_k, x_{k+1}]}f_i'^2+\varepsilon^2$ (для достаточно мелких разбиений и любого эпсилон, большего нуля). Тогда можно получить верхнюю оценку: $\leq \sum_{k=0}^{n-1}(x_{k+1}-x_k)\sqrt{\sum_{l=1}^n \min_{[x_k]}(f_l'^2)}+\varepsilon \sqrt{n}(b-a)\leq \int_a^b\sqrt{\dots}+\varepsilon \sqrt{n}(b-a)$ (устремляем разбиение к бесконечно малому). А затем делаем аналогично снизу и получаем требуемое равенство.

\section{Лекция 2.}

Пусть $\varphi$ - функция, которая определялась на прошлой лекции, а $\psi$ - обратная ей. $\psi$ - биекция, рассматриваем $f\circ \psi$. Посмотрим на $\psi([0, \beta])=[a,b]$, тогда для любых $c, d\in [0, b]$ $V_{f\circ\psi}([c, d])=d-c$.



Естественная параметризация гладкого пути практически не отличается от того, что мы уже рассматривали за одним небольшим исключением. 
\[
    \varphi(x)=V_f([a, x])=\int_a^x\vert f'(s)\vert ds=\int_a^x\sqrt{f_1'^2+\ldots+f_n'^2}ds,
\]
причём предпоследнее вырежение равно $\vert (f_1', \ldots, f_n')\vert$, а под корнем все функции от $s$. Рассмотрим опять $\psi$, и как выглядит вектор $f(\psi(x))=(f_1(\psi(x)), \ldots, f_n(\psi(x)))$, рассмотрим его производную, берём покоординатно: $f'(\psi(x))=(f_1'(\psi(x)), \ldots, f_n'(\psi(x)))$. Но $\psi'(x)=\frac{1}{\varphi(\psi(x))}$, тогда $\varphi(s)=\vert f'(s)\vert$, а также $\vert f'(\psi(x))\vert=1$.

\begin{remark}
    Если $f$ - гладкая на $[a, b)$ и существует $\int_a^b|f'(s)|ds$, тогда выполнено то же самое, просто $\varphi(x)=\int_z^x|f'(s)|ds$.
\end{remark}

Перейдём теперь к тригонометрии. Рассмотрим окружность $x^2+y^2=1$, мы планируем её обходить (то есть, через каждую точку по разу, с одинаковой скоростью, и так далее). Введём попутно также комплексное обозначение (мы не будем заниматься комплексным анализом, просто это удобно). Отождествим $\RR^2$ с $\CC$ понятно каким образом. Тогда какое вращение мы хотим? Мы хотим найти функцию $\Gamma:\RR\rightarrow \mathbb{T}=\{z:|z|=1 \text{ или } x^2+y^2=1, z=x+iy\}$, а хотим потребовать также следующее:

\begin{itemize}
    \item $\Gamma\in C^1$ (гладкая),
    \item $\Gamma(0)=1, \: \Gamma'(0)=i$ (место старта и начальная скорость, с которой мы идём), 
    \item $|\Gamma'(t)|=1$ для любого $t$ (постоянная скорость 1).
\end{itemize}

Сформулируем теорему:\\

\begin{theorem}
    Функция с данными свойствами существует и единственна.
\end{theorem}

\begin{proof}
    $\Gamma(t)\in\mathbb{T}$ тогда и только тогда, когда $\Gamma(t)\overline{\Gamma(t)}=1$. Продифференцируем последнее, получим
    \[
        \Gamma'(t)\overline{\Gamma(t)}+\Gamma(t)\overline{\Gamma'(t)}=0, 
    \]
    что также равно 
    \[
        2\Real(\overline{\Gamma'(t)}\Gamma(t))=0.
    \]
    То есть, мы получили, что $\Gamma(t)\overline{\Gamma'(t)}=ih(t)$, $h(t)\in\RR$. Применим теперь оставшееся неиспользованное условие: $|\Gamma(t)|=1$, а чтобы параметризация была естественна, $|\Gamma(t|$ должно быть равно 1. То есть, $h(t)=\pm1$. Подставим теперь нуль и получим, что функция в этой точке должна быть равна единице, а производная - $i$. Тогда остаётся один вариант: $ h(t)\equiv 1$. \ 

    Посмотрим теперь ещё раз на начальные уравнение: $\Gamma'(t)\overline{\Gamma(t)}\equiv i$, то есть, 
    
    \begin{equation}
        \Gamma'(t)=i\Gamma(t).
    \end{equation}
    
    Таким образом, мы уже пришли к тому, что если вращение существует, то оно должно удовлетворять последнему уравнению, а также $\Gamma(0)=1$. Это означает, что вращение, которое мы получаем, будет дифференцируемо бесконечно много раз. \ 

    Пока что, казалось бы, ни единственности, ни существования, однако из последних утверждений легко получается единственность. Пусть у нас есть $\Gamma_{1,2}$ - два простых вращения. Дначит, они оба удовлетворяют (1). Тогда завайте запишем их частное через сопряжённые и возьмём производную: $ \biggl(\Gamma_1(t)\overline{\Gamma_2(t)}\biggr)'=\Gamma_1'(t)\overline{\Gamma_2(t)}+\Gamma_1(1)\overline{\Gamma_2'(t)}$, что равно $i\Gamma_1\overline{\Gamma_2}+\Gamma_1\overline{i\Gamma_2}=0$. \ 

    Таким образом, мы получили, что $\Gamma_1\Gamma_2=\const$, но поскольку $\Gamma(0)=1$, то эта константа и равна единице. То есть, $\Gamma_1\overline{\Gamma_2}=1$, следовательно, эти функции равны, единственность доказана. \ 

    Докажем теперь существование. Предъявим сначала произвольную параметрицацию окружности, а затем постараемся сделать в ней замену переменной, чтобы получить хорошую функцию (которая должна быть, конечно, гладкой). Давайте параметризуем верхнюю половину $\mathbb{T}$ самым естественным образом: примем $x=t, \: y=\sqrt{1-t^2}, \ -1\leq t\leq 1$ (двигаемся по часовой стрелке). Теперь нам нужно отпараметризовать нижнюю половину, возьмём для этого $x=-t, \: y=-\sqrt{1-t^2}, \: -1\leq t\leq 1$, двигаться мы теперь будем по нижней половине, но в другом направлении, то есть, одну из половин нужно перевернутьт и ''склеить'' в один целостный проход. Тогда в нижней половине ''сдвинем'' рассмотрение на $1\leq t\leq 3$, и преобразуем: $y=-\sqrt{1-(2-t)^2}$. \ 

    Осталось проверить, что полученная функция гладкая. Вообще, это почти везде очевидно, кроме $\pm 1$, это и проверим. $f(t)=(t, \sqrt{1-t^2})$, а вектор $f'(t)=(1, \frac{-t}{\sqrt{1-t^2}})$. Функция $\varphi(x)$ на $(-1, 1)$ выглядит как
    \[
        \int_{-1}^x|f'(s)|ds=\int_{-1}^x\sqrt{1+\frac{t^2}{1-t^2}}dt=\int_-1^x\frac{dt}{\sqrt{1-t^2}}.
    \]
    Функция $\varphi(x)$ - возрастающая биекция, значит, мы можем посмотреть на обратную функцию $\psi(x)=\varphi^-1(x)$. Рассмотрим теперь для $x\in(-1, 1)$,
    \[
         (f^{-1}(\psi(x)))'=(f_1'(\psi(x))\psi'(x), f_2'(\psi(x))\psi'(x)).
    \]
    Тогда, так как $\psi'(x)=\frac{1}{\varphi'(\varphi(x))}$, это также и равно $\sqrt{1-\psi^2(x)}$, что также равно 
    \[
    (\psi'(x), \frac{-\psi(x)}{\sqrt{1-\psi(x)}}\sqrt{1-\psi^2(x)}).
    \]
    В последнем также можно сократить числитель и знаменатель. Итого, $f(\psi(x))$ - гладкая на $(-1, 1)$, и более того, если $x\rightarrow\pm 1$, производная имеет конечный предел. Получается, дифференцируема на интервале, и производная имеет предел в крайних точках, тогда она в них также дифференцируема. Таким образом, для верхней половины мы всё показали, для нижней - аналогично, всего лишь с линейной заменой. \ 


\end{proof}

После доказательства теоремы, можно, наконец, ввести определения:

\begin{defn}
    \[
        \cos(x)=\Real(\Gamma(x)), 
    \]
    \[
        \sin(x)=\Imf(\Gamma(x)).
    \]
\end{defn}

Далее уже можно поговорить о бесконечной дифференцируемомти и формуле Муавра, этим, вместе с доказательством, что мы нашли привычные функции, мы, кажется, и планируем заниматься далее.

\end{document}